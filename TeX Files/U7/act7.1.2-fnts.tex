\section{Check for understanding: 1-D Momentum}
\label{14C.1}
\note{For Activity~\ref{14C.1} (\about\unit[70]{min})}{
\subsection*{Learning Goals:}
\begin{itemize}
\item Get practice identifying momentum and change in momentum in simple 1-D collisions
\item Get practice identifying impulse
\item Get practice representing impulse, momentum, and change in momentum as vectors using the scaled-arrow representation
\item Get practice defining and analyzing collisions/interactions, representing conservation of momentum using both the scaled-arrow and vector-equation representations
\item Get practice drawing separate force diagrams for each object involved in the interaction
\end{itemize}
}

\begin{overview}
	\textbf{Overview:} In this section, we'll practice what we've learned so far about momentum, change in momentum, and momentum conservation.	
\end{overview}

\subsection{Collisions and Other Interactions between Objects}

\note{\FNT~\thechapter-\ref{fnt7.1.1-1} }{
\begin{itemize}
\item All groups should put up their responses to this \FNT along with the responses to the added problem.  
\item To expedite matters you can have Part C be a \WC{} discussion.
\end{itemize}
}
\note{For part b}{Have them write out $\vec{p}_{1,i }+ \Delta \vec{p}_{2\;on\;1} = \vec{p}_{1,f}$ and similarly for cart 2.  The emphasis here is that they realize that the change in momentum of cart 1 is due to the interaction with cart 2 (which we will link to net forces acting on cart 1, and they will see this link to forces when they draw their force diagrams for each cart during the collision/interaction).  Then they write an expression for conservation of momentum for the system, and then write it out in terms of the individual object�s initial and final momenta (m1v1,i + m2v2,i = m1v1,f + m2v2,f and if v2,i=0, then have them simplify, etc�).  }
\begin{FNTenv}
	\label{fnt7.1.1-1}

You're playing with two of the carts (each with mass $m$) that you used in \hyperref[act7.1.1c]{Activity~\ref*{act7.1.1c}}. Initially, these two carts are moving toward each other with the same initial speed $v_i$ along the track. The carts collide and the result is one of these final states:
\begin{enumerate}[(a)]
	\item Assume that the carts hit each other and stop so that the final state of the system has both carts just sitting still (not moving). Draw a \pchart{} for this situation. Make a separate row for each cart.% Refer to \pConsModel{} 
	\item Assume that the carts bounce off each other so that the final state of the system has each cart moving opposite to its initial motion but with the same speed. Draw a \pchart{} for this situation.
	\label{fnt7.1.1-1b}
	\item As in \eqref{fnt7.1.1-1b}, assume that the carts bounce off each other but now assume that the final speeds are smaller than the initial speeds, equal and in opposite directions. Draw a \pchart{}.
	\item For each case above, does the total momentum of the system that contains the two carts change? How do the \pcharts{} help you answer this question?
	\item Is the total kinetic energy constant for all three cases? How do you know?
\end{enumerate}

\end{FNTenv}

\note{\FNT~\thechapter-\ref{fnt7.1.1-2} and \thechapter-\ref{fnt7.1.1-3} }{Work and discuss these first, as these should be quick (simpler than the carts).  We will refer to these examples again in a few \DLMs{}  to bridge the connection to Newton�s laws.}
\begin{FNTenv}
	\label{fnt7.1.1-2}

A rocket expels gas at a high speed out of its back for a short period of time. We are going to treat the rocket as being far away from any gravitational objects.

\begin{enumerate}[(a)]
	\item Draw a \pchart{} for the rocket expelling gas in space. Take the initial time before expelling gas and the final time after the rocket has finished expelling gas. The rocket has an initial constant speed in the horizontal direction. Put the rocket and the expelled gas on separate rows.
	\item Use your chart to explain why the rocket speed increases. 
	\item Does the rocket have to keep expelling gas to stay at a constant speed? Explain.
\end{enumerate}
\end{FNTenv}

\begin{FNTenv}
	\label{fnt7.1.1-3}

Victoria is standing on a boat, during a perfectly calm day. Initially, both Victoria and the boat are not moving. Then Victoria walks from one end of the boat to the other. Take the initial time to be before she walks and the final time at some point while she is still walking.

\begin{enumerate}[(a)]
	\item Draw a \pchart{} for this situation. Does the boat move, and if so, in which direction?
	\item Compare the speed of the boat with Victoria's speed. Are they the same or different? Why?
\end{enumerate}
\end{FNTenv}

%\subsection*{In Your Small Group}
Compare your responses to \ref{fnt7.1.1-1}, \ref{fnt7.1.1-2}, and \ref{fnt7.1.1-3} with the other members of your small group. Come to a consensus on the appropriate \pcharts{} and answers, and put these on your board.

\textbf{Added problem:} Draw an appropriately scaled force diagram (for each object) that shows all the forces acting during the interaction (when the impulse occurs). Do this for each \FNT{}.
\note{Added Problem:}{You may need to remind the students of the 2 types of forces here: contact and long range. 
	They should label the force of the earth on the carts as $FE_{Earth on cart}.$  
}
\WCD

\subsection{More Collisions of Two Carts}

\noindent Your instructor will assign each group a situation from 1--3 below and one from 4--5. Use the \pConsModel{} to analyze each of the collisions between two carts in the Situations~1--3 below. You may treat the system made up of the two carts as a closed physical system because there is no net external impulse imparted on the system during the collision.\\

\noindent\textbf{Situations:}
\note{Situations:}{
\begin{itemize}
\item Since the students only have to do the 2 situations, they should put up charts for both.  Have all groups put up their momentum charts with properly scaled arrows.
\item Be sure students do not confuse velocity and momentum in situations 4 \& 5 (unequal masses).  The speed of the cart with the two masses is small after the collision, but its momentum is not!
\item Help the students realize that the change in momentum of an individual object determines the direction of the net force on it.  So, the ?p column in their momentum chart is where they should look to figure out the direction of the net force on the object.  Their force diagrams should also have the (canceling) vertical forces FEarth on cart1 and Ftrack on cart1.  Make sure they are labeling all their forces and labeling them correctly.
\item \WC:  Remember to stress that our assumption is that there is no external net impulse during the collision, so the total change in momentum is zero in each collision (modeling as closed with each cart being an object in the total system).
\end{itemize}
}
\begin{enumerate}
	\item Use two carts of equal mass, physically arranged so the carts will bounce off one another. Start with \textbf{\em one cart stationary}.
	\item Use two carts of equal mass, physically arranged so the collision ends with the \textbf{\em carts locked (stuck together)}.
	\item Use two carts of equal mass initially \textbf{\em moving toward each other} with equal speed and \textbf{\em ending with the carts locked (stuck together)}.
	\item Place two carts of \textbf{\em unequal mass} on the track, turned so the collision \textbf{\em ends with the carts locked (stuck together)}. Start with one cart stationary, and have the other move to collide with it. Make the stationary cart have more mass. Repeat, switching carts.
	\item Place two carts of \textbf{\em unequal mass} on the track, turned so the collision \textbf{\em ends with the carts locked (stuck together)}. Start with both carts \textbf{\em moving toward each other} with the same speed.
\end{enumerate}
\note{Remember:}{The momentum chart is a TOOL to use, it helps lead the students to write a conservation of momentum expression (vector equation) for the system:  $\vec{p}_{i, tot} = \vec{p}_{f, tot}$.  }
\noindent
\textbf{For each Situation:}
\begin{enumerate}[(a)]
	\item Draw and fill in a \pchart{} to help you describe momentum conservation in this closed physical system.
	\item For each line of the \pchart{}, write an algebraic vector equation, with appropriate subscripts on the symbols, to express what that line tells you about the collision.
	\item Draw a force diagram for each cart that shows the forces \emph{during} the collision.
	\item Describe in words what physically happened and then how conservation of momentum applies to each cart and to the system as a whole. Put your equation and word statement on the board.
	\item Compare the total kinetic energy before the collision to the total kinetic energy after the collision.% Classify each case as elastic or inelastic.
\end{enumerate}

\noindent Observe what is similar and what is different in your \pcharts{}. What patterns can you observe? Discuss in your group any rules you come up with for the patterns.

\WCD \note{WHOLE CLASS SHARING }{  Again, try to use student voices, but make sure the big ideas come across loud and clear.  You might need to do a bit of direct instruction here, depending on how much students have seen in lecture.}