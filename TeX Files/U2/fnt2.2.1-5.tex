\label{fnt2.2.1-5}

Christine drops a water balloon from the top of the Science Building. Let's assume that the balloon does not break when it strikes the ground. There are many questions we could ask about this situation. To answer any of them, it makes sense to model the Energy dynamics of the scenario first. Let's do that and then answer some particular questions!

\begin{enumerate}[(a)]
	\item \label{fnt2.2.1-5a} Create a particular model for this phenomenon by making an \EnergyDiagram{} for the process that takes place from the time the water balloon is dropped until it is motionless on the ground. Consider the indicators to determine what energy systems must be present/can be excluded. Have you included enough systems?
	\item If the water balloon falls a distance of \unit[21]{m}, what is the maximum temperature rise of the water balloon due to its being dropped? (Does you answer seem reasonable? Why or why not? It may help to check your units.)
	\item Is there anything that prohibits the water balloon from suddenly cooling off to its original temperature and leaping \unit[21]{m} into the air? From the random nature of thermal energy, why do you think we never see this happen? Respond {\em briefly}.
\end{enumerate}