The Universal Behavior of Oscillating Objects
A) Do:
Set up and observe how each of the following three systems behaves when the object is set into motion.
Systems:	1)	Pendulum:  A mass (hook weight) hanging on a string.
2)	Mass and Spring:  A mass (hook weight) hanging vertically on a spring.
3)	Metal meter stick: The meter stick is clamped to the table so 50 - \unit[90]{cm} protrudes from the table edge.
   Write: Each person must write out his or her own responses to these questions. You will need this for some \FNTs.  Respond to questions a)-d) below with a couple of sentences for each of the three systems without using physics language. Put these everyday language responses on the board and be prepared to share them with the whole class.

a)	What you had to do to get the object to move.
b)	What the resulting motion is like.
c)	What aspect of the motion you could change depending on how you started it moving and what aspect(s) of the motion did not appear to depend on how you started it moving.
d)	How the motion changed (or did not change) when you physically changed the system in the following ways (actually make the changes and observe them)
1)	Pendulum:		change the length of the string, by at least a factor of two
				change  the mass hanging on the string
2)	Mass and Spring:	change the mass, by at least a factor of two
3)	Meter stick:	change the length sticking out past the edge of the table
 e)	Discuss with your table partners what is common across the three systems about your responses to each of the four questions a)-d).
 
\WCD

 B) Think, Discuss, and Write:
	\item Rewrite on the board the response you just wrote in A) using the words amplitude, period, oscillation, and force, instead of some of the everyday words you used in A).  
	\item Write out on the board a definition of each of these four words.

\WCD

 C) Think, Discuss, and Draw:
a)	Sketch a graph on the board of the y-position coordinate for the mass spring system as a function of time when it is oscillating.  Let y = 0 correspond to the equilibrium position of the mass.  Show on your graph the amplitude, A, and period, T.  (Leave graphs on board for next activity)
b)	On the same axis, draw a second graph (use a dashed line) that corresponds to starting the oscillation with a different pull.
c)	On the same axis, draw a third graph (using a different length dash) that corresponds to using the same pull as the first, but with a much smaller mass.
d)	Describe in technical words the similarities and the differences in the three graphs you just made.

\WCD
