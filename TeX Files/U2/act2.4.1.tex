\section{Connecting Potential Energy with Forces}
\label{act2.4.1}
\note{The Connection between Force and Potential Energy}{You can move this activity along very quickly if you run it as hybrid Small Group/Whole Class activity. Get them up to the board, and herd the entire class through one letter at a time. Don�t give them more than a minute or two for each letter prompt. Have someone show/explain the answer, and move right along to the next one. This works pretty well for this activity because most of the questions are short and straightforward. 
Another thing you can do if you are short on time is to have them read and discuss the question for a minute, and then tell you what to put on the board (In effect, you can act as their scribe.)
}

\begin{overview}

\textbf{Overview:} We have studied two different potential energy systems so far,  $PE_\text{gravity}$ and $PE_\text{mass-spring}$. Our goal in this section is to understand how the forces associated with these potential energy systems are related to the graph of Potential Energy as a function of position.

\end{overview}

\note{General Notes}{The goal of this instructional cycle is to help students develop an understanding of the connection between the potential energies that exist between atoms and molecules and the forces that exist between those atoms and molecules.  Understanding the pair-wise potential associated with two particles is crucial to understanding many points in Unit 3.  

This activity focuses on the relation of force to the shape of the potential energy curve in general.  (We know this relation as:  �force is the negative gradient of the potential� or $F(r) = -\frac{\partial U}{\partial r}$, but all the students need to recognize by the end of this activity is $\left|F\right| = \left|\frac{d(PE)}{dr}\right|$ and the direction of force is in the direction of decreasing $PE$. 
}
\subsection{Graphs for Gravitational Potential Energy}
\label{act2.4.1-1}
\note{Graphs for gravitational $PE$}{
2) $PE = mgy$. 
3)	Arrow should be pointing to the left (in the direction of decreasing $PE$). 
4) Near the surface of the earth he magnitude is constant, and the direction is towards decreasing $PE$. 
}
\begin{enumerate}
	\item Sketch a graph of $PE_\text{gravity}$ vs.\ the vertical position $y$ of a ball that is thrown upward from the surface of the Earth. Let $y = 0$ be at the ground.
	
		\textbf{Note:} Be sure to plot $PE$ on the vertical axis and $y$ on the horizontal axis.
	
	\item Next to your graph, write a mathematical expression for $PE_\text{gravity}$ as a function of $y$.
	
	\item Consider a particular value of $y$ that is greater than $y_{min}$ and less than $y_{max}$. At your value of $y$, what is the direction of the force of gravity on the ball (in the direction of \emph{decreasing $y$} or \emph{increasing $y$})?
	
		Put an arrow on your graph showing the direction of the force. Repeat for a different value of $y$. How do the magnitudes of the forces at these two points compare?
	
	\item What, in general, can you say about the magnitude and direction of the force at different $y$ values?
\end{enumerate}

\subsection{Graphs for Mass-Spring Potential Energy}
\label{act2.4.1-2}
\note{Graphs for mass-spring $PE$}{
2) $PE = \frac{1}{2} k x^{2}$. 
3)	Arrows should be pointing in the direction of decreasing $PE$. 
4) The magnitude is larger for larger values of $x$, and the direction is towards decreasing $PE$.  
}
\begin{enumerate}
	\item Sketch a graph of $PE_\text{mass-spring}$ vs.\ the distance $x$ of a mass from its equilibrium position in an oscillating mass-spring system. Remember that for a mass-spring system, we always set the origin of the coordinate system at the equilibrium point.
	
	\item Next to your graph, write an algebraic expression for $PE_\text{mass-spring}$ as a function of $x$.
	
	\item Consider a particular value of $x$ that is greater than zero and less than $x_{max}$. At your value of $x$, what is the direction of the force on the mass (\emph{away} from equilibrium or \emph{toward} equilibrium), and is that in the direction of \emph{increasing $PE$} or \emph{decreasing $PE$}?
	
		\textbf{Note:} You can use the mass-spring system at your table to test this. Pull the mass down \unit[\about2]{cm} from equilibrium, and then push it up \unit[\about2]{cm} from equilibrium.
		
		Put an arrow on your graph showing the direction of the force. Repeat for a different value of $x$. How do the magnitudes of the forces at the two points compare?
	
	\item What, in general, can you say about the magnitude and direction of the force at different $x$ values?
\end{enumerate}

\subsection{The Big Idea}

\note{The Question}{Based on this activity, they can only conclude that the magnitude of the force is proportional to the magnitude of the potential energy, and the direction of the force is in the direction of decreasing $PE$. Assure them that this is always true, and that the proportionality is an equality, i.e., $\left| F\right| = \left| \frac{d(PE)}{dr}\right|$. Point out that this is relationship 2 in the Particle Model of Matter. Remind them to keep looking at the blue foldouts, which have the key constructs and relationships, of each model in their textbook! }
\begin{overview}

\textbf{Remember our goal:} We want to relate our observations about the gravitational forces in \ref{act2.4.1-1} and the mass-spring forces in \ref{act2.4.1-2} to features of the graphs of the potential energy systems associated with these forces. That is, we want to find a relationship that \textbf{holds true for both forces}.
	
\end{overview}

\noindent This relationship will need to relate \textbf{both} the \textbf{magnitude} of the force and the \textbf{direction} of the force to \textbf{specific features} of the \textbf{graphs} of the respective potential energies. The two features of the graphs we will focus on are the \textbf{instantaneous slope} of the graph of $PE(y)$ or $PE(x)$, and the \textbf{change in $PE$} as the position changes.

\begin{enumerate}
	\item Can you find any correlation between the magnitude of the force and the magnitude of the instantaneous slope of the $PE$ for these two forces?
	
	\item Can you find any correlation between the direction of the force and the direction of decreasing $PE$ for these two forces?
	
	\item Come to a group consensus about how to state this relationship most succinctly and clearly in words. \textbf{Write this statement on the board.}
\end{enumerate}

\subsubsection*{``Physicality Check''}

When you use the feature you found above to predict a force from the potential energy vs.\ distance graphs, do your results make sense physically?  For example, do the directions and magnitudes of the forces found at various points on the graph agree with what you know about the force?  Try this with various points on both sets of graphs. Be ready to demonstrate this for the whole class.\\

\WCD