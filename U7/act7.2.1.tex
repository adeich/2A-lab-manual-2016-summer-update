\section[2D- and 3D-Momentum and Change in Momentum]{Momentum and Change in Momentum in Two and Three Dimensions}

\begin{overview}
	\textbf{Overview:} So far, we have only concerned ourselves with momentum in one dimension -- which means we've only looked at movement in the ``forward/backward'' directions. However, as you know, we can also move left and right, as well as up and down. In order to consider most real-life phenomena, we have to expand our ideas about momentum to those additional dimensions.
\end{overview}

\subsection{Impulse in Two Dimensions}

\begin{fnt}
	\label{fnt7.2.1-1}

\begin{wrapfigure}{R}{0.25\textwidth}
	\vspace{-20pt}
  	\centering
	\begin{center}
	\begin{tikzpicture}[thick,scale=0.9, every node/.style={transform shape},background rectangle/.style={fill=white}, show background rectangle]
		% draw xy-axes
		\draw[-{Stealth[scale=1.2]}, line width=0.5pt] (0,-0.5) -- (0,5) node[left=6pt,align=center] {$y$};
		\draw[-{Stealth[scale=1.2]}, line width=0.5pt] (-0.5,0) -- (3,0) node[below=6pt,align=center] {$x$};
		
		% draw F1
		\draw[-{Stealth[scale=1.2]}, line width=1.5pt] (0,0) -- (1.15,0.67) node[above=3pt,align=center] {$\vec{F}_1$};
		% Label F1 = 5N
		\node[label={[rotate=30]above:\unit[5]{N}}] at (0.577,0.333) {};
		% draw F1 arc
		\draw[line width=0.5] (0.6,0) arc (0:30:0.6);
		% label F1 arc 30deg
		\node[label={above:30\textdegree}] at (1.25,-0.25) {};
		
		% draw F2
		\draw[-{Stealth[scale=1.2]}, line width=1.5pt] (0,0) -- (0,4) node[right=6pt,align=center] {$\vec{F}_2$};
		% label F2
		\node[label={[rotate=90]left:\unit[15]{N}}] at (-0.125,2.5) {};
		
		% draw and label v_i
		\draw[->] (1.5,3) -- (2.5,3);
		\node[label={center:$\vec{v}_i=\unitfrac[3]{m}{s}$}] at (2,2.5) {};
	\end{tikzpicture}
	\vspace{20pt}
	\end{center}
\end{wrapfigure}

\noindent\textbf{Note:} This is an extension of \ref{fnt6.1.2-6}. Two force vectors ($\vec{F}_1$ and $\vec{F}_2$, as shown to the right) act on a \unit[2]{kg} object that has an initial velocity $\vec{v}_i$ of \unitfrac[3]{m}{s} in the $+x$-direction.

\begin{enumerate}[(a)]\setcounter{enumi}{2}
	\item Use the $x$- and $y$-components of the force you found in (a) to determine the $x$- and $y$-components of impulse that would act on the \unit[2]{kg} object if the forces were applied for a time interval of \unit[0.50]{s}. Always include units with your answers. Start with the relationship of impulse and force. Find each component of impulse separately.
	
	\item Find separately, for each component, the change in velocity of the \unit[2]{kg} object, due to the impulse from (c).
	
	\item Find the magnitude of the velocity of the object after the impulse has acted and the direction the velocity vector makes with the positive $x$-axis.
\end{enumerate}
\end{fnt}

\noindent Compare with your group your responses to \ref{fnt7.2.1-1}. Come to a consensus and put your response on the board.

\WCD

\subsection{A Mass Swinging in a Horizontal Circle}
\label{MassCircle}

\textbf{Phenomenon:} You are going to observe, talk about, and analyze a three dimensional problem. This will give you practice working with vectors and the concept of impulse in more than one dimension and practice determining the net force.

\begin{center}
	\begin{tikzpicture}[decoration={markings,mark=at position 0.5*\pgfdecoratedpathlength-25pt with {\arrow[thick]{<}},mark=at position 0.5*\pgfdecoratedpathlength+5.5cm+5pt with {\arrow[thick]{<}}}]
		\node[inner sep=0pt] (fingers) at (-.5,3) {\includegraphics[width=1cm]{pinchedFingers.png}};
		\draw[postaction={decorate},dashed] (0,0) ellipse (3cm and .8cm);
		\draw (-.18,2.82) -- (1.95,.65);
		\draw[fill=gray] (2,.6) circle (.1);
	\end{tikzpicture}
\end{center}

\noindent Hold the ball and swing it so it revolves in a horizontal circle as in \hyperref[act6.1.3]{Activity~\ref*{act6.1.3}}. Focus on a small section of the arc of the circle. Put your responses to the following parts on the board and be prepared to discuss them with the whole class.

\begin{enumerate}
	\item If we treat the moving mass (just the ball) as our physical system, is there a net impulse on this physical system? How do you know? (Hint: Is the momentum changing?)
	
	\item What must be the direction of the net impulse acting on the ball? How do you know this from the motion? Explicitly show the vectors $\vec{v}_i$, $\vec{v}_f$, and $\Delta \vec{v}$ on a diagram, drawn from above (looking down on the motion).
	
	\item In which direction must the net force $\Sigma \vec{F}$ be? How do you know this from the motion?
	
	\item
	\parbox[t]{\dimexpr\textwidth-\leftmargin}{%
	      \vspace{-3mm}
	      \begin{wrapfigure}[7]{r}{3cm}
	        \centering
%	        \vspace{-\baselineskip}
		\begin{tikzpicture}{r}{2cm}
			\draw (-.18,2.82) -- (1.95,.65);
			\draw[fill=gray] (2,.6) circle (.1);
		\end{tikzpicture}
		\end{wrapfigure}
	
	 Analyze the forces acting on the revolving ball. What objects exert forces on the ball? Remember, these could be contact forces or long-range forces. Now think about a side view of the revolving ball (see figure at right). Use what you know about the directions of these forces and the direction of $\Sigma \vec{F}$ to draw a force diagram for the ball in this side view. Be sure to label all forces with two subscripts and make their lengths appropriately scaled with respect to each other. Show the net force separately with a double arrow.
	 }
	
	\item Write a few sentences explaining why the ball revolves in a horizontal circle in terms of momentum, impulse and net force.
	
	\item Still treating our physical system as the moving ball, is mechanical energy conserved -- that is, does the mechanical energy of the system remain constant? How do you know? You should answer this in terms of energy systems changing or not changing, as well as whether energy is transferred as work.
	
	\item Is the net force more closely related to $\vec{v}_i$, $\vec{v}_f$, or $\Delta \vec{v}$?
\end{enumerate}

\WCD