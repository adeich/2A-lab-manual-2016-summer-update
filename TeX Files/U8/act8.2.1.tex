\section{Using the \FModel{} to Analyze the Motion of Dropped Objects}
\label{act8.2.1}

\textbf{Phenomenon:} Dropping a wadded up piece of paper and a flat bottomed coffee filter

\begin{enumerate}
	\item Observing the Motion. Do (a) and (b) as a group, then put up (c) and (d) on the board.
	
	\begin{enumerate}
		\item Obtain a flat-bottomed coffee filter that still has it original shape. Wad up a piece of paper until it is in the shape and size of a golf ball. Drop both simultaneously from a height of about six feet.
		
		\item Carefully observe the motion of the two objects.
		
		\item Write out a description of the motion you observed using the terminology ``velocity.''
		
		\item Sketch a velocity graph on the board (only the vertical component) for each object on the same set of velocity versus time axes.
	\end{enumerate}

\WCD

	\item Analyzing the Motion. Discuss (a), (b), and (c) as a group while working at the board.
	
	\begin{enumerate}
		\item Create \forcediags{} for both objects at two times, 
		\begin{enumerate}
			\item first at the instant each object has been released and
			\item second after each has fallen about \nicefrac{1}{4} of the way to the floor.
		\end{enumerate}
		Show the net force vector explicitly (use double lines). What does this tell you about the accelerations of these two objects as they fall? Apply Newton's 2nd law to both objects in order to determine their accelerations.
		
			\textbf{Put all of this on the board.}
		
		\item Use the accelerations you obtained in (a) along with the actual motions you observed to make an acceleration graph ($\vec{a}$ versus time) for the two dropped objects from the time they are released to just before hitting the floor.
		
			\textbf{Put these graphs on the board.}
		
		\item What is the mathematical relation between the instantaneous acceleration and the instantaneous velocity? How does this relation show up on the two graphs you made for the two cases? Adjust your curves to be consistent with both (b) and (c).
	\end{enumerate}
\end{enumerate}

\textbf{\em Be prepared to share your responses with the whole class.}

\WCD
