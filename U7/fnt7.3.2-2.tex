\label{fnt7.3.2-2}

A Physics 2A student runs along a line tangent to the edge of a motionless merry-go-round and jumps on at the very outside. The merry-go-round has the shape of a uniform disk. 

a)  Make an angular \pchart{} to help you keep track of what is and is not changing.  Consider the two objects, the student and merry-go-round, with the total angular momentum of these two objects conserved.  Must the student have angular momentum before jumping on the merry-go-round?  To help you understand what is happening in this problem it will be useful to i) refer to the chart of the analogs of rotational variables to linear variables in the Course Notes and maybe ii) go to someone's office hours.
b)  Is linear momentum conserved in this interaction if you just consider the merry-go-round and the student (make a \pchart{} to help you answer this)?