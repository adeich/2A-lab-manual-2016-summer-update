\note{Brief Overview}{
	The first two \CLASP{} Activities in this \DLM{} continue \hyperref[\DLM2]{\DLM\ref*{\DLM2}} with the qualitative study of both the \ThreePhaseModel{} and the \EnergyInteractionModel{}. They extend the work you did on the \FNTs{} assigned at the end of \hyperref[\DLM2]{\DLM\ref*{\DLM2}}.
	
	The last \CLASP{} Activity begins \texttt{Module 1.2}: Getting Quantitative with Models.
	
	\subsubsection*{If you run out of time to complete DL Activities}
	
		In general, for any unfinished DL activities from the previous DL (which should have been assigned as an \FNT{}), give a brief summary (\textless \unit[5]{min}) at the beginning of the current DL and sum up the main points for them. Don't spend time in DL as you normally would having the students discuss and explain.
	
	\subsubsection*{Instructor notes}
	
		Read the instructor notes before the TA meeting!! They contain not only answers (to most of the questions), but also information about what the students should be getting out of the activity, and, at times, instructions on how to run the activity. Be prepared to ask questions in the DL meeting to clarify anything you are not sure of, e.g., ``What is important for the students to get from this?''
		
	\subsubsection*{\FNTs}
	
		This is the first time the students will see what they are expected to do with the \FNTs{}. Specifically, they all should have worked hard on them outside of class. Then, in their small groups, they will have the opportunity to clear up their uncertainties about making \EnergyDiagrams{} and using them to make sense of physical phenomena. You should ``check off'' the \FNTs{} or in some way note which of your students have put in an acceptable amount of effort working on them and which have not. Students need to know that you are keeping a record of whether they are doing their homework or not. What's important is that they work on the \FNTs{}, not that they ``get them exactly right'' when they first do them.
		
	\subsubsection*{Explanations on the board}
	
		Students are often asked to write explanations in words on the board. Encourage them to outline, abbreviate, paraphrase, use sentence fragments, etc. in order to reduce the time required to write those explanations. In other words they don't have to be quite as careful in producing explanations for \WC{} consumption as they should on exams, because they can easily clarify in the \WC{} discussion.
	
	\subsubsection*{No equipment needed for \DLM03}
}

\note{Reminder}{The goal in this course is ``making sense,'' not getting answers.}

\section{Practicing Our Two Models}
\label{act1.1.4}

\begin{overview}

	\noindent
	{\bfseries Overview:} We continue to practice using the \ThreePhaseModel{}, and we'll start to get more familiar with the \EnergyDiagrams{} of the \EnergyInteractionModel{}.

\end{overview}


\note{Timing: \unit[65]{min}}{

	\subsubsection*{Purpose}
		\begin{itemize}
			\item To become more familiar with the \ThreePhaseModel{} and the \EnergyInteractionModel{}, the meaning of the model constructs and the diagrammatic representations of the models
			\item To become more familiar with applying these two models to particular thermal processes
		\end{itemize}
	
	\subsubsection*{Learning Outcome}
		\begin{itemize}
			\item Be able to quickly and confidently apply both the \ThreePhaseModel{} and the \EnergyInteractionModel{} to the kinds of phenomena treated in this activity. This means being able to confidently use the diagrammatic representations of both models to develop explanations and answer specific questions related to these kinds of thermal phenomena.
		\end{itemize}
}

\subsection{Basics of the \ThreePhaseModel{}}

\begin{FNTenv}
	\input{U1/FNT1.1.1-1}
\end{FNTenv}
\note{}{
	Get students started checking their responses to this \FNT{} as they come into the room.
	
	Before the \WC{} discussion, ask each group to put the response to one of the prompts in \#2 of the Activity on the board. Then in the \WC{} discussion, ``randomly'' call on someone from each group to explain how they knew how to respond to that prompt. Emphasize that they really need to know how to use this modeling tool: the \TempGraph{}.
	
	You might also call on someone to explain anything in part \#1 if you feel it is necessary.
}

\noindent
You've all done this \FNT{} individually at home. Now you have an opportunity to practice creating scientific arguments together. Back up your claims with evidence and try to convince each other of \emph{your} solution!

\begin{enumerate}
	\item Discuss with your group and make sure \emph{every} group member is confident that she/he can explain \hyperref[\FNT1.1.1-1A]{Question~\ref*{\FNT1.1.1-1A}} on the \FNT{}.
	
	\item While there is not one single correct answer for \hyperref[\FNT1.1.1-1B]{Parts~\ref*{\FNT1.1.1-1B}} -- \hyperref[\FNT1.1.1-1D]{\ref*{\FNT1.1.1-1D}} of the \FNT{}, you probably estimated the values somewhere in the vicinity of those in the box. Work out in your small groups any significant discrepancies you might have. EVERYONE in your group should be ready to explain how you obtained these results in the whole class discussion. Your instructor will tell your group which prompt you should respond to on the board.
\end{enumerate}

\begin{ans}
	\begin{enumerate}[(a)]
	  \setcounter{enumi}{1}
		\item liquid at \unit[$\sim$350]{K}
		\item completely solid at \unit[273]{K}
		\item $\sim$1/3 gas, $\sim$2/3 liquid at \unit[373]{K}
		\item initial conditions: all gas at \unit[373]{K}, final conditions: liquid at \unit[$\sim$50]{\textdegree C}; $\Delta$E~$\approx$~\unit[2470]{kJ}
		\item initial conditions: $\sim$25\% liquid at \unit[0]{\textdegree C}, final conditions: gas at \unit[100]{\textdegree C}; $\Delta$E~$\approx$~\unit[3000]{kJ}
	\end{enumerate}
\end{ans}


\WCD

\subsection{Basics of the \EnergyInteractionModel{}}

%\noindent
%{\bfseries Overview:} Using the \EnergyInteractionModel{} to describe simple processes.

\begin{FNTenv}
	\input{U1/FNT1.1.3-1}
\end{FNTenv}
\note{}{
	{\em Stress} to the \WC{} that they should have little trouble with these. They need to look at ``General Process of Constructing an \EnergyDiagram{}'' and quickly compare with each other. Give no more than \unit[5]{min} to do their comparing.
	
	You can ask if there are any issues the entire \SG{} is unsure of, and then respond to the \WC{} very briefly.
}

\begin{enumerate}
	\item Compare your responses for the processes in \hyperref[act1.1.3]{Activity~\ref*{act1.1.3}}, both the \TempGraphs{} and the \EnergyDiagrams{}.
	
	\item Your instructor will tell you which one of the scenarios (a) through (g) to put on the board.
		
	Put the \TempGraph{} and the \EnergyDiagram{} near each other. Make sure the two diagrams are consistent with each other and make sure everyone in the group can explain precisely why both are drawn the way you have them.

\WCD

\end{enumerate}
