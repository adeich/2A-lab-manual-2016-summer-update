\section{Exploring \motiongraphs{}}
\label{act8.2.2}
\note{For Activity~\ref{act8.2.2} (\about\unit[50]{min})
}{
\subsection*{Learning Goals:}
\begin{enumerate}
\item Get practice using Newton�s 2nd law approach to make sense of the real behavior of dropped objects
\item Practice extracting meaning from velocity and acceleration graphs
\end{enumerate}
NOTE:  This activity may take awhile if it is not managed well but you should have time in this \DLM{} to complete it.  Let the students explain in their own words what they have on the boards during the \WC{} discussion.
}
\begin{overview}

\textbf{Overview:} \motiongraphs{} are useful tools for analyzing, well, motion. We will now practice generating and interpreting these graphs for a variety of situations.

\end{overview}
\note{\FNT~\thechapter-\ref{fnt8.2.1-1}
}{For graph c they need to realize that a has same sign (direction) as the $\Delta v$ vector.  Let them write their example and then during the \WC{} they can explain it; if corrections need to be made ask them questions about their example that lead them to see for themselves what the correction is.
}
\begin{FNTenv}
	\label{fnt8.2.1-1}

State whether the acceleration is positive, negative, or zero for each of the position functions $\vec{x}(t)$ in the position versus time graphs below. How do you know?

\begin{center}
	\begin{tikzpicture}[thick,scale=0.63, every node/.style={transform shape},background rectangle/.style={fill=white}, show background rectangle]
		%Diagram (a)
		% draw label
		\draw (2.5,3.5) node {(a)};
		% draw xy-axes
		\draw[-{Stealth[scale=1.2]}, line width=0.5pt] (0,-0.5) -- (0,3) node[left=6pt,align=center] {$x$};
		\draw[-{Stealth[scale=1.2]}, line width=0.5pt] (-0.5,0) -- (5,0) node[below=6pt,align=center] {$t$};		
		% draw curve
		\draw[line width=1.5pt] (0.5,-0.25) -- (4,2.25);
		
		%Diagram (b)
		% draw label
		\draw (8.5,3.5) node {(b)};
		% draw xy-axes
		\draw[-{Stealth[scale=1.2]}, line width=0.5pt] (6,-0.5) -- (6,3) node[left=6pt,align=center] {$x$};
		\draw[-{Stealth[scale=1.2]}, line width=0.5pt] (5.5,0) -- (11,0) node[below=6pt,align=center] {$t$};		
		% draw curve
		\draw[line width=1.5pt,domain=6.5:10,smooth,variable=\x] plot ({\x},{.07*(\x-6)*(\x-6)+.5});

		%Diagram (c)
		% draw label
		\draw (14.5,3.5) node {(c)};
		% draw xy-axes
		\draw[-{Stealth[scale=1.2]}, line width=0.5pt] (12,-0.5) -- (12,3) node[left=6pt,align=center] {$x$};
		\draw[-{Stealth[scale=1.2]}, line width=0.5pt] (11.5,0) -- (17,0) node[below=6pt,align=center] {$t$};		
		% draw curve
		\draw[line width=1.5pt,domain=12.5:16,smooth,variable=\x] plot ({\x},{-.08*(\x-12)*(\x-12)+2});

		%Diagram (d)
		% draw label
		\draw (20.5,3.5) node {(d)};
		% draw xy-axes
		\draw[-{Stealth[scale=1.2]}, line width=0.5pt] (18,-0.5) -- (18,3) node[left=6pt,align=center] {$x$};
		\draw[-{Stealth[scale=1.2]}, line width=0.5pt] (17.5,0) -- (23,0) node[below=6pt,align=center] {$t$};		
		% draw curve
		\draw[line width=1.5pt] (17.75,2) -- (22,-.25);
	\end{tikzpicture}
\end{center}

\end{FNTenv}

\note{\FNT~\thechapter-\ref{fnt8.2.1-2}}{
Each group should be assigned to a different scenario.  The plots will vary greatly depending on the scenario they come up with.  Your task (and the task of the group 1) is to check for consistency in 
\begin{enumerate}
\item the situation verbally described and what is plotted, 
\item the derivative graphs, 
\item the equations and models.   Is it all put together well?  Only be a little picky with changing slopes.  Mainly, make sure they can explain and basically draw slopes related to plotting the derivative.  
\end{enumerate}
}
\begin{FNTenv}
	\label{fnt8.2.1-2}

For the each of the following scenarios make a position vs.\ time graph.  Directly below it draw a velocity vs.\ time graph, and beneath that draw acceleration vs.\ time. 

\begin{enumerate}[(a)]
	\item A dropped object as it is falling (before it hits the floor). 
	\item A rocket firing its engines for a certain length of time descending on Mars.
	\item A racecar during the first 10~seconds after it starts from a stop.
\end{enumerate}
\end{FNTenv}


\begin{FNTenv}
	\label{fnt8.2.1-3}

Underneath the appropriate column of graphs from \ref{fnt8.2.1-2}, write an equation that solves for the variable in question (see below).  Write which model you used: \FModel{}, \EnergyInteractionModel{}, the \pConsModel{}, %the \LConsModel{}, 
etc... If you introduce any new variables, clearly indicate what they mean.

\begin{enumerate}[(a)]
	\item The velocity of a dropped object just before it hits the floor. 
	\item The time it takes the dropped object in (a) to reach the floor after being dropped.
	\item The change in velocity for a spacecraft firing its rockets for a certain length of time.
	\item The speed of the race car after the first 10~seconds.
\end{enumerate}
\end{FNTenv}
\note{\FNT~\thechapter-\ref{fnt8.2.1-3}}{
The models they should have associated with the scenarios in \FNT\thechapter-\ref{fnt8.2.1-2} are respectively, 
\\[0.25in]
a) Conservation of energy\\
b) Newton�s 2nd law \\
c) Momentum conservation\\
d) Impulse or 2nd Law 
}
%\begin{FNTenv}
%	\label{fnt8.2.1-4}

Interpret the graph below as representing the velocity of an object. Rank the points in order of increasing acceleration (from most negative to most positive). Practice walking this plotted motion.

\def\FunctionF(#1){0.02*(#1)^5-0.007*(#1)^4-0.051*(#1)^3-0.49*(#1)^2+0.235+1/(((#1)-2.55)^2+0.52)}%

\begin{center}
	\begin{tikzpicture}[background rectangle/.style={fill=white}, show background rectangle,scale=0.9, every node/.style={transform shape}]


	\begin{axis}[
		axis y line=center,
		axis x line=middle,
		inner axis line style={-Stealth},
		axis on top=true,
		xlabel=$t$,
		ylabel=$v$,
        		every axis x label/.style={at={(current axis.right of origin)},anchor=north west},
	        every axis y label/.style={at={(current axis.above origin)},anchor=north east},
		xmin=0,
		xmax=3.1,
		ymin=-.6,
		ymax=.6,
		xticklabels={,,},
		yticklabels={,,},
		ticks=none,
		height=8cm,
		width=15cm
		]
		\addplot [domain=-.05:3.05, samples=50, mark=none, very thick] {\FunctionF(x)}
			node[pos=.05,circle,fill=black,scale=0.4] {}
			node[pos=.05,pin={[inner sep=1pt,pin distance=2mm]45:$A$},inner sep=2pt] {}
			node[pos=.2,circle,fill=black,scale=0.4] {}
			node[pos=.2,pin={[inner sep=1pt,pin distance=2mm]45:$B$},inner sep=2pt] {}
			node[pos=.31,circle,fill=black,scale=0.4] {}
			node[pos=.31,pin={[inner sep=1pt,pin distance=2mm]45:$C$},inner sep=2pt] {}
			node[pos=.345,circle,fill=black,scale=0.4] {}
			node[pos=.345,pin={[inner sep=1pt,pin distance=2mm]225:$D$},inner sep=2pt] {}
			node[pos=.42,circle,fill=black,scale=0.4] {}
			node[pos=.42,pin={[inner sep=1pt,pin distance=2mm]225:$E$},inner sep=2pt] {}
			node[pos=.53,circle,fill=black,scale=0.4] {}
			node[pos=.53,pin={[inner sep=1pt,pin distance=2mm]225:$F$},inner sep=2pt] {}
			node[pos=.6,circle,fill=black,scale=0.4] {}
			node[pos=.6,pin={[inner sep=1pt,pin distance=2mm]270:$G$},inner sep=2pt] {}
			node[pos=.66,circle,fill=black,scale=0.4] {}
			node[pos=.66,pin={[inner sep=1pt,pin distance=2mm]315:$H$},inner sep=2pt] {}
			node[pos=.74,circle,fill=black,scale=0.4] {}
			node[pos=.74,pin={[inner sep=1pt,pin distance=2mm]315:$I$},inner sep=2pt] {}
			node[pos=.82,circle,fill=black,scale=0.4] {}
			node[pos=.82,pin={[inner sep=1pt,pin distance=2mm]315:$J$},inner sep=2pt] {}
			node[pos=.875,circle,fill=black,scale=0.4] {}
			node[pos=.875,pin={[inner sep=1pt,pin distance=2mm]90:$K$},inner sep=2pt] {}
			node[pos=.93,circle,fill=black,scale=0.4] {}
			node[pos=.93,pin={[inner sep=1pt,pin distance=2mm]110:$L$},inner sep=2pt] {}
			node[pos=.99,circle,fill=black,scale=0.4] {}
			node[pos=.99,pin={[inner sep=1pt,pin distance=2mm]135:$M$},inner sep=2pt] {}
		;
	\end{axis}
	\draw (-0.25,3.24) node[] {0};
	\end{tikzpicture}
\end{center}
%\end{FNTenv}

\begin{center}\noindent\textbf{Each group is responsible for putting one of the following on the board.\\ Write so that all other groups can easily follow your presentation!}\end{center}

\noindent\framebox[1.1\width][c]{\textbf{Group 1}}

	\begin{enumerate}
		\item Decide on your group's answer to \ref{fnt8.2.1-1}, redraw each graph and explain next to it how you know whether the $x$-component of the acceleration is positive, negative, or zero.
		\item In each case, is the $x$-component of the velocity positive or negative?
		\item Does a negative acceleration always mean a decreasing speed? Come up with a relationship between the sign of acceleration and the change in velocity and write this on the board.
		\item Go to the boards of the groups working on \ref{fnt8.2.1-2}. Check each group's answers and graphs and assign them a ``grade'' for their work on \ref{fnt8.2.1-2}; seriously do this. Grade their graphs taking into account the initial points, the signs of the slope of the line in each graph, and how the slope of the line in each graph changes with time. Don't give them a low grade just because their answer doesn't look like your answer. Only give them a low grade if their answer is an unreasonable one for the problem at hand.
	\end{enumerate}
	
\noindent\framebox[1.1\width][c]{\textbf{Groups 2, 3, \& 4}}\\
	
\noindent For your group's assigned scenario from \ref{fnt8.2.1-2}, make a column of three graphs with the top most being position vs.\ time and the bottom most acceleration vs.\ time. Line up your graphs so that a vertical line through all three would correspond to the same instant in time in each.\\
	
\noindent\framebox[1.1\width][c]{\textbf{Group 5}}\\
	
\noindent Go to the boards of the groups working on \ref{fnt8.2.1-2} and answer \ref{fnt8.2.1-3} by writing the appropriate equation and naming which model is used.\\
	
\noindent\framebox[1.1\width][c]{\textbf{All Groups}}\\
	
\noindent What kinds of questions can the \FModel{} answer that the conservation models cannot answer? Explain why. (Use what you have learned from lecture and/or the Course Notes to help you answer this question.) Give two specific examples.\\

\note{}{In the \WC{} discussion ask the students what is similar about the energy interaction model and the momentum conservation models in terms of the kinds of questions they can answer and the kinds of questions they can�t answer.  Have them explain why the models can or cannot answer these questions.  Have them give examples of both kinds of questions.
	\\[0.25in]
	Both energy interaction and momentum conservation models look at changes in energy systems, or changes in momentum due to impulses.  Only the Newtonian mechanics model includes information about the time and type of interaction during a change of state.   }
\WCD
