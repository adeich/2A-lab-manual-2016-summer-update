\section[Representing the Motion of a Mass along a Circle]{Using Vectors to Represent the Motion of a Mass Moving in a Circle}
\label{act6.1.4}

\begin{overview}

\textbf{Overview:} So far, we've only talked about linear motion -- an object moving along a straight path. However, many phenomena have objects moving along curved paths. As an example, we'll take a closer look at an object moving along a circle.

\end{overview}

\noindent\textbf{Phenomenon:} You are going to swing a ball in a circle and represent its velocity in several different ways using vectors.\\

\noindent\textbf{Make sure everyone in your group fully understands the ideas behind each question or part in these activities before going on to the next part.}


\begin{enumerate}
	\item \begin{enumerate}
		\item Get the ball swinging in a \emph{large} horizontal \emph{circle}, going \emph{clockwise} when viewed from above:
	\begin{center}
		\begin{tikzpicture}[decoration={markings,mark=at position 0.5*\pgfdecoratedpathlength-25pt with {\arrow[thick]{<}},mark=at position 0.5*\pgfdecoratedpathlength+5.5cm+5pt with {\arrow[thick]{<}}}]
			\node[inner sep=0pt] (fingers) at (-.5,3) {\includegraphics[width=1cm]{pinchedFingers.png}};
			\draw[postaction={decorate},dashed] (0,0) ellipse (3cm and .8cm);
			\draw (-.18,2.82) -- (1.95,.65);
			\draw[fill=gray] (2,.6) circle (.1);
		\end{tikzpicture}
	\end{center}	
		Imagine looking down on the ball. Pick a position in space that you will identify as the ``12~o'clock'' position. Draw the circular path of the ball on the board with the 12~o'clock position toward the top of the board.
		
		Choose a coordinate system and sketch it on the board.
		
		\item Draw a position vector identifying the position of the ball when it is in the 4~o'clock position. Show the $x$- and $y$-components of this vector on your diagram. Discuss in your group how to do this. Be prepared to share with the whole class.
		
		\item Draw another position vector when the ball is in the 5~o'clock position. To the side of your diagram, graphically subtract the position vectors \textbf{as accurately as possible} to find the displacement vector; label it $\Delta \vec{r}$.
		
		\textbf{\emph{Describe in a sentence what the delta means here.}}
	\end{enumerate}

\vspace{6pt}
\hspace{-\textwidth}\hspace{\linewidth} \textbf{Brief}
\hspace{\textwidth}\hspace{-\linewidth}
\WCD
	
	\item \begin{enumerate}
		\item Describe in words the \textbf{velocity} of the ball as it moves in its circle. Be prepared to share.
		
		\item What do you think is an \emph{instantaneous velocity vector}?  How might it be different from a position vector?  Does it always point in the same direction? If you are having trouble answering these questions, apply them to a specific situation (e.g. driving northwest on I-5 at \unit[65]{mph} and then curving toward the north).
	\end{enumerate}
	
	\item \begin{enumerate}
		\item Imagine that you are sitting on the ball and ``driving'' it in a circle. Which way are you moving when you are at the 4~o'clock position?  Draw a velocity vector (put the tail of the vector at the ball's location) showing the velocity of the ball at the 4~o'clock position. Show the $x$- and $y$-components of this velocity vector on your diagram. Discuss in your group how you do this.
		
		\item  Looking at the vectors you have drawn, is a \emph{position} vector or a \emph{displacement} vector more closely related to a velocity vector? Make sure this agrees with your knowledge of the definition of velocity.
		
		\item Which vectors on your diagram would be different if you changed the origin?
	\end{enumerate}
	
\WCD
	
	\item Draw the velocity vector for the ball at the 6~o'clock position. Off to the side, graphically determine the change in velocity, $\Delta \vec{v}$, between the 4~o'clock position and the 6~o'clock position.
	
	\item \begin{enumerate}
		\item Find the $x$- and $y$-components of the 6~o'clock velocity vector.
		
		\item Find the $x$- and $y$-components of the vector $\Delta \vec{v}$.
		
	\end{enumerate}
	
	\item How can you use the $x$- and $y$-components of the 4~o'clock and the 6~o'clock positions to get the change in velocity, $\Delta \vec{v}$~?  Summarize your method and be prepared to share it with the class.
\end{enumerate}

\WCD