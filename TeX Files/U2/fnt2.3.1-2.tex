\label{fnt2.3.1-2}

A \unit[200]{g} mass is attached to a spring, just like in \hyperref[SpringMassActivity]{Activity~\ref{SpringMassActivity}}. The mass is lifted up \unit[5]{cm} and released so that it begins to oscillate about the equilibrium point. The spring has a spring constant $k = \unitfrac[500]{N}{m}$ ($= \unitfrac[500]{J}{m^2}$).

\begin{enumerate}[(a)]
	\item Calculate and accurately plot on a letter size ($\unit[8.5 \times 11]{in}$) sheet of graph paper $PE_\text{spring-mass}$, $E_\text{total}$, and $KE$. The vertical axis of the graph should be energy (in Joules). The horizontal axis is ``distance from equilibrium'' (in meters). 

	\item On the same graph, quickly sketch (without calculating values) the $PE_\text{spring-mass}$, $KE$ and $E_\text{total}$ of the system if the mass were initially pulled back (stretched) \unit[2.5]{cm} from its equilibrium point, instead of lifted up (compressed) \unit[5]{cm}.
\end{enumerate}