%Not making this a section because it's really just more practice and should be a part of the previous activity
%\section[Quantitative Applications of the Model]{Quantitative Applications of the \EnergyInteractionModel{}}
\label{act1.2.4}

\note{

	\begin{itemize}
		\item Groups 1 - 3 discuss and respond to \ref{fnt1.2.1-6} as directed below.
		\item Groups 4 - 6 discuss and respond to \ref{fnt1.2.1-7} as directed below.
	\end{itemize}

}

\subsection{Boiling Liquid Nitrogen with Ice}  

\begin{fnt}
	\label{fnt1.2.1-6}

A \unit[2.2]{kg} block of ice (\ce{H2O}) initially at a temperature of \unit[-20]{\textdegree C} is immersed in a \emph{very large} amount of liquid nitrogen (\ce{N2}) at a temperature of \unit[-196]{\textdegree C}. The \ce{N2} and \ce{H2O} are allowed to come to thermal equilibrium.  [T$_\text{BP(\ce{N2})}$ = \unit[-196]{\textdegree C}]

Create a particular model of this process and use it to determine how much liquid \ce{N2} is converted to gas (vapor).

[Hint: The emphasis on ``very large'' implies that there will still be liquid \ce{N2} left when the two come to thermal equilibrium.]
\end{fnt}

\note{Timing: \unit[\about10]{min}}{
	
}

\begin{enumerate}
	\item There exists a definite correlation between each energy system in your \EnergyDiagram{} and each $\Delta E$ term in your energy conservation equation. In addition to ensuring that you include all the appropriate $\Delta E$'s (and no extra ones) in your algebraic equation expressing conservation of energy, what other very important detail can your \EnergyDiagram{} tell you about each $\Delta E$ term in your equation?
	
	\item Construct a complete \EnergyDiagram{} (including an algebraic statement of energy conservation in terms of the $\Delta E$'s) for this process. Then rewrite the equation expressing energy conservation substituting in the appropriate expressions for changes in energy of the energy systems using variables only (no numbers). \textbf{Do not substitute any numbers for the variables, and do not solve for the unknown, but circle the unknown variable in the equation.} Put the algebraic sign in parentheses above each algebraic term in the algebraic equation.

	We know you know how to do simple algebra and arithmetic. However, the most common mistake students often make is in the algebraic sign of the whole term when subtracting \emph{initial} values from \emph{final} values of the indicators. By using the \EnergyDiagram{} to check that the sign is correct, you will avoid making this error.\footnote{Just for reference: You likely found that \about\unit[4]{kg} of liquid \ce{N2} was converted to vapor.}

\end{enumerate}

\subsection{Warming Ice with Liquid Water}  

\begin{fnt}
	\label{fnt1.2.1-7}

The \unit[2.2]{kg} block of ice from the \ref{fnt1.2.1-6} is eventually removed from the liquid \ce{N2} (after reaching thermal equilibrium with the liquid \ce{N2}) and placed in a {\em very large} amount of liquid \ce{H2O} at \unit[0]{\textdegree C}, where it comes to thermal equilibrium with the liquid \ce{H2O}. Create a particular model of this process and use it to determine the final mass of the ice.
\end{fnt}

\note{Timing: \unit[\about10]{min}}{}

\begin{enumerate}
	\item Describe in words what is going on in this process, specifically: What physical system is changing phase? What system is changing temperature? What energy systems need to be included in the particular model you create for this process?
	
	\item Construct a complete \EnergyDiagram{} (including an algebraic statement of energy conservation in terms of the $\Delta E$'s) for this process. Then rewrite the equation that expresses energy conservation, substituting in the appropriate expressions for changes in energy of the energy systems using variables only (no numbers). \textbf{Do not substitute any numbers for the variables, and do not solve for the unknown,} but circle the unknown variable in the equation.\footnote{Again, for reference: You likely got a final mass of ice of \about\unit[5]{kg}.}

\WCD

\end{enumerate}

\subsection{Melting Gold}

\begin{fnt}
	\label{fnt1.2.1-8}

\unit[2.0]{kg} of solid gold (\ce{Au}) at an initial temperature of \unit[1000]{K} is allowed to exchange heat with \unit[1.5]{kg} of {\em liquid} gold at an initial temperature of \unit[1336]{K}. The solid and liquid can only exchange heat with each other. When the two reach thermal equilibrium, will the mixture be entirely solid, or will there be a mixed solid/liquid phase? {\em Explain how you know.}

\end{fnt}

\note{Timing: \unit[\textless5]{min}}{
	
}

\noindent Discuss in your group and make sure everyone is prepared to explain your group's response in the whole class discussion.

\WCD
