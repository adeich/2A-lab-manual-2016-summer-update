\label{fnt1.2.1-4}

\ref{fnt1.2.1-2} and \ref{fnt1.2.1-3} emphasized the significance of the beginning and end of the interval -- we've called them ``initial'' and ``final'' states so far. This FNT illustrates
\begin{enumerate}[(i)]

	\item that the question you're asking determines the interval being analyzed with the \EnergyInteractionModel{}; and

	\item that sometimes it's necessary to adjust the size of the interval in order to be able to use the model.

\end{enumerate}

\noindent The physical process we want to analyze with the \ThreePhaseModel{} and the \EnergyInteractionModel{} is the following:\\

\noindent Imagine you are using your hot pot to gradually heat a \unit[500]{g} block of ice that has just been removed from a freezer with an internal temperature of \unit[-25]{\textdegree C}. The hot pot is fairly well insulated, so it is reasonable to assume that all of the heat put into the pot from the electrical heater located in the bottom of the pot goes into the \ce{H2O}. We can also assume that the heat capacity of the pot is much smaller than the heat capacity of \unit[500]{g} of \ce{H2O} (whether solid or liquid), so we can ignore the thermal energy system of the pot itself.

One question we could ask is, ``What is the final state of the water (phase and temperature) after the addition of \unit[252]{kJ} of heat?'' 

\begin{enumerate}[(a)]

	\item Explain why -- based {\em only} on the given information and without further analysis or calculation -- it is \textbf{\em not possible} to construct a {\em single} \EnergyDiagram{} that could be used to determine the final state of the water.
	
		[Hint: try to construct such a diagram. Another way to think about this is whether you can define the final state so that it depends on only one variable without making unjustified assumptions?]

	\item In cases like this, you must first use the model to analyze a {\em shorter interval}. Can you construct an \EnergyDiagram{} that could be used to determine how much energy is needed to increase the temperature of the ice to \unit[0]{\textdegree C}?
	
		Now explain in a few sentences how to proceed using the \EnergyInteractionModel{} to find the final state of the \ce{H2O}.

\end{enumerate}