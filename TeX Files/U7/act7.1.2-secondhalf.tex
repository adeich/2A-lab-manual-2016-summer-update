\textbf{Phenomenon:} Collisions of two carts in one dimension, both elastic and inelastic collisions.

Your instructor will assign each group a Situation from 1 -- 3 below and one from 4--5. Use the \pConsModel{} to analyze each of the collisions between two carts in the Situations~1 -- 3 below. You may treat the system made up of the two carts as a closed physical system because there is no net external impulse transferred to the system during the collision.

\noindent
\textbf{Situations:}
\begin{enumerate}
	\item Use two carts of equal mass, physically arranged so the carts will bounce off one another. Start with \textbf{\em one cart stationary}.
	\item Use two carts of equal mass, physically arranged so the collision ends with the \textbf{\em carts locked (stuck together)}.
	\item Use two carts of equal mass initially \textbf{\em moving toward each other} with equal speed and \textbf{\em ending with the carts locked}.
	\item Place two carts of \textbf{\em unequal mass}, on the track turned so the collision \textbf{\em ends with the carts locked}. Start with one cart stationary, and have the other move to collide with it. Make the stationary cart have more mass (repeat, switching carts).
	\item Place two carts of \textbf{\em unequal mass}, on the track turned so the collision \textbf{\em ends with the carts locked}. Start with both carts \textbf{\em moving toward each other} with the same speed.
\end{enumerate}

\noindent
\textbf{For each Situation:}
\begin{enumerate}[(a)]
	\item Draw and fill in a \pchart{} to help you describe momentum conservation in this closed physical system.
	\item For each line of the \pchart{}, write an algebraic vector equation, with appropriate subscripts on the symbols, to express what that line tells you about the collision.
	\item Draw a force diagram for each cart that shows the forces \emph{during} the collision.
	\item Describe in words what physically happened and then how conservation of momentum applies to each cart and to the system as a whole. Put your equation and word statement on the board.
	\item Compare the total kinetic energy before the collision to the total kinetic energy after the collision.% Classify each case as elastic or inelastic.
\end{enumerate}

Observe what is similar and what is different in your \pcharts{}. What patterns can you observe? Discuss in your group any rules you come up with for the patterns.

\WCD