Describing Simple Harmonic Motion Mathematically

A) Starting the mass-spring at different places.
1)	Look at the graphs you made in the previous activity.  Let's let the $y$ position of the mass on the spring be measured from its equilibrium position with the positive direction as up.  According to your graph, where was the mass and what was it doing at the time $t = 0$.

2)	Make two graphs (with the same amplitude and on the same time axis) of the motion that results when you i) release the mass from its lowest point at $t = 0$ and ii) release the mass from its lowest point but call $t = 0$ the time when it reaches the equilibrium point moving upward.


3)	Discuss in your group how you would write a mathematical expression for each of the graphs you made in (2).   Use general symbols (A, T), not numerical values.  Hint: use a sine function for one and a cosine for the other.  Be careful with minus signs.  Think about what the ``angle'' of the sine or cosine function has to be.  What happens when an angle goes through 2\pi radians?  Remember, an angle has to be in units of degrees or radians.  We want it in radians.   Write your expressions on the board.

\WCD 

B) Turning ``every equation'' (that describes simple oscillation) into a sine function
1)	a)	Discuss in your group how you could write both expressions in (3) using only a sine function, by adding a constant angle to the angle involving the period and time.  Call this constant angle \phi (often called the phase angle, it depends on the initial conditions, e.g., the y-value at t = 0).  Make sure everyone in your group understands how this works.  Write a sine function with the actual value of \phi next to each graph (reference p. 91 in your course notes).


b)	Two ways to graphically think of \phi: (1) Which way (left or right) does the sine function ``shift'' if + or - \phi is added?  OR, (2) Which way does the y-axis ``shift'' if + or - \phi is added?  Check this with your expressions in 4a above.

2)	Draw a graph and write an expression for the following situation: mass released from its lowest point but take $t = 0$ for when the mass is passing through equilibrium point moving down.

\WCD

