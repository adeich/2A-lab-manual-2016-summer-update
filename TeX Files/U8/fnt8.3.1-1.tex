\label{fnt8.3.1-1}

A pendulum of length r with a weight m is raised to an angle \thetamax and when released will oscillate for a period T.
a.	Write out an appropriate equation that describes its oscillation using the variables above.
b.	Draw a large arc to show the path of the pendulum's mass.  Label three representative times for the motion of the pendulum during 0 to T/2: i) the instant it is released (call this point A), ii) when it is at equilibrium (point B), and iii) somewhere in between equilibrium and the maximum amplitude on its way up (point C).  Draw a second arc of the same size, label the exact same places, but now call them D, E, and F.  The second sketch shows the motion from T/2 to T.
c.	To the side of the sketch, draw a \forcediag{} for each point.  Are any diagrams the same?
d.	When is the pendulum moving fastest?  Slowest?  How does this velocity relate to the position graph (Hint: refer to the graphs of position and velocity from previous \FNTs).
e.	On your sketch from part (b), draw appropriately scaled velocity vectors from points A, B, C, D, E, and F.
f.	Explain how you might determine appropriate acceleration vectors for points A-F.