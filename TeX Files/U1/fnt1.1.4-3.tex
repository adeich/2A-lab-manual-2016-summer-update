\label{FNT1.1.4-3}

Perhaps you recall that when table salt, \ce{NaCl}, is added to water, the freezing point of water is lowered. Consider a system composed of a mixture of \unit[2.5]{kg} of ice and \unit[50]{g} of liquid water and a small, separate container of finely powdered salt. This physical system is contained in a fully insulated container that prevents all thermal interactions with the environment. Both the salt and the ice-water mixture are initially at the freezing point of water, \unit[0]{\textdegree C}. The salt is then added to the ice-water mixture, and the system of ice-water and salt is allowed to come to thermal equilibrium. The final equilibrium temperature is less than \unit[0]{\textdegree C}.\\

Use the \EnergyInteractionModel{} to predict if there will be a greater or lesser amount of ice in the final equilibrium state than in the initial state before the salt was added. Your explanation should include a complete \EnergyDiagram{}.\\

[The simplest way to model this physical system is with one thermal energy system for everything and one bond energy system, i.e., in terms of the model, it is not useful to distinguish between the various chemical components in order to answer this particular question.]