\section*{\DLM~\ref{dlm8} Overview}

\subsection*{Brief Overview}
The first Activity reviewing the \FNTs{} from the previous \DLM{} provides you a chance to make sure you comfortable working with both forms of the energy conservation equations (sum of changes and energy totals) as well as graphing energy changes as a function of the position variable in the potential energy.  The second and third Activities introduce new concepts and ideas: the very important relationship between force and potential energy and the physical situation of two masses connected by a spring.

\subsection*{\CLASP{} Activities}
\hyperref[act2.3.2]{Activity \ref{act2.3.2}} \nameref{act2.3.2}: Follow up on \hyperref[fnt2.2.3-1]{\FNT~\thechapter-\ref{fnt2.2.3-1}} through \hyperref[fnt2.3.1-3]{\FNT~\thechapter-\ref{fnt2.3.1-3}}  	(\about\unit[50]{min})

\textbf{Purpose:}
\begin{itemize}
\item To make sure you are sure about and confident with applying the Energy-Interaction Model to various mechanical and combined mechanical and thermal systems
\item Confident in graphing energies as a function of the position variable in the PE.
\item Confident is constructing scientific explanations based on the Energy-Interaction Model
\end{itemize}
\textbf{Learning Outcomes:}
\begin{itemize}
\item \hyperref[fnt2.3.1-1]{\FNTs{} \ref{dlm8}-\ref{fnt2.3.1-1} \& \ref{dlm8}-\ref{fnt2.3.1-2}} :	Ability to accurately graph energies as a function of the position variable of a $PE$. Develop better understanding of the graphical representation of energy relationships: $E_{TOTAL} =$constant and sum of changes in energy.
\item \hyperref[fnt2.2.1-7]{\FNT{} \ref{dlm8}-\ref{fnt2.2.1-7}}	Ability to accurately use the energy-interaction model to obtain numerical predictions for more complicated mass-spring situations.
\item \hyperref[fnt2.2.1-8]{\FNT{} \ref{dlm8}-\ref{fnt2.2.1-8}}	Ability to reason with the Energy-Interaction Model with combined mechanical and thermal systems.
\item \hyperref[fnt2.3.1-3]{\FNT{} \ref{dlm8}-\ref{fnt2.3.1-3}}	Ability to use the �total energy stays constant� approach in mechanical systems
\item \hyperref[fnt2.2.3-1]{\FNT{} \ref{dlm8}-\ref{fnt2.2.3-1}}	Ability to develop solid, logical explanations based on the �givens� in the physical situation and the relationships of the Energy-Interaction Model.  
\end{itemize}

\subsubsection*{\hyperref[act2.4.1]{Activity \ref{act2.4.1}} \nameref{act2.4.1}:	(\about\unit[40]{min})}

\textbf{Purpose:}
\begin{itemize}
\item To introduce the relationship of force to potential energy
\end{itemize}
\textbf{Learning Outcomes:}
\begin{itemize}
\item Achieve understanding of the connection between the force acting on a mass and the shape of the potential energy vs. position curve for that mass.
\item Ability to predict the force acting on a mass from knowledge of the potential energy curve and the location of the mass.
\end{itemize}

%\subsubsection*{\hyperref[act2.4.2]{Activity \ref{act2.4.2}} \nameref{act2.4.2}:	(\about\unit[40]{min})}
%
%\textbf{Purpose:}
%\begin{itemize}
%\item To provide practice in using the relationship $F = -\frac{dPE}{dx}$
%\item Introduction to the Lennard-Jones potential
%\end{itemize}
%\textbf{Learning Outcomes:}
%\begin{itemize}
%\item Be familiar with the general shape of the Lennard-Jones potential
%\item Ability to describe (direction and relative magnitude) the force acting between two masses subject to a Lennard-Jones potential energy
%\end{itemize}


\subsubsection*{Reminder}

Keep an eye on the Small Groups as they work on the activities to ensure that they are following the instructions on the activity sheets. 

\subsubsection*{Equipment needed in the room:}
\begin{itemize}
\item Hanging brass spring with 200g mass.  
\end{itemize}

\subsection*{The single most important thing you can do is to get your students talking to each other. }