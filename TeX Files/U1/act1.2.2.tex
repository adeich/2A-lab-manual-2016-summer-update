\note{Brief Overview}{

    \subsection*{\ref{act1.2.2}	Mostly quantitative application of the \EnergyInteractionModel{} to thermal and bond systems		(\about50 mins)}
    \subsubsection*{Purpose}
    
    \begin{itemize}
    	\item To provide practice in reasoning with the \EnergyInteractionModel{} and explicitly using the constructs of heat capacity and heats of vaporization and melting and the distinction between power and energy.
    	\item Practice going through the modeling process without taking shortcuts.
    	\item Practice getting quantitative answers using the logic of the model instead of simply ``grabbing an equation and plugging in numbers'' without understanding what is going on.
    \end{itemize}
    \subsubsection*{Learning Outcomes}
    
    \begin{itemize}
    	\item Confidence in the ability to apply the \EnergyInteractionModel{} to answer questions and make predictions for more complicated thermal phenomena.
    	\item Understanding that consistently going carefully through the steps of the \EnergyInteractionModel{}, it is possible to avoid the all too common mistakes that occur when you simply grab an equation to plug and chug.
    \end{itemize}
    
    \section*{Handouts}
    
    In this DL, each student should receive Handout 1.2.2:  (for use with Act 1.2.2)
    
    \section*{In general, when you run out of time to complete DL Activities}
% testxxx}   
    	For any unfinished DL activities from the previous DL (which should have been assigned as an \FNT), give a brief summary (\textless 5 mins.) at the beginning of the current DL and sum up the main points for them. Don't spend time in DL as you normally would having the students discuss and explain.
    	
    \section*{Suggested times}
    	
    The suggested times {\em do not} include the \WC{} discussion.

    %\vspace*{1cm}
    Make sure students follow instructions! Prompts and questions are carefully structured to encourage them to process the material in particular ways.

   % \vspace*{1cm}
  %  \begin{mdframed}
  \textbf{
    The emphasis in much of this DL is understanding and using the logic of the model, {\em not} getting a numerical answer!
	}
   % \end{mdframed}
}     	

%Making this a subsection because it's still about practicing the models...
\subsection{Thermal and Bond Energy Systems}
\label{act1.2.2}

\begin{FNTenv}
	\input{U1/FNT1.2.1-1}
\end{FNTenv}

\note{Timing: \unit[\textless3]{min}}{
	\begin{itemize}
		\item The answer may be obvious, but make them explain how they know in the \WC{} discussion. For example, ``With only two energy systems, the magnitude of energy leaving the one must be the same as the magnitude of energy into the second.'' As always, it is important for them to put things into their own words. 
		\item Follow-up question: Ask them what the effect of the different heat capacities will be.
	\end{itemize}
}

\noindent Discuss this \FNT{} in your group and be prepared to explain in the whole class discussion. You do not need to put anything on the board.

\WCD

\begin{FNTenv}
	\input{U1/FNT1.2.1-2}
\end{FNTenv}

\note{Timing: \unit[$\sim$15]{min}}{
	\subsubsection*{What is the purpose of Handout 1.2.2?  (distributed to students)}
	\begin{itemize}
		\item Save time in today's DL. None of the diagrams in the handout needs to be put on the board by students or instructors. 
		\item Everyone has uniform conventions for \EnergyDiagrams{}. 
		\item Illustrates what is appropriate to put on the board for \WC{} discussion. 
		\begin{itemize}
			\item It is not useful to show the algebraic manipulations necessary to get the final answer on the board. Make it clear that in the future, only what is shown in this diagram is what they should put on the board for the \WC{} discussion.
			\item Discourage the students from working out the algebra in class. We want them to understand the difference between ``the physics'', which is illustrated in the diagrams on the handout, and ``the algebra''.
		\end{itemize}
	\end{itemize}
}


\begin{enumerate}[1.]
	\item A student constructed the following \EnergyDiagram{} for Part (a) of \hyperref[\FNT1.2.1-2]{this \FNT}:

	\includegraphics[width=0.7\linewidth]{handout-FNT121-2a}
	
	Discuss the diagram briefly in your group, and identify any questions you have. You don't need to put anything on the board.\footnote{Incidentally, this diagram illustrates how much algebra is useful to put down for the purpose of clear communication in whole class discussions. In general, it is not useful to show the details of solving the algebra on the board. Being able to construct the correct diagram and getting the first three lines of algebra (including verifying the signs of the terms in the third line) are worth most of the credit on an exam.}
	
	\note{}{
		Students inspect diagram for part (a) on the handout and identify any questions. Not necessary to put anything on the board.
		
		If they have questions about the diagram you can respond either to the SG individually, or if you think others might benefit, with the \WC{}. 
	}
	
	\item Put a complete \EnergyDiagram{} on the board for the shorter interval described in Part (d) of \hyperref[\FNT1.2.1-2]{this \FNT}. What specifically is happening in the process that ``connects'' the two final temperatures (of lead and aluminum) in your diagram? Illustrate this on the board with two separate \TempGraphs{}, one for each substance.
	
	\note{}{
		Put a complete \EnergyDiagram{} on the board for the {\em shorter} interval described in part (d). What is the connection between the two final temperatures in your diagram? 
		\begin{itemize}
			\item The \EnergyDiagram{} is shown directly below.
			\item The connection: the two indicators are correlated with the end of the interval, i.e., they have those values at the same time. This may seem obvious to the point of being trivial, but as the number of terms in the equation expressing energy conservation increases, students have more difficulty seeing the relationship among the various initial and final values.
		\end{itemize}
	}
	
	\note{}{
		\noindent
		\includegraphics[width=\linewidth]{FNT121-2b}
	}

\WCD 

\end{enumerate}

\begin{FNTenv}
	\input{U1/FNT1.2.1-3}
\end{FNTenv}

\note{Timing: \unit[\about10]{min}}{
	
}

\begin{enumerate}[1.]

	\item Again, we consider a student's response to \hyperref[\FNT1.2.1-3]{this \FNT}. Discuss the diagrams for Part (a) below and identify any questions you have. As before, you don't have to put anything on the board for this part.
	
		\includegraphics[width=0.55\linewidth]{handout-FNT121-3a1}
		\;
		\includegraphics[width=0.35\linewidth]{handout-FNT121-3a2}
	
	\item A student objects to the diagram for Part (b) of \hyperref[\FNT1.2.1-3]{this \FNT} below because some of the indicators do not correspond to the beginning and end of the overall interval.
	
		\includegraphics[width=0.7\linewidth]{handout-FNT121-3b}
		
		Another student says that's ok because each physical process in the interval happens sequentially and nothing has been left out. Come to a consensus about which of these two students you agree with and put a brief explanation on the board.
	
	\item Discuss the question about power in Part (c) in your group and identify any questions you have. Make sure you have a clear understanding of the relationship of power to energy. You don't have to put anything on the board for Part (c), but be ready to explain the difference between energy and power if called upon in the whole class discussion.

\WCD

\end{enumerate}

\subsection{Asking Questions and Determining Intervals of Interest}
\label{act1.2.2D}

\begin{FNTenv}
	\input{U1/FNT1.2.1-4}
\end{FNTenv}

\note{Timing: \unit[\about10]{min}}{
	
}

\begin{enumerate}[1.]

	\item Why is it \emph{not} possible to apply the \textbf{\EnergyInteractionModel{}} in a {\em straightforward} way to the interval of interest -- the interval that ends when all of the \unit[252]{kJ} has been added to the water? In other words, why can't the overall interval be diagrammed using {\em only} the given information prior to doing any further analysis?   Draw a \TempGraph{} to help you make sense of this and to use to explain it. In particular, be sure to make explicit which part of the graph in the diagram refers to which ``energy bubble'' on the \EnergyDiagram{}. [To save time, explanations on the board can be more abbreviated than they would be on an exam. In this case, anything that is unclear can be easily clarified in the whole class discussion.]
	
	\item Explain how to use the \textbf{\EnergyInteractionModel{}} to analyze shorter intervals in order to determine the final state (phase and temperature) of the water.  (Note: ``water,'' without a modifier, usually means \ce{H2O} in any one of its three phases.) Use a \TempGraph{} in your analysis and to use in your explanation.

\WCD 

\end{enumerate}