\label{fnt1.2.1-2}

Consider the following example of two substances at different temperatures exchanging energy via heating as they come to thermal equilibrium.

A \unit[3.50]{kg} piece of lead and a \unit[1.5]{kg} piece of aluminum, at different temperatures, are placed in contact. They are able to exchange energy via heating and they can only exchange it with each other. The initial temperature of the lead is \unit[48]{\textdegree C}, and the initial temperature of the aluminum is \unit[35]{\textdegree C}. The table on Page 5 the course notes, and Table A.6 in Appendix A, has phase change temperatures and values of specific heats for these substances. 

\begin{enumerate}[(a)]
	\item If it is possible (How do you know this?), construct a single, closed \EnergyDiagram{} for the process of lead and aluminum coming to thermal equilibrium.

	\item Write an algebraic expression of energy conservation in terms of the $\Delta E$s of the two energy systems, and then substitute in the appropriate expressions for the $\Delta E$s.

	\item Substitute in all known numerical values. How many unknowns are there in your equation expressing energy conservation?

	\item At an {\em intermediate} point in the entire process described above the lead has reached a temperature of \unit[43]{\textdegree C}.
	
	Follow the instructions for part (a) for the interval that \emph{ends when the lead is at a temperature of \unit[43]{\textdegree C}}. What feature related to the process connects the final temperatures of lead and aluminum that appear in your diagram to each other?

\end{enumerate}