\label{fnt1.1.3-1}

Neatly write out all of the \EnergyDiagrams{}, with the accompanying \TempGraphs{}, listed below (these are the same scenarios that were requested in \hyperref[act1.1.3]{Activity~\ref*{act1.1.3}}).

Remember that complete \EnergyDiagrams{} always include algebraic expressions of energy conservation. Refer to the \EnergyInteractionModel{} discussion in the online resources.

\begin{enumerate}[(a)]
	\item Cooling a piece of solid copper (Cu) from \unit[500]{\textdegree C} to \unit[350]{\textdegree C}.
	\item Warming a piece of ice from \unit[-20]{\textdegree C} to the melting point.
	\item Condensing steam completely to liquid at \unit[100]{\textdegree C}.
	\item Completely sublimating a chunk of dry ice at \unit[-79]{\textdegree C}.
	\item Partially melting 25\% of ice initially at \unit[0]{\textdegree C}.
	\item Heating a piece of copper initially at \unit[300]{\textdegree C} until it is half melted.
	\item Cooling and completely freezing H$_2$O initially at \unit[80]{\textdegree C}.
\end{enumerate}

