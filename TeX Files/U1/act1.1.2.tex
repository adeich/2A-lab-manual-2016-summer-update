\section[Analyzing a Heat Pack]{Analyzing a Heat Pack Using the \ThreePhaseModel{}}
\label{act1.1.2}

\todo[inline]{I noticed that many times, the "\ThreePhaseModel{}..." is referred to as "\ThreePhaseModel{}" without hyphen. This is a minor issue, but we should be consistent. I've replaced it in this file, but maybe we can automatically find/replace this in all files while importing?

-BH}
\todo[inline]{I've got a python script that will take care of this. It's in the \texttt{python} folder and called \texttt{ctrl\_r.py}. You're welcome to edit the list of stuff to search for and change at the beginning of the script or leave me a note and I can do it.

-EH}

\note{Timing: \unit[40]{min}}{
	
	\subsubsection*{Purpose}
		\begin{itemize}
			\item Provide practice using the \ThreePhaseModel{} applying it to the heat pack.
			\item Provide awareness that a general model, in this case the \ThreePhaseModel{}, might not apply without modification in some situations such as cooling of sodium acetate (heat pack) through the freezing temperature.
		\end{itemize}
	
	\subsubsection*{Learning Outcomes}
		\begin{itemize}
			\item Be able to explain how a specific scientific model can be used to constrain what aspects of a phenomenon need to be taken into account and what it means to make sense of a phenomenon in terms of a model.
			\item Be able to explain during which heating and cooling processes of a heat pack can be explained using the simple \ThreePhaseModel{} and which processes cannot.
		\end{itemize}
}

\begin{overview}
	\noindent
	{\bfseries Overview:} Our goal is to make sense of the heat pack behavior using both the \ThreePhaseModel{} and the \EnergyInteractionModel{}. We'll start with a focus on the \ThreePhaseModel{}.

\end{overview}

\subsection{Macroscopic properties and energy transfers}
\label{A1}

When analyzing a physical phenomenon, a model greatly simplifies and restricts what we focus on. \textbf{The constructs that make up the model {\em limit} the aspects of the phenomenon that we must pay attention to.}

\begin{enumerate}
	\item Using {\em only} the constructs of the {\bfseries \ThreePhaseModel{}}, what are \textbf{\em macroscopic properties} of the heat pack that changed (and how did they change) during the first few minutes after triggering?
	
		Look at the Model Summary Sheet and try to use only those words.
		
		Write your new story on the board.
		
	\note{Optimal student response}{
		``{\em Temperature} increased and {\em phase} changed from liquid to solid.''
		
		Students should be focused on macroscopic properties of the model, not simply repeating their response to the first question in \hyperref[act1.1.1]{Act.~\ref*{act1.1.1}}; e.g., color changed from clear to white.
	}
	\note{General}{
		Make sure all \SGs{} actually write these things on the board. You will probably have to encourage them to do so. A loud announcement directed to the whole class can be effective with this kind of reminder. Make sure the students get used to following these instructions to write responses on the board.
	}

	
	\item How does the amount of energy associated with clicking the small disk inside the heat pack compare to the other energy transfers that occur? Think, for example, about the amount of heat transferred from the heat pack to your hands or to the environment.
	 
	 	Is it about the same amount of energy? Much larger? Much smaller?
	 	
	 	Write your group's response on the board.
	
	\note{}{
		Help students convince themselves that the mechanical energy associated with the trigger is insignificant compared with the other energy changes and transfers. You can suggest they repeatedly squeeze a liquid heat pack other than on the disk and see if it makes it any warmer.
	}

\WCD

\end{enumerate}

\note{}{
	\begin{itemize}
		\item Pick out one or two groups to share their responses to each of the questions. Alternatively, you can read responses from one or two groups. You can ask the whole class if they like the way a particular response is worded or how it could be improved, etc. Try to get them to do the explaining!
		\item Students should leave this discussion with the notion that a lot more energy ``comes out'' as heat than was put in with the clicker. You can talk about the clicker action as simply a ``trigger.''  So, the bottom line, which you should state, is that we agree that changes in clicker energy-systems are not significant and that a lot of energy comes out as heat.
	\end{itemize}
}

\subsection{Completing the cycle}

\note{General}{
	Suggest to the class that they can start this part before the Whole Class Discussion has occurred for the previous part. Remind them to put their response on the board and to constantly refer to the \textbf{\ThreePhaseModel{}} Summary and to use that language.
	
	{\em Don't wait for all groups to finish before wrapping it up in the Whole Class Discussion.}
}

Now, let's take the heat pack we had triggered before through the rest of its cycle.

\begin{enumerate}
	\item Read and follow the instructions on the heat pack to get it back to being a liquid at room temperature.
	\item Describe qualitatively and briefly how the properties of the heat pack identified in Part \ref{A1} changed during the rest of the cycle. 
	\item Write your new story on the board.
\end{enumerate} 

\noindent\textbf{Attention:} You will continue your analysis of the heat pack in your homework assignment. Therefore, please write down in your notes how the properties changed as the heat pack went through its complete cycle. You will need this!

\WCD

\note{}{
	\DL{} instructor should quickly summarize the changes in temperature and phase of the pack when it is in the boiling water and then cools down to room temperature. Students should understand that this takes the pack through a cycle, i.e.\ it returns to the original state. You can remind them that one of their homework assignments will ask them to diagram a series of processes that take the heat pack through an entire cycle.
}

\subsection{Using our model}
\label{act1.1.2B.3}

\begin{enumerate}
	\item Think about your response to \hyperref[Part B 3j]{Question~\ref*{Part B 3j}} in \hyperref[act1.1.1]{Activity~\ref*{act1.1.1}} on \hyperref[Part B 3j]{page~\pageref{Part B 3j}}.\footnote{Here's the question again, for reference, and so you don't need to flip back and forth: ``When two phases of a substance are present and in thermal equilibrium (i.e., both phases have the same temperature), what do you know with certainty about certain properties or physical conditions as they relate to that substance?''} When your heat pack is in a mixed solid/liquid state, what do you know about its temperature?
	
		Use this knowledge to experimentally determine the melting-point temperature of the heat pack. In order to ensure equilibrium conditions, you can minimize energy transfers to or from the heat pack by wrapping plastic bubble-wrap around the heat pack and thermometer. You can also fold your heat pack around the end of the thermometer.
		
		Does it matter how you attained the mixed state: melting when in the hot pot or freezing when triggered?
		
		Record your result on the board, indicating a ``direction'' on your diagram.
	
	\note{}{
		In principle, the phase temp between solid and liquid can be determined by getting to the phase change going from either direction. It is much easier, however, to get an equilibrium mixed state going from liquid to solid by triggering, rather than using a hot water bath to partially melt a solid heat pack. Don't let the \SGs{} spend any significant amount of time figuring out that triggering is the quickest way to achieve the equilibrium phase change temperature. The main point is you want both phases in equilibrium. (T$_\text{MP} \approx 54^\circ$C)
	}
	
	\item Two processes that you can readily observe with the heat pack are: i) changing from liquid to solid and ii) changing from solid to liquid.
	\begin{enumerate}
		\item For which of these two processes does the heat pack seem to follow the predictions of the {\bfseries \ThreePhaseModel{}} and for which process does it \textbf{\em not} follow the model? Be brief, but specific. Put your response on the board. [Hint: Use your finger to trace the graph.]
		
		\item For which of these two processes does the heat pack behavior pretty much agree with your intuition about phase changes? For which process does it not agree? Put your response on the board. Does your intuition now agree pretty much with the model? If not, what is different?
	\end{enumerate}
	
	\note{}{
		The liquid to solid phase transition does not occur as predicted by the model. As the heat pack is cooled it remains in the liquid state significantly below the phase change temperature. Melting a solid heat pack in boiling water does follow the model. 
	}
	
\WCD

\end{enumerate}

\note{}{
	\DL{} instructor does brief summary of entire activity. Main points are:
	\begin{itemize}
		\item Liquid/solid phase change temperature is $\sim54^\circ$C.
		\item The heat pack follows the \ThreePhaseModel{} when going from solid to liquid, but not going from liquid to solid.
		\item The heat pack drops below the temp at which it should change to a solid, instead remaining a liquid. We give the label (name) supercooling to this phenomenon of staying a liquid below the phase change temp. (Don't go into any more detail. There is a homework exercise on how to represent this on a Temp to Energy Added Diagram, with follow-up discussion in the \hyperref[act1.1.5]{Activity~\ref*{act1.1.5}}. This will be the first example of modifying and extending a model.
	\end{itemize}
}