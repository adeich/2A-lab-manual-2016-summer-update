\label{fnt7.1.1-6}

%You are riding in a car that crashes into a solid wall.  The car comes to a complete stop without bouncing back.  The car has a mass of \unit[1500]{kg} and has a speed of \unitfrac[30]{m}{s} before the crash (this is about \unitfrac[65]{mi}{hr}).  

Refer to your \pcharts{} from \ref{fnt7.1.1-4}. Consider the following two situations.

\begin{enumerate}[I.]
	\item You remain buckled into the seat and the seat remains attached to the center of the car.
	\item You are not buckled into your seat and you fly through the windshield and hit the wall.
\end{enumerate}

\begin{enumerate}[(a)]
	\item Is the magnitude of the impulse the same in both cases?
	\item Are the magnitudes of the forces acting on you the same?
	\item Using the words impulse, force, time, and momentum explain why one scenario is safer for you.  Hint: Think about how the shape of the car changes when it hits the wall.
\end{enumerate}