\label{fnt2.2.3-1}

This FNT is excellent practice in constructing a scientific argument based on logic that follows directly from the physical situation and the relationships of the models you use to, well, model the situation. This is what science is all about!\\

\noindent Please refer back to \hyperref[act2.2.3]{Activity~\ref*{act2.2.3}}:

\begin{enumerate}[(a)]
	\item Finish any of the prompts from \ref{act2.2.3-2} and \ref{act2.2.3-3} that you were unable to finish during the last discussion lab.
	\label{fnt2.2.3-1a}
	
	\item  Respond to the prompt in \ref{act2.2.3-4}, which asks you to turn your responses to prompts from \ref{act2.2.3-2} and \ref{act2.2.3-3} into a set of statements that constitute a logical argument based on the \EnergyInteractionModel{}.
	
		With your argument, explain why the set of masses with smaller total mass moves faster than the set with greater total mass even though the difference between the two masses in each set is the same.
	\label{fnt2.2.3-1b}
\end{enumerate}