\section[Analyzing the Motion of Dropped Objects]{Analyzing the Motion of Dropped Objects with the \FModel{}}
\label{act8.2.1}
\note{For Activity~\ref{act8.2.1} (\about\unit[60]{min})}{
\subsection*{Learning Goals:}
\begin{itemize}
\item Get practice using the Newtonian Model to make sense of the real behavior of dropped objects
\item Develop understanding of the differences between an approach using the Newtonian Model an approach using the Momentum Conservation Model and how both relate to energy conservation
\item Get practice figuring out which mode can be used to answer particular questions about various phenomena
\end{itemize}

The quantitative analysis is done as \FNTs{} (see exit handout).  We will spend \DLM{} time doing the more qualitative analysis.  

The purpose of the two objects (coffee filter and wadded up piece of paper) is to emphasize they have very different net forces a short time after they begin to fall.  With the wadded up piece of paper, the viscous drag is negligible.  With the coffee filter, the viscous drag (force of air on filter) quickly increases to a value equal to the force of the Earth on the filter.  Whence, net force on filter equals zero shortly after it begins to fall.  However, viscous drag is dependent on speed (often v2), so initially, both objects have the same acceleration when they are dropped, since initial speed is zero.  And if we dropped the wadded up piece of paper from the top of the physics building it, too, would quickly come to terminal velocity as its speed increased.

}

\begin{overview}
	\textbf{Overview:} Now that we've spent some time getting to know Newton's laws -- especially in static cases without movement -- we'll use them to model and understand motion. We'll start with the case of falling objects.
\end{overview}

\noindent\textbf{Phenomenon:} Dropping a wadded up piece of paper and a flat bottomed coffee filter

\begin{benumerate}
	\bitem{Observing the Motion.}
	
	\begin{enumerate}
		\item Obtain a flat-bottomed coffee filter that still has its original shape. Wad up a piece of paper until it is in the shape and size of a golf ball. Drop both simultaneously from a height of about six feet.
		
		\item Carefully observe the motion of the two objects.
		
		\item Write out a description of the motion you observed. We talked about ``velocity'' before in this course -- use this concept in your description.
		
		\item \label{821,1d}Sketch a velocity graph on the board (only the vertical component) for each object on the same set of velocity versus time axes.
		\note{}{Graphing $v$ vs $t$ does not come easily to the students.  \textbf{Note that the graphs of these motions should be done semi- quantitatively.  That is, use the info at hand:  actual distances and known accelerations. }}
	\end{enumerate}

\WCD
\filbreak
	\bitem{Analyzing the Motion.}
	
	\begin{enumerate}
		\item Create \forcediags{} for both objects at two points in time, 
		\begin{enumerate}
			\item at the exact instant each object is released, and
			\item after each object has fallen about \nicefrac{1}{4} of the way toward the floor.
		\end{enumerate}
		Show the net force vector for each object at each point in time explicitly (use a thick line). What does this tell you about the accelerations of these two objects as they fall? Apply Newton's 2nd law to both objects in order to determine their accelerations.
		
		\item Use the accelerations you obtained in (a) along with your observations about the motion of each object to make an acceleration graph ($\vec{a}$ versus time) for the two dropped objects from the time they are released to just before hitting the floor.

		\item \label{821,2c} What is the mathematical relationship between the instantaneous acceleration and the instantaneous velocity? How does this relationship show up in the two graphs you made for the two cases? Adjust your curves to be consistent with both (b) and (c).

	\end{enumerate}
\end{benumerate}
		\note{For \ref{821,2c}}{This pushes students to connect to what they know from math.  What we are after is the mathematical relation between the shape of the velocity and acceleration curves.  

They need to come up with the mathematical relation:  $a = dv/dt$;

We know the acceleration for each (constant = g and constant = 0).  So, using the definition of $a$ in terms of the derivative of $v$, we can �work backwards� to get what we must have had to differentiate to get $a$.  We know that the $a$ vs. $t$ graph must be the derivative of the $v$ vs. $t$ graph at each point.

Have the students compare their $a$ vs. $t$ graphs they draw here to the respective $v$ vs. $t$ graphs they made in \ref{821,1d}; they should check/confirm that the $a$ vs. $t$ graphs are the derivative of the $v$ vs. $t$ graphs.  If they aren�t then have the students make the corrections.
}
\WCD
