\label{fnt1.2.1-6}

A \unit[2.2]{kg} block of ice (\ce{H2O}) initially at a temperature of \unit[-20]{\textdegree C} is immersed in a \emph{very large} amount of liquid nitrogen (\ce{N2}) at a temperature of \unit[-196]{\textdegree C}. The \ce{N2} and \ce{H2O} are allowed to come to thermal equilibrium.  [T$_\text{BP(\ce{N2})}$ = \unit[-196]{\textdegree C}]

Create a particular model of this process and use it to determine how much liquid \ce{N2} is converted to gas (vapor).

[Hint: The emphasis on ``very large'' implies that there will still be liquid \ce{N2} left when the two come to thermal equilibrium.]