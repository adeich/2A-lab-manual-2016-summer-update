\label{FNT1.1.4-4}

\todo[inline]{\FNT{} 1.1.4-4) change to thought experiment and we will do in class}

This \FNT{} is a \textbf{\emph{thought experiment}} to make predictions about an actual experiment we will do in class, where you will get to observe the phenomenon described in \ref{FNT1.1.4-3}.\\

Imagine you fill an insulated cup almost full with chopped or crushed ice, and measure the temperature after a minute or two, once it's all come to thermal equilibrium. Since this ice is frozen water, the temperature should be at \unit[0]{\textdegree C} or not more than a couple of degrees below. Then, imagine you add a bunch of salt and stir it around. 

What do you think the lowest temperature you can attain will be? Why? What happens to the amount of liquid present if you keep stirring and adding salt? How can we understand this phenomenon in terms of thermal and bond energy systems?

Develop an explanation for the changes you would observe in this physical system (decrease in temperature and change of phase) in terms of the \EnergyInteractionModel{}.