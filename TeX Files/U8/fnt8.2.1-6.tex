\label{fnt8.2.1-6}

\def\FunctionFa(#1){-5.2*(#1)^2+1.5} % left part of first plot
\def\FunctionFb(#1){40*((#1)-.55)^2+.1} % middle part of first plot
\def\FunctionFc(#1){-4.4*((#1)-1.1)^2+1.3} % right part of first plot

\def\FunctionGa(#1){-10*(#1)} % left part of second plot
\def\FunctionGb(#1){-9*(#1)+9.9} % right part of second plot
\def\FunctionGc(#1){190*(#1)-104.7} % middle part of middle part of second plot
\def\FunctionGd(#1){3200*((#1)-.51)^2-5.1} % left part of middle part of second plot
\def\FunctionGe(#1){-3000*((#1)-.59)^2+4.6} % right part of middle part of second plot

\noindent Refer to the graphs below. Data for these graphs was collected using a motion detector mounted above a basketball that was dropped from a height of \unit[1.5]{m} above the floor. The position measured and indicated is the position of the top surface of the ball. The velocity graph was computed by the software as the derivative of the position vs.\ time graph. Complete the following tasks related to this situation.

\begin{enumerate}[(a)]
	\item Determine accurately the acceleration from the velocity graph and plot it on the acceleration axis. Make sure you extend the acceleration curve as far as the other two are extended in time.
%\hspace{-4.805cm}\parbox[t][][t]{.95\textwidth}{%
\hspace{-10cm}\parbox[t][][t]{.95\textwidth}{%
      \vspace{2mm}
\begin{wrapfigure}{R}{0.5\textwidth}
	\vspace{-20pt}
  	\centering
\begin{center}
	\begin{tikzpicture}[scale=0.75, every node/.style={transform shape},background rectangle/.style={fill=white}, show background rectangle]
	% groupplot environment to allow for aligned arrangement of three plots on top of each other
	\begin{groupplot}[group style={group size=1 by 3},xmin=0,height=5.5cm, width=7cm, no markers]
	\nextgroupplot[					%position graph
		axis y line=center,			%not entirely sure what these do what it looks good
		axis x line=middle,
		inner axis line style={-},		%no arrows on axes
		axis on top=true,
		label style={font=\scriptsize},	%small font
		xlabel={$t~[\unit[]{sec}]$},		%x-axis label
		every axis x label/.style={
		    at={(ticklabel* cs:1.02)},
		    anchor=west,
		},
		ylabel={position $y~[\unit[]{m}]$},	%y-axis label
		ylabel near ticks,			%positioning of the label on the left and rotated
		ylabel shift=-1mm,			%fixing the distance between the axis label and tick labels
		xmin=0,				%coordinate ranges for the axes
		xmax=1.2,
		ymin=0,
		ymax=1.51,
		tick label style={font=\footnotesize},		%making all tick labels small
		xticklabel style={					%formatting specific for x axis
			/pgf/number format/precision=1,	%two decimal precision
			/pgf/number format/fixed,			%no 10^x notation
			/pgf/number format/fixed zerofill,	%add zeros even if they're not necessary to fill up precision notation
		},
		minor x tick num={4},		%add four ticks in between major ticks on x axis
		minor y tick num={4},		%add four ticks in between major ticks on y axis
		xtick distance={.5},
		ytick distance={.5},			%major ticks 0.5 on the y axis
		extra y ticks={0},			%add a tick in the origin
		clip=false,					%so that the pin nodes are not clipped
		enlargelimits=false,
		]
		\addplot					%draw left part of first plot
		[
			domain=0:.5,			%range 0-.5
			samples=50,
			mark=none,
			very thick
		]
		{\FunctionFa(x)}
		node[pos=0,circle,fill=black,scale=0.4] {}	%add first numbered node
		node[pos=0,pin={[inner sep=1pt,pin distance=2mm]45:$1$},inner sep=2pt] {}
		;
		\addplot					%draw right part of first plot
		[
			domain=0.6:1.2,			%range .6-1.2
			samples=50,
			mark=none,
			very thick
		]
		{\FunctionFc(x)}
		node[pos=0.75,circle,fill=black,scale=0.4] {} %add fourth numbered node
		node[pos=0.75,pin={[inner sep=1pt,pin distance=2mm]115:$4$},inner sep=2pt] {}		
		node[pos=0.918,circle,fill=black,scale=0.4] {} %add fifth numbered node
		node[pos=0.918,pin={[inner sep=1pt,pin distance=2mm]115:$5$},inner sep=2pt] {}		
		node[pos=1,circle,fill=black,scale=0.4] {} %add sixth numbered node
		node[pos=1,pin={[inner sep=1pt,pin distance=2mm]45:$6$},inner sep=2pt] {}		
		;
		\addplot					%draw middle part of first plot
		[
			domain=0.5:0.6,			%range .5-.6
			samples=50,
			mark=none,
			very thick
		]
		{\FunctionFb(x)}
		node[pos=.5,circle,fill=black,scale=0.4] {} %add second numbered node
		node[pos=.5,pin={[inner sep=1pt,pin distance=2mm]90:$2$},inner sep=2pt] {}
		node[pos=.85,circle,fill=black,scale=0.4] {} %add third numbered node
		node[pos=.85,pin={[inner sep=1pt,pin distance=2mm]0:$3$},inner sep=2pt] {}		
		;

	\nextgroupplot[					%velocity graph
		axis y line=center,			%not entirely sure what these do what it looks good
		axis x line=middle,
		inner axis line style={-},		%no arrows on axes
		axis on top=true,
		label style={font=\scriptsize},	%small font
		xlabel={$t~[\unit[]{sec}]$},		%x-axis label
		every axis x label/.style={		%position the label
		    at={(ticklabel* cs:1.02)},		%put it at 102% of the axis width
		    anchor=west,			%put it to the right of the axis
		},
		ylabel={velocity $v~[\unitfrac[]{m}{s}]$},			%y-axis label
		ylabel near ticks,			%positioning of the label on the left and rotated
		ylabel shift=-3mm,			%fixing the distance between the axis label and tick labels
		xmin=0,				%coordinate ranges for the axes
		xmax=1.2,
		ymin=-12,
		ymax=12,
		tick label style={font=\footnotesize},		%making all tick labels small
		xticklabel style={					%formatting specific for x axis
			/pgf/number format/precision=1,	%two decimal precision
			/pgf/number format/fixed,			%no 10^x notation
			/pgf/number format/fixed zerofill,	%add zeros even if they're not necessary to fill up precision notation
		},
		yticklabel style={					%formatting specific for x axis
			/pgf/number format/precision=0,	%null decimal precision
			/pgf/number format/fixed,			%no 10^x notation
			/pgf/number format/fixed zerofill,	%add zeros even if they're not necessary to fill up precision notation
		},
		minor x tick num={4},		%add four ticks in between major ticks on x axis
		minor y tick num={4},		%add four ticks in between major ticks on y axis
		xtick distance={.5},
		ytick distance={10},			%major ticks 10 on the y axis
		extra y ticks={0}			%add a tick in the origin
		]
		\addplot					%draw left part of second plot
		[
			domain=0:0.51,
			samples=50,
			mark=none,
			very thick
		]
		{\FunctionGa(x)};
		\addplot					%draw right part of second plot
		[
			domain=0.59:1.2,
			samples=50,
			mark=none,
			very thick
		]
		{\FunctionGb(x)};
		\addplot					%draw middle part of middle part of second plot
		[
			domain=0.5325:0.5675,
			samples=50,
			mark=none,
			very thick
		]
		{\FunctionGc(x)};
		\addplot					%draw left part of middle part of second plot
		[
			domain=0.51:0.533,
			samples=50,
			mark=none,
			very thick
		]
		{\FunctionGd(x)};
		\addplot					%draw right part of middle part of second plot
		[
			domain=0.567:0.59,
			samples=50,
			mark=none,
			very thick
		]
		{\FunctionGe(x)};

	\nextgroupplot[					%acceleration graph
		axis y line=center,			%not entirely sure what these do what it looks good
		axis x line=middle,
		inner axis line style={-},		%no arrows on axes
		axis on top=true,
		label style={font=\scriptsize},	%small font
		xlabel={$t~[\unit[]{sec}]$},		%no x-axis label on the first plot (we'll put that on the next one)
		every axis x label/.style={
		    at={(ticklabel* cs:1.02)},
		    anchor=west,
		},
		ylabel={acceleration $a~[\unitfrac[]{m}{s^2}]$},			%y-axis label
		ylabel near ticks,			%positioning of the label on the left and rotated
		ylabel shift=-3mm,			%fixing the distance between the axis label and tick labels
		xmin=0,				%coordinate ranges for the axes
		xmax=1.2,
		ymin=-60,
		ymax=120,
		tick label style={font=\footnotesize},		%making all tick labels small
		xticklabel style={					%formatting specific for x axis
			/pgf/number format/precision=1,	%two decimal precision
			/pgf/number format/fixed,			%no 10^x notation
			/pgf/number format/fixed zerofill,	%add zeros even if they're not necessary to fill up precision notation
		},
		minor x tick num={4},		%add four ticks in between major ticks on x axis
		minor y tick num={4},		%add four ticks in between major ticks on y axis
		xtick distance={.5},
		ytick distance={100},			%major ticks 0.1 on the y axis
		extra y ticks={0}			%add a tick in the origin
		]			
	\end{groupplot}
	\draw (2.3,-9.5)--(2.3,4);
	\draw (2.67,-9.5)--(2.67,4);	
	\draw (4.97,-9.5)--(4.97,4);	
	\end{tikzpicture}
\end{center}
\end{wrapfigure}

	\item Describe the motion of the ball at the six indicated times numbered 1 to 6. That is, describe where it is located and say something about its speed, direction of motion, and acceleration. If you look closely, you should see that at position 3 the ball is still in contact with the floor.
	
	\item Draw \forcediags{} for each of the six marked times. For which times are the \forcediags{} identical?
	
	\item For the times when the \forcediags{} are identical, which aspects of the motion are identical? Which aspects are different? Are your answers to the previous two questions consistent with Newton's 2nd law?
	
	\item Determine the average value of the force exerted by the floor on the ball between the times numbered 2 and 3 two different ways:
	\begin{enumerate}[(i)]
		\item from the impulse imparted to the ball from the floor and the velocity graph; and 
		\item using Newton's 2nd law.
	\end{enumerate}
}

\vspace{.1mm}
	Explain in two ways (one for each approach) why the force of the floor on the basketball [while it bounces] is so much greater than the basketball's weight.
\end{enumerate}



