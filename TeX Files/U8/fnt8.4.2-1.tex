\label{fnt8.4.2-1}

NOTE: A is the amplitude, T is the period of oscillation, f is a constant phase factor, and B is the equilibrium value of y(t), if it is not zero.  We will normally let B = 0.  The next several \FNTs{} have to do with making sense of this relationship.

Use the general expression above, with B = 0, to sketch the following graphs, using the same time axis for all of them (all on the same graph).  Your time axis should go from $t = 0$ to $t = \unit[10]{s}$.   
I)  A = 2, T = 5, \phi = 0           II)  A = 2, T = 5, \phi = \pi/2          III)  A = 2, T = 5, \phi = -\pi/2
Explain in words how you knew how to ``start'' each graph where t = 0.