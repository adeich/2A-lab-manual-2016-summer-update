\section{More Practice with the \EnergyInteractionModel{}}
\label{act1.1.8}

\begin{overview}

	\textbf{Overview:} We continue our efforts to practice using our models -- particularly the \EnergyInteractionModel{} -- by investigating new phenomena.
	
\end{overview}

\subsection{Cooling Water Below its Freezing Point}

\begin{fnt}
	\label{fnt1.1.4-3}

Perhaps you recall that when table salt, \ce{NaCl}, is added to water, the freezing point of water is lowered. Consider a system composed of a mixture of \unit[2.5]{kg} of ice and \unit[50]{g} of liquid water and a small, separate container of finely powdered salt. This physical system is contained in a fully insulated container that prevents all thermal interactions with the environment. Both the salt and the ice-water mixture are initially at the freezing point of water, \unit[0]{\textdegree C}. The salt is then added to the ice-water mixture, and the system of ice-water and salt is allowed to come to thermal equilibrium. The final equilibrium temperature is less than \unit[0]{\textdegree C}.\\

Use the \EnergyInteractionModel{} to predict if there will be a greater or lesser amount of ice in the final equilibrium state than in the initial state before the salt was added. Your explanation should include a complete \EnergyDiagram{}.\\

[The simplest way to model this physical system is with one thermal energy system for everything and one bond energy system, i.e., in terms of the model, it is not useful to distinguish between the various chemical components in order to answer this particular question.]
\end{fnt}

\note{Timing: \unit[\textless3]{min}}{
%Note:  You will have done these FNTs BEFORE coming to class, so you have only a few minutes (\unit[\textless3]{min}) to work on these together in your group.	
}

\noindent On the board, construct the simplest \EnergyDiagram{} that represents a particular model of the process described in this FNT. 

Make sure everyone in your group is ready to explain which elements of your completed diagram represent information that was given, or known, and which elements you had to infer, based on given or known information and the logic of the model? 

\WCD  

\begin{fnt}
	\label{fnt1.1.4-4}

\todo[inline]{FNT 1.1.4-4) change to thought experiment and we will do in class}

This FNT is a \textbf{\emph{thought experiment}} to make predictions about an actual experiment we will do in class, where you will get to observe the phenomenon described in \ref{fnt1.1.4-3}.\\

Imagine you fill an insulated cup almost full with chopped or crushed ice, and measure the temperature after a minute or two, once it's all come to thermal equilibrium. Since this ice is frozen water, the temperature should be at \unit[0]{\textdegree C} or not more than a couple of degrees below. Then, imagine you add a bunch of salt and stir it around. 

What do you think the lowest temperature you can attain will be? Why? What happens to the amount of liquid present if your keep stirring and adding salt? How can we understand this phenomenon in terms of thermal and bond energy systems?

Develop an explanation for the changes you would observe in this physical system (decrease in temperature and change of phase) in terms of the \EnergyInteractionModel{}.
\end{fnt}

\note{Timing: \unit[\textless3]{min}}{
	
}

\noindent Now we'll actually do this experiment:

\begin{enumerate}

	\item On your table, you have an insulated mug and a thermometer. Get some ice from the ice chest on the counter and fill your insulated mug. Measure the temperature inside the ice-filled cup and verify it is at around \unit[0]{\textdegree C}.

	\item Now, add some salt and stir it into the ice with a stirrer (don't use the thermometer; it might break!). Measure the temperature of the salt-ice mixture inside the cup. Add some more salt and repeat the stirring and temperature measurement. How far can you get the temperature to drop?

	\item On the board, list the low temperatures attained by your group.

\end{enumerate}

\noindent Would the following statement be something that would be good to memorize?

\begin{quote}
	{\em ``When bond energy increases, thermal energy must decrease, and vice versa.''}
\end{quote}

\noindent Explain on the board why it would or would not be good to memorize this.

\WCD  

\subsection{Heating Water with Microwaves}

\begin{fnt}
	\label{fnt1.2.1-5}

\todo[inline]{1.2.1-5) print FNT doesn't match Canvas FNT - change instructor notes to cover canvas FNT}

%\subsubsection*{Print Version}

%Use your insulated cup (and some larger microwaveable container, if available) and your thermometer to determine the effective ``cooking power'' of a microwave oven for different amounts of water. Try to use 4 or 5 different amounts of water spread over as wide a range of volumes as possible. [from a few tens of cubic centimeters (cc) to 5000 or more]. Use what you know about the thermal properties of water to design an experiment to test how much energy is transferred to the water in a given amount of time. From this measurement you can determine the actual power, in watts, that the microwave delivers to the water. This will typically not be the same for different amounts of water, so you should make measurements using different amounts. How does the maximum ``cooking power'' you measure compare to the electrical power the microwave uses (printed on the back of your microwave in watts)? Based on your data, what amounts of water corresponded to the most efficient use of electrical power?

%A microwave oven works by converting electrical energy to microwaves. Some of the energy in the microwaves is absorbed by the food/liquid placed in the oven. In terms of energy systems this means that the energy in the microwave system decreases and the thermal energy in the liquid/food increases. However, not all of the energy in the microwaves gets absorbed by the food/liquid. You are going to determine how much of the energy used by the microwave actually goes into heating your food.


%\subsubsection*{Canvas Version}

This is an actual experiment that you should perform at home:\\

\noindent Put approximately 1 cup of cold water in a microwave safe dish (Tupperware, drinking glass, etc.), measure the temperature (you can use a thermometer you would use to measure your body temperature), then heat it for a set period of time (somewhere between 30 seconds and a minute will work well). When you take the water out, measure the temperature again.

\begin{enumerate}[(a)]
	\item Draw an \EnergyDiagram{} for the water in your cup, and determine the change in thermal energy of the water. Convert this to Watts (a unit of power).
	
	\item Compare the number of Watts you determined in part a) to the power rating listed on the microwave information plate (usually located on the back of the oven, but you can always look up the model online). Which is greater? What does this mean?
	
	\item Where might the energy go if it is not heating up the food?
	
	\item Does the percentage of energy going to the food depend on the amount of water? You might choose a small amount, say half a cup, and a large amount, say two cups to investigate this question.
\end{enumerate}
\end{fnt}

\note{Timing: \unit[\textless3]{min}}{
	
}

\noindent On the board, choose the data from a member of your group, and sketch a graph of cooking power vs.\ amount of water. 

You'll notice that there appears to be some ``missing power.'' That is, the difference in power actually used to heat the water is less than the power used by the unit as stated on a label on the back of the microwave unit.

{[}\textbf{Hint:} Can you feel or hear anything near the back of the microwave unit when it is turned on?{]}

\WCD  
