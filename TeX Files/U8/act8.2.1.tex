\section[Analyzing the Motion of Dropped Objects]{Analyzing the Motion of Dropped Objects with the \FModel{}}
\label{act8.2.1}

\begin{overview}
	\textbf{Overview:} Now that we've spent some time getting to know Newton's laws -- especially in static cases without movement -- we'll use them to model and understand motion. We'll start with the case of falling objects.
\end{overview}

\noindent\textbf{Phenomenon:} Dropping a wadded up piece of paper and a flat bottomed coffee filter

\begin{benumerate}
	\bitem{Observing the Motion.}
	
	\begin{enumerate}
		\item Obtain a flat-bottomed coffee filter that still has its original shape. Wad up a piece of paper until it is in the shape and size of a golf ball. Drop both simultaneously from a height of about six feet.
		
		\item Carefully observe the motion of the two objects.
		
		\item Write out a description of the motion you observed. We talked about ``velocity'' before in this course -- use this concept in your description.
		
		\item Sketch a velocity graph on the board (only the vertical component) for each object on the same set of velocity versus time axes.
	\end{enumerate}

\WCD

	\bitem{Analyzing the Motion.}
	
	\begin{enumerate}
		\item Create \forcediags{} for both objects at two points in time, 
		\begin{enumerate}
			\item at the exact instant each object is released, and
			\item after each object has fallen about \nicefrac{1}{4} of the way toward the floor.
		\end{enumerate}
		Show the net force vector for each object at each point in time explicitly (use a thick line). What does this tell you about the accelerations of these two objects as they fall? Apply Newton's 2nd law to both objects in order to determine their accelerations.
		
		\item Use the accelerations you obtained in (a) along with your observations about the motion of each object to make an acceleration graph ($\vec{a}$ versus time) for the two dropped objects from the time they are released to just before hitting the floor.
		
		\item What is the mathematical relationship between the instantaneous acceleration and the instantaneous velocity? How does this relationship show up in the two graphs you made for the two cases? Adjust your curves to be consistent with both (b) and (c).
	\end{enumerate}
\end{benumerate}

\WCD
