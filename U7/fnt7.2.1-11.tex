\label{fnt7.2.1-11}

\begin{enumerate}[(a)]
	\item Wrap up what you have done in \ref{fnt7.2.1-9} and \ref{fnt7.2.1-10} by explaining, in as few words as possible, why the ball moves along the path you sketched.
	\item For any object to be moving in a curved path what is necessary about the relationships between $\vec{p}_i$, $\Delta \vec{p}$, $\vec{p}_f$, and $\sum F$?
	\item If $\Delta \vec{p}$ were always perpendicular to $\vec{p}_i$ what type of path would this result in?
\end{enumerate}