\section{Explicit Reasoning Using Models}
\label{act2.2.3}

\begin{overview}

	\textbf{Overview:} Now that we've discussed many different physical scenarios, our models probably have become second nature to you. Maybe so much so that we may have to revisit how to explicitly model a phenomenon. We'll do that with a new phenomenon.

\end{overview}

\subsection{Two Masses Over a Pulley -- Atwood's Machine}

\begin{benumerate}
	\bitem{Think and Discuss}
	
	Imagine this situation:  A string connecting a smaller mass, $m$, and a larger mass, $M$, passes over a small pulley that is almost frictionless.  The masses are initially held at rest; then are released and allowed to move:
	
	\begin{center}
		\includegraphics[width=0.1\linewidth]{act223-atwood}
	\end{center}
	
	Based on your intuition which set of attached masses, each differing by \unit[20]{g}, will move faster when released?
	
	\begin{center}
	\begin{tabular}{lll}
		\textbf{Set A:}	&	$M_A = \unit[220]{g}$,	&	$m_A = \unit[200]{g}$\\
		\textbf{Set B:}	&	$M_B = \unit[70]{g}$,	&	$m_B = \unit[50]{g}$
	\end{tabular}
	\end{center}
	\begin{center}
		\begin{tabular}{ll}
		\textbf{Circle your choice here:}	&	(a) Set A moves faster\\
								&	(b) Set B moves faster\\
								&	(c) Both move with the same speed
	\end{tabular}
	\end{center}	
	
	
	\bitem{Put your group's choice on the board}
	
	How sure are you about your reason(s) and predictions? What is your intuition based on?
	
	\bitem{Try it out}

	Pair up with an adjacent table.  One table will attach the heavier set of masses, the other table should attach the lighter set. Try it out and see which set moves faster. 
	
	What do you observe?

\WCD
\end{benumerate}

\subsection{Applying the \EnergyInteractionModel{}}
\label{act2.2.3-2}

Apply the \EnergyInteractionModel{} to the two-masses-over-a-pulley situation. Model this system as if the pulley is massless and frictionless, so you won't have to worry about energy systems associated with the pulley.\\

\noindent Take the \textbf{initial state} to be just as you release the masses (what is $v_i$? what is $\Delta v$?).\\

\noindent Take the \textbf{final state} to be when the masses have moved a distance $d$ and have speed $v$, but before they hit anything or run out of string.\footnote{Since $d$ denotes a distance, it is a positive number; so, $\Delta y = \pm d$, as appropriate.}

\begin{enumerate}
	\item Create a particular model of the phenomenon described above by constructing a complete \EnergyDiagram{} for \textbf{each} of the mass sets using the initial and final states described above.
	\item	\label{act2.2.3-(2)} When you substitute algebraic expressions for changes in individual energy systems in the algebraic representations of your particular \EnergyInteractionModels{}, you will find the symbols $m$, $M$, $g$, $v$, and $d$ useful. Watch your ``minus signs!''
	\item One important practical use of the \EnergyDiagram{} is in the interpretation of algebraic expressing of energy conservation. For example, what do each of the terms in the equation mean, and what should their sign be? Check, and be ready to show, that the signs of the terms in the equation from \eqref{act2.2.3-(2)} are consistent with the increases and decreases in energy of the energy systems in your diagram.
	\item\label{act2.2.3-(4)} What energy systems have you ignored by your assumptions about the pulley?
	
	[You don't have to put \eqref{act2.2.3-(4)} on the board.]
	
\WCD  
\end{enumerate}
 
\subsection{Reasoning and Explaining with Particular Models}
\label{act2.2.3-3}

You have constructed two particular \EnergyInteractionModels{}, one for each of the different sets of masses. You are now going to use these models to make sense of and explain why one set of the masses (the set with the smaller masses) moved faster than the other set.

\todo[inline]{Should the ``\EnergyInteractionModels{}'' here be ``\EnergyDiagrams{}''? The same goes for any uses of the word ``model.''}

\subsubsection*{Comparing the changes in energy of the various energy systems}

\begin{enumerate}
	\item During the process (from beginning to end), does the total $KE$ of the two masses (the sum of the $KE$s) increase, decrease, or stay the same? How do you know?  Is your response the same for both the heavier and the lighter pair of masses?
	\label{act2.2.3-(5)}
	
	\item During the process, does the total $PE$ of both masses (sum of the $PE$s) increase, decrease, or stay the same? How do you know?  Is your response the same for both the heavier and the lighter pair of masses?
	
	\item Compare the magnitudes of the total change in $PE$ of both masses that occurs during the process across the two systems (the heavier pair and the lighter pair).  Are the total changes the same or different?  Explain.
	
	\item Compare the magnitudes of the total change in $KE$ of both masses that occurs during the process across the two systems (the heavier pair and the lighter pair).  Are the total changes the same or different?  Explain.
	
	\item Did the original question regarding the two sets of masses have to do with the $KE$ of the masses or something else?  Was it related to the $KE$?  Compare this across the two systems (the heavy pair and the lighter pair).  Is this the same or different?  Explain.
	\label{act2.2.3-(9)}
\end{enumerate}

\subsection{Creating a Convincing Scientific Explanation}
\label{act2.2.3-4}

Turn your responses to the questions \eqref{act2.2.3-(5)} through \eqref{act2.2.3-(9)} from \ref{act2.2.3-3} into a set of statements that constitute a logical argument based on your particular \EnergyInteractionModels{} that explains why the set of masses with smaller total mass move faster than the set with greater total mass, as long as the difference between the two masses in each set is the same.\\

\WCD