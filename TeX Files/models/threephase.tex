\appendixchapter{\ThreePhaseModel{}}

\subsection*{Graphical Representation}

\begin{center}
\begin{tikzpicture}
	% title the diagram
	 \draw (7.5,7) node[align=center] {\textbf{\TempGraph{}}};

    % draw horizontal axis
    \draw[-{Stealth[scale=1.2]}, line width=1pt] (0,0) -- (14.5,0);
    % label the horizontal axis
    \draw (7.5,0) node[below=3pt] {Energy Added or Removed $\Delta E$ (at constant pressure)};

    % draw vertical axis
     \draw[-{Stealth[scale=1.2]}, line width=1pt] (0,0) -- (0,7);
    % label the vertical axis
     \draw (0,7.2) node[left=10pt, rotate=90] {Temperature $T$};
    % add melting point and boiling point on axis
     \draw[line width=1pt] (-3pt,1.8 cm) -- (3pt,1.8 cm) node[left=3pt] {\scriptsize{$T_\text{MP}$}};
     \draw[line width=1pt] (-3pt,3.8 cm) -- (3pt,3.8 cm) node[left=3pt] {\scriptsize{$T_\text{BP}$}};

	% draw first segment
	 \draw[line width=.8pt,dashed] (0,0) -- (1.5,1.8) node[below left=10pt, rotate=50.19442891] {\tiny{solid phase}};
	 
	% draw second segment
	 \draw[line width=.8pt,dashed] (1.5,1.8) -- (6,1.8) node[below left] {\tiny{mixed phase: solid/liquid}};
	 
	% draw third segment
	 \draw[line width=.8pt,dashed] (6,1.8) -- (8,3.8) node[below left=10pt, rotate=45] {\tiny{liquid phase}};

	% draw fourth segment
	 \draw[line width=.8pt,dashed] (8,3.8) -- (11.5,3.8) node[below left] {\tiny{mixed phase: liquid/gas}};

	% draw fifth segment
	 \draw[line width=.8pt,dashed] (11.5,3.8) -- (14,6.5) node[below left=10pt, rotate=47.20259816] {\tiny{gas phase}};
	 
	% draw labels
	 \draw[{Stealth[scale=1]}-] (1.5,1.8) -- (1,2.8) node[above,align=center] {\tiny{All solid at $T_\text{MP}$}};
	 \draw[{Stealth[scale=1]}-] (6,1.8) -- (5.5,2.8) node[above,align=center] {\tiny{All liquid at $T_\text{MP}$}};
	 \draw[{Stealth[scale=1]}-] (8,3.8) -- (7.5,4.8) node[above,align=center] {\tiny{All liquid at $T_\text{BP}$}};
	 \draw[{Stealth[scale=1]}-] (11.5,3.8) -- (11,4.8) node[above,align=center] {\tiny{All gas at $T_\text{BP}$}};
\end{tikzpicture}
\end{center}

\subsection*{Algebraic Representations}

\noindent Change in temperature $\Delta T$ of an amount $m$ of a substance with specific heat $C_p$ when energy is \emph{added} or \emph{removed} as \emph{heat} $Q$:
\begin{align*}
	\Delta T = \frac{Q}{C_p \cdot m}
\end{align*}

\noindent Amount $\Delta m$ of a substance with heat of melting/vaporization/sublimation $\Delta H$ that changes phase when energy is \emph{added} or \emph{removed} as \emph{heat} $Q$:
\begin{align*}
	\left|\Delta m\right| = \frac{Q}{H_p}
\end{align*}


\pagebreak


\newcommand{\leftcolumn}{0.35\linewidth}
\newcommand{\rightcolumn}{0.65\linewidth}

\noindent
\parbox[b]{\leftcolumn}{
	\textbf{Constructs}}
\parbox[b]{\rightcolumn}{
	\textbf{Relationships}}

\noindent
\hrulefill

\noindent
\begin{minipage}[c]{\leftcolumn}
	\baselineskip=1.5\baselineskip
	Pure Substances\\
	Three \emph{Phases}: \emph{solid}, \emph{liquid}, \emph{gas}\\
	Temperature\\
	Energy \emph{transferred} as\\
	\hspace*{5mm}\emph{Heat} or \emph{Work}
\end{minipage}
\begin{minipage}[c]{\rightcolumn}
	\begin{enumerate}
		\item {\em Pure substances} exist in one of three phases, depending on the \emph{temperature} and \emph{pressure}: solid, liquid, and gas. {\em Non}-pure substances, e.g., solutions and composites, require more complex models for analysis.
		\item To change either the \emph{temperature} or the \emph{phase} of a substance, \emph{energy} must be \emph{added} or \emph{removed}. Often, this energy is \emph{transferred} to or from the substance as \emph{heat} $Q$ but can also be transferred as \emph{work} $W$.\\[0.2mm]
	\end{enumerate}
\end{minipage}

\noindent
\begin{minipage}[c]{\leftcolumn}
	\baselineskip=1.5\baselineskip
	Phase Change Temperatures:\\
	\hspace*{5mm}$T_\text{MP}$, $T_\text{BP}$, $T_\text{SP}$\\
	\emph{Melting}, \emph{Boiling}, \emph{Sublimation}
\end{minipage}
\begin{minipage}[c]{\rightcolumn}
\begin{enumerate}\setcounter{enumi}{2}
	\item At constant pressure, changes of {\em phase} (solid~$\Leftrightarrow$~liquid and liquid~$\Leftrightarrow$~gas, or at some values of pressure, solid~$\Leftrightarrow$~gas) occur at specific temperatures: the \emph{phase change temperatures} $T_\text{MP}$, $T_\text{BP}$, and $T_\text{SP}$. They have particular values for each pure substance. The values of these temperatures are the same when ``going through'' the respective phase change in ``both directions.'' However, phase change temperatures are \emph{dependent on the pressure}.\\[0.2mm]
	\end{enumerate}
\end{minipage}

\noindent
\begin{minipage}[c]{\leftcolumn}
	\baselineskip=1.2\baselineskip
	Heat of \emph{Melting}\\
	Heat of \emph{Vaporization} (Boiling)\\
	Heat of \emph{Sublimation}\\[0.1mm]
\end{minipage}
\begin{minipage}[c]{\rightcolumn}
	\begin{enumerate}
		\item[] The amount of \emph{energy added} or \emph{removed} during a particular \emph{phase change} (written as $\Delta H$, indicating constant pressure) is unique to each substance and has been measured and tabulated for most substances.\\[0.2mm]
	\end{enumerate}
\end{minipage}

\noindent
\begin{minipage}[c]{\leftcolumn}
	\baselineskip=1.5\baselineskip
	Thermal Equilibrium\\
	Mixed phase
\end{minipage}
\begin{minipage}[c]{\rightcolumn}
	\begin{enumerate}
		\item[] If the substance is in \emph{thermal equilibrium} (i.e., if the entire substance is at the same temperature) at the \emph{phase change temperature}, both phases will \emph{remain} at the phase change temperature as the phase change occurs. \emph{Mixed phases} can exist in thermal equilibrium {\em only} when the temperature has the value of the phase-change temperature.\\[0.2mm]
	\end{enumerate}
\end{minipage}

\noindent
\begin{minipage}[c]{\leftcolumn}
	\baselineskip=1.5\baselineskip
	Heat Capacity\\
	Specific Heat
\end{minipage}
\begin{minipage}[c]{\rightcolumn}
	\begin{enumerate}\setcounter{enumi}{3}
		\item Changes of {\em temperature} of a substance occur when energy is added or removed while the substance is \emph{not} at a phase-change temperature.

		When the energy added is in the form of heat, the change in temperature $\Delta T$ is related to the amount of energy added by a property of the substance called \emph{heat capacity} $C$. The \emph{specific heat} $C_p$ (heat capacity per unit mass) has a particular value for each substance. This value has been measured and tabulated for most substances (see Table \ref{HeatCapTab} in Appendix \ref{UnitsConv}).
	\end{enumerate}
\end{minipage}