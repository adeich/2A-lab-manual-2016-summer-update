\section{Exploring Motion Graphs}
\label{act8.2.2}

\begin{fnt}
	\label{fnt8.2.1-1}

State whether the acceleration is positive, negative, or zero for each of the position functions $\vec{x}(t)$ in the position versus time graphs below. How do you know?
\end{fnt}

\begin{fnt}
	\label{fnt8.2.1-2}

For the each of the following scenarios make a position vs.\ time graph.  Directly below it draw a velocity vs.\ time graph, and beneath that draw acceleration vs.\ time. 

\begin{enumerate}[(a)]
	\item A dropped object as it is falling (before it hits the floor). 
	\item A rocket firing its engines for a certain length of time descending on Mars.
	\item A racecar during the first 10~seconds after it starts from a stop.
\end{enumerate}
\end{fnt}

\begin{fnt}
	\label{fnt8.2.1-3}

Beneath the appropriate column of graphs from \ref{fnt8.2.1-2}, write an equation that solves for the variable in question (see below).  Write which model you used: \FModel{}, \EnergyInteractionModel{}, the \pConsModel{}, %the \LConsModel{}, 
etc... If you introduce any new variables, clearly indicate what they mean.

\begin{enumerate}[(a)]
	\item The velocity of a dropped object just before it hits the floor. 
	\item The time it takes the dropped object in (a) to reach the floor after being dropped.
	\item The change in velocity for a spacecraft firing its rockets for a certain length of time.
	\item The speed of the racecar after the first 10~seconds.
\end{enumerate}
\end{fnt}

%\begin{fnt}
%	\label{fnt8.2.1-4}

Interpret the graph to the right as representing the velocity of an object. Rank the points in order of increasing acceleration (from most negative to most positive). Practice walking this plotted motion.
%\end{fnt}

What kinds of questions can the various models answer?

In Your Small Group discuss the following and put your responses on the board.

Your instructor will assign each group one of the following tasks:

\begin{enumerate}
	\item \textbf{Group 1:}
	\begin{enumerate}
		\item Decide on your group's answer to \ref{fnt8.2.1-1}, redraw each graph and explain next to it how you know whether the $x$-component of the acceleration is positive, negative, or zero.
		\item In each case, is the $x$-component of the velocity positive or negative?
		\item Does a negative acceleration always mean a decreasing speed? Come up with a relationship between the sign of acceleration and the change in velocity and write this on the board.
		\item Go to the boards of the groups working on \ref{fnt8.2.1-2}. Check each group's answers and graphs and assign them a ``grade'' for their work on \ref{fnt8.2.1-2}; seriously do this. Grade their graphs taking into account the initial points, the signs of the slope of the line in each graph, and how the slope of the line in each graph changes with time. Don't give them a low grade just because their answer doesn't look like your answer. Only give them a low grade if their answer is an unreasonable one for the problem at hand.
	\end{enumerate}
	
	\item \textbf{Groups 2, 3, \& 4:} For your group's assigned scenario from \ref{fnt8.2.1-2}, make a column of three graphs with the top most being position vs.\ time and the bottom most acceleration vs.\ time. Line up your graphs so that a vertical line through all three would correspond to the same instant in time in each.
	
	\item \textbf{Group 5:} Go to the boards of the groups working on \ref{fnt8.2.1-2} and answer \ref{fnt8.2.1-3} by writing the appropriate equation and naming which model is used.
	
	\item \textbf{All groups:} What kinds of questions can the \FModel{} answer that the conservation models cannot answer? Explain why. (Use what you have learned from lecture and/or the Course Notes to help you answer this question.) Give two specific examples.
\end{enumerate}

\WCD
