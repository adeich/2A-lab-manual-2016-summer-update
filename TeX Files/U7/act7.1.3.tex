\section{The Importance of $\Delta t$ in Collisions}



\subsection{Automobile crash}

\begin{overview}
\textbf{Overview:} These \FNTs{} and the next part of the activity --- pulling a tablecloth out from under the dishes --- illustrate how the time duration of an impulse relates to the net force $\Sigma F$ and $\Delta \vec{p}$.
\end{overview}

Use the following phenomenon for \ref{fnt7.1.1-4}, \ref{fnt7.1.1-5}, and \ref{fnt7.1.1-6}.

\textbf{Phenomenon:} You are riding in a car that crashes into a solid wall.  The car comes to a complete stop without bouncing back.  The car has a mass of \unit[1500]{kg} and has a speed of \unitfrac[30]{m}{s} before the crash (this is about \unitfrac[65]{mi}{hr}). 

\begin{FNTenv}
	\label{fnt7.1.1-4}

%You are riding in a car that crashes into a solid wall.  The car comes to a complete stop without bouncing back.  The car has a mass of \unit[1500]{kg} and has a speed of \unitfrac[30]{m}{s} before the crash (this is about \unitfrac[65]{mi}{hr}).  

\begin{enumerate}[(a)]
	\item What is the car's initial momentum?
	\item What is your initial momentum? Recall that the weight of one kilogram is \unit[2.2]{lbs}.
	\item Draw separate \pcharts{} for the car and the person. Treat both as open systems with a net impulse.
	\item What is the change in the momentum of the car?
	\item What is the change in your momentum?
\end{enumerate}
\end{FNTenv}

\begin{FNTenv}
	\label{fnt7.1.1-5}

%You are riding in a car that crashes into a solid wall.  The car comes to a complete stop without bouncing back.  The car has a mass of \unit[1500]{kg} and has a speed of \unitfrac[30]{m}{s} before the crash (this is about \unitfrac[65]{mi}{hr}).  

\begin{enumerate}[(a)]
	\item What is the net impulse that acts on the car to bring it to a stop?
	\item What is the net impulse that acts on you to bring you to a stop?
\end{enumerate}
\end{FNTenv}

\begin{FNTenv}
	\label{fnt7.1.1-6}

%You are riding in a car that crashes into a solid wall.  The car comes to a complete stop without bouncing back.  The car has a mass of \unit[1500]{kg} and has a speed of \unitfrac[30]{m}{s} before the crash (this is about \unitfrac[65]{mi}{hr}).  

Refer to your \pcharts{} from \ref{fnt7.1.1-4}. Consider the following two situations.

\begin{enumerate}[I.]
	\item You remain buckled into the seat and the seat remains attached to the center of the car.
	\item You are not buckled into your seat and you fly through the windshield and hit the wall.
\end{enumerate}

\begin{enumerate}[(a)]
	\item Is the impulse the same in both cases?
	\item Are the forces acting on you the same?
	\item Using the words impulse, force, time, and momentum explain why one scenario is safer for you.  Hint: Think about how the shape of the car changes when it hits the wall.
\end{enumerate}
\end{FNTenv}

\begin{enumerate}
	\item In your group decide on two specific scenarios (\ref{fnt7.1.1-6}) and use these scenarios below (in Parts~2, 3, and 4) to answer all of the other questions in this set of \FNTs. For these two scenarios, think about how much time passes between the time the force is first applied by the object and the time when you have zero momentum. In which scenario will this time difference, $\Delta t$, be larger and why?
	
	\item  On the board, put up complete \pcharts{} (with force diagrams and equations worked through to numerical values) for the car and for the person in each of the two scenarios you have chosen. Describe the scenario above each of the \pcharts{} for the person.
	
	\item Use your \pcharts{} to explain why the net force acting on the person will not be the same in both scenarios.
	
	\item For each of the scenarios that you described in (\ref{fnt7.1.1-6}), you will estimate the magnitude of the average force that would have been acting on you to bring you to a stop.
	\begin{enumerate}
		\item You will have to determine the time during which the impulse acts for the different scenarios. To do this you must first decide the initial and final momentum for the cases you are describing and over \emph{what distance} the impulse acts.
		
		\item Now you need to \emph{calculate} the time duration of the impulse using your knowledge of how distance and time are related. If you assume that you slow down at a constant rate, then your average speed during this time is one-half your initial speed.
		
		\item Make an estimation of the average force for the two situations. If your answers differ for the situations explain what factor is causing this difference.
	\end{enumerate}
	
	\item If you know the initial and the final momentum, then you know the impulse. In each of your scenarios did you have the same or a different impulse? What then is the effect of changing $\Delta t$ (with the given constraints in the case of this automobile crash)?
\end{enumerate}

Be prepared to explain what $\Delta t$ is and why it is important in an momentum problem!

\WCD

\subsection{The Tablecloth Trick}

\textbf{Phenomenon:} If a tablecloth (large piece of paper) is pulled quickly enough from under the objects sitting on it, those objects slide only a very short distance on the table top after the tablecloth has been pulled out from under them. This implies that they acquired only a small velocity from the table cloth moving out from under them. If the table cloth is pulled a little less quickly, the objects slide a little further on the table top, implying they acquired a slightly greater velocity.

\textbf{Try the trick:} Use a mass or another object and a fairly large piece of paper. Try pulling at different rates, so that the mass
	\begin{itemize}
		\item moves a lot (but does not fall off the table) and
		\item moves very little.
	\end{itemize}
	In all cases, make sure you are pulling sufficiently fast so that the objects are continually sliding on the paper as it is being pulled. That is, you need to pull sufficiently fast so that the objects do not move with the paper.
	
\textbf{Our goal is to make sense of this phenomenon using the relation of impulse to force and the time interval over which it acts.}

\subsubsection*{Establishing which Part of the Phenomenon we need to focus on}

\begin{enumerate}
	\item The momentum of an object is first increased as the paper is pulled out from under it. Then the momentum is decreased back to zero as it slides to a stop on the table. How far it slides on the table after the paper is pulled out from under it is a qualitative measure of the speed it acquired when the paper was sliding under it: the greater the distance it slid, the greater the speed it had acquired.
	
	So thinking in terms of impulse and momentum, \textbf{which force} acting through \textbf{what time interval} determined the maximum speed the object acquired before sliding to a stop on the table top?

\hspace{-\textwidth}\hspace{\linewidth} \textbf{Brief}
\hspace{\textwidth}\hspace{-\linewidth}
\WCD

	Express the friction force, which is the horizontal component of the force the paper exerts on the object, $F_{||\text{ paper on object}}$, as a constant (coefficient of friction $\mu$) times the object's weight $mg$.\footnote{Note that the coefficient will depend on the surfaces of the two materials in contact, but for reasonably smooth surfaces like paper and metal, it generally has a value in the range of 0.5 to 1.0.} \emph{The important point for the analysis here is that \textbf{the friction force is only proportional to the weight of the object} $mg$ and does \emph{not} depend on the speed of the pull.}
	
	\item In your small group, you will make two complete \pcharts{} including a force diagram for one of the objects on your table (such as keys, small bottles, etc.) that is sitting on a piece of paper, which is pulled out from under it. Make two \pcharts{} for this object: one for when the paper is pulled quickly and the second when it is pulled less quickly. Consider the interval to be just before the pull until immediately after the paper is no longer under the object.
	
	\item Make sure all forces in the force diagrams are appropriately labeled. Does how fast you pull the tablecloth affect the net force acting on the object? Hint: Identify exactly what things are exerting forces on your object.
	
	\item Develop an explanation of these phenomena using the \pcharts{} you have prepared. Hint: Start by writing out an expression for impulse.
	
	Extra question: Try pulling different objects. Do all the objects appear to move about the same distance for a given pull? How can you explain this?
	
	\item If you know the initial and the final momentum, then you know the impulse. In each of your scenarios did you have the same or a different impulse? So what is the effect of changing $\Delta t$? How does this compare from your response in Question~5 for the car crash? Which parameters are variable and which are constrained between these scenarios?
\end{enumerate}

Be ready to illustrate and give your explanation to the whole class.

Be prepared to explain what $\Delta t$ is and why it is important in a momentum problem!

\WCD