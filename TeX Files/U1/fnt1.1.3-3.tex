\label{fnt1.1.3-3}

We want to analyze triggering step (b) in \hyperref[fnt1.1.3-2]{\thefnt} more closely using the \TempGraph{} of the \ThreePhaseModel{}. This will help us get a deeper understanding of this part of the process and will enable us to extend the \ThreePhaseModel{} to make it more realistically explain actual phenomena. (It turns out that super cooling and super-heating are rather common. Usually, however, they occur over a very small temperature range and so go unnoticed.)

\begin{enumerate}
	\item Sketch a \TempGraph{} for sodium acetate between room temperature at approximately \unit[150]{\textdegree C}, assuming the sodium acetate does not undergo any super-cooling, i.e., assuming that the heat pack is solid at room temperature, and changes phase at its normal phase change temperature, which you determined in \hyperref[act1.1.2]{Activity~\ref*{act1.1.2}}.
	
	\item On the same diagram, sketch the path representing the process of the liquid heat pack cooling down from \unit[150]{\textdegree C} to room temperature with no phase change (now assuming there is supercooling). Check that the path you sketched makes sense by verifying that the changes in temperature and energy shown on the diagram as you move along the path you sketched match what you know about the actual changes in temperature and energy of the heat pack as it cools to room temperature without changing phase from liquid to solid.
\end{enumerate}