\section*{\DLM~\ref{dlm7b} Overview}

\subsection*{Brief Overview}

\subsection*{\CLASP{} Activities}
\hyperref[act2.3.1]{Activity \ref{act2.3.1}} \nameref{act2.3.1}:	(\about\unit[?]{min})

\textbf{Purpose:}
\begin{itemize}
\item Provide practice working with the conservation of energy equation in the form of $E_{total}$ = constant = $PE + KE$
\item Introduction to graphing energies as a function of the distance parameter of the PE when the total energy is constant and can be expresses as $E_{total}$ = constant = $PE + KE$
\end{itemize}
\textbf{Learning Outcomes:}
\begin{itemize}
\item Realizing and Understanding that in those situations when there is only one other energy system, besides $PE_{grav}$, then, since $PE_{grav}$ depends only on height, the other energy system depends only on height as well.
\item Understanding the graphical interpretation of a conservation of energy equation involving two energy systems, one of which depends explicitly on the observable position.
\item Understanding the constancy of the {\em total energy} and its graphical interpretation in a graph of two energy systems, both plotted as a function of a position.
\end{itemize}

\hyperref[act2.2.3]{Activity \ref{act2.2.3}} \nameref{act2.2.3}:	(\about\unit[30]{min})

\textbf{Purpose:}
\begin{itemize}
\item Provide practice applying the Energy-Interaction Model to more complicated mechanical systems.
\end{itemize}
\textbf{Learning Outcomes:}
\begin{itemize}
\item Gain more confidence in using the energy-interaction model to analyze more complicated mechanical systems.
\item Gain more confidence in using the algebraic results obtained from application of the model to gain insight into the physical situation.
\end{itemize}


\subsubsection*{Reminder}

If there is a single most important task for the \DL{} instructor working in the classroom, it is to get the students up out of their chairs and talking together about whatever activity they are working on. There are prompts written into the activities instructing them to do that (e.g., All members of your group must now go to the board and work on the following together), but most of the time they need additional direction from the \DL{} instructor. You should even discourage them from putting up separate parts simultaneously, instead telling them to all work on each part, discussing and making sure everyone in the group understands the response that is being put on the board.  


%%%%%

%\subsubsection*{Caution}
%Watch out for the order in today�s activities. \hyperref[act2.2.3]{Activity~\ref{act2.2.3}} is last, after \hyperref[act2.3.1]{Activity~\ref{act2.3.1}}. This is to insure that students finish \hyperref[act2.3.1]{Activity~\ref{act2.3.1}}, which they must do in order to be able to do the \FNTs.


%\subsubsection*{Equipment needed in the room:}
%\begin{itemize}
%\item Plenty of graph paper. 
%\item Hanging mass-spring at each table. 
%\item Mounted pulleys with string and masses. 
%\end{itemize}