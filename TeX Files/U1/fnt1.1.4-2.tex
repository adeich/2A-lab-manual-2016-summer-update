\label{fnt1.1.4-2}

\begin{enumerate}[(a)]
	\item Think about the general case of chemical reactions. When a single compound breaks up into separated atoms, what can you say with absolute certainty regarding the \textbf{\em change} in bond energy of that compound? Why? What can you say with absolute certainty regarding the change in bond energy of a compound that is formed from separated constituent atoms? Why?
	
	\item Consider this chemical reaction: the combustion of propane.
	\begin{center}
		\ce{C3 H8 + 5 O2  ->  3 CO2 + 4 \ce{H2O}}
	\end{center}
	We can model this process as if the reactant molecules are broken down into their constituent atoms, which are then re-assembled into the product compounds.

	\begin{enumerate}[i.]
		\item Represent this process with an \EnergyDiagram{} that contains four separate bond energy systems, one for each molecular species. There will be one for the \ce{C3 H8}, one for the \ce{5 O2}, etc.
	
		\item Using your result from part (a), show the direction of each bond energy change with an arrow in the standard way. The indicator for each compound is the number of moles of that compound.
		
		\item Show the initial and final values of each of the indicators in the standard way.
	\end{enumerate}
	
	\item The magnitude of the bond energy changes for all the molecules involved in this process when they are separated into atoms are:

	\begin{center}
	\ce{C3 H8}: \unitfrac[4002]{kJ}{mol}; \;
	\ce{O2}: \unitfrac[495]{kJ}{mol}; \; 
	\ce{CO2}: \unitfrac[1607]{kJ}{mol}; \;
	\ce{H2O}: \unitfrac[925]{kJ}{mol}.
	\end{center}
	
	\begin{enumerate}[i.]
		\item Determine the bond energy changes, the $\Delta E_\text{bond}$, for each of the reactants and products.
		\item Write a conservation of energy equation in the standard way  ($\Delta E_1 + \Delta E_2 + \Delta E_3 + \Delta E_4 = Q$). Write this equation out with appropriate subscripts to clearly identify each term.
		\item Rewrite it with the numerical values being careful to get the algebraic sign correct.
		\item Calculate $Q$, which will be the heat released from the combustion of one mole of propane.
		\item Is the algebraic sign of $Q$ that you calculate consistent with our sign convention for $Q$?
	\end{enumerate}
\end{enumerate}