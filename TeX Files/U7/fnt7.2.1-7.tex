\label{fnt7.2.1-7}

Two asteroids identical to those in \ref{fnt7.1.1-7} collide at right angles and stick together. ``Collide at right angles'' means that their initial velocities were perpendicular to each other. You can assume that Asteroid A initially moved to the right and Asteroid B initially moved up.

Use the \pModel{} (make a complete \pchart{}) to find the velocity (magnitude \emph{and} direction, expressed as the angle with the initial velocity vector of Asteroid A) of the asteroids after the collision.  