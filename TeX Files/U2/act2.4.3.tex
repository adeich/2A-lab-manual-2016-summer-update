\todo{This badly needs formatting! Skipping for now because this is already printed and it does not seem like we have these \FNTs{} in this version of the manual. Need to look at this before next semester!}
\note{For Activity~\ref{act2.4.1} (\about\unit[30]{min})}{
	(\FNT~\ref{fnt2.4.1-1} and Practice Problem 1)
	\subsection{Purpose}Provide practice working with the relationship between force and PE and the graphical representations of this relationship
	\subsection{Learning Outcomes:}
\begin{itemize}	\item A solid understanding of the relation of force to change in potential with respect to distance
	\item The ability to switch between graphical representations, physical phenomena, and mathematical representations in the context of various potentials
		\item Understanding of the reasonableness of defining the zero of potential where the particles are sufficiently far apart to be no longer interacting.
\end{itemize}
}
\label{act2.4.3}

\begin{itemize}
\item Groups 1 \& 2 discuss and respond to \FNT~\ref{fnt2.4.1-1} as directed below. 
\item Groups 3 \& 4 discuss and respond to \FNT~\ref{fnt2.4.2-1} as directed below. 
\item Groups 5 \& 6 discuss and respond to \FNT~\ref{fnt2.4.2-2} as directed below. 
\end{itemize}

\begin{FNTenv}
	
  
2.4.1-1) Consider a ball thrown upward from the surface of the Earth, and the same ball thrown upward from the surface of the moon with the same initial speed. (What does that imply about their maximum PE?) Construct a complete \EnergyDiagram{} for each of these situations. Then carefully plot a graph (on graph paper) of $PE_\text{gravity}$ vs. y (height) from the point of release to the maximum height for each ball. (Put both on the same graph.) Set the origin (where y = 0) at the height the ball is released. Assume that gmoon ≅ (1/6)gearth,  and that the magnitude of the force of the moon on the ball is approximately 1/6 the magnitude of the force of the earth on the ball. 
How do the change in potential energy and the maximum height compare for the two situations? Make sure this is reflected in your graph. 
\end{FNTenv}

Follow-up:

\FNT~\ref{fnt2.4.1-1}  (\unit[\textless10]{min})  
\begin{enumerate}
\item Come to a consensus in your \SG{} about how the changes in $PE_\text{grav}$ compare for the two situations. \label{fnt2.4.1-1a}
\item Come to a consensus in your \SG{} about how the maximum heights compare.  \label{fnt2.4.1-1b}
\end{enumerate}

All members of your group must now go to the board and work on this together 

\begin{itemize}
\item After you reach consensus (and check with your instructor), sketch the two plots of $PE_\text{gravitational}$ on the same graph of Energy vs.\ Height.  
\item You know that the force of gravity is constant near the surface of the earth and near the surface of the moon, and that it is approximately six times stronger for the earth than for the moon.  
\end{itemize}
Make sure everyone in your \SG{} is ready to explain how your graph conveys this information.  


All members of your group must now go to the board and work on this together 

\begin{FNTenv}
	\label{fnt2.4.2-1}

Refer to \hyperref[act2.4.2]{Activity~\ref*{act2.4.2}}, ``A New Physical situation: Two Masses connected by a Spring'', part A (1).

\begin{enumerate}[(a)]
	\item Redraw that graph.  Now draw a second curve, on the same graph, for the same two-mass system that has a spring with a spring constant twice as large.  ($r_0$ does not change.).
	
	\item On a new graph, redraw the same graph from part A(1).  Now draw a second curve, on the same graph, for the same two-mass system with the same spring constant, but with an $r_0$ that is twice as big.
	
	\item What physically is the difference between the systems in part a)? part b)?
\end{enumerate}
\end{FNTenv}

\FNT~\ref{fnt2.4.2-1}  (\unit[\textless10]{min})  
\begin{itemize}
\item For two masses connected by a spring the potential energy is given by: 
$PE_\text{spring} = \frac{1}{2} k(r-r_0)^2$
How does doubling the force constant affect $PE_\text{spring}$ at any particular value of $r$? 
\item On one graph, sketch two plots of $PE$ for the two different springs. Pick some particular value of $r$, and show explicitly on your graph how doubling the spring constant affects the potential energy of the system. How do these two systems differ physically?
\item On a second graph, sketch two plots of $PE$ for the two different values of $r_0$. How do these two systems differ physically?  
\end{itemize}


All members of your group must now go to the board and work on this together 

\begin{FNTenv}
	\label{fnt2.4.2-2}

As you know, the electrical force between two positively charged objects is always repulsive and becomes weaker as distance increases, eventually approaching zero as separation goes to infinity.

\begin{enumerate}[(a)]
	\item Relationship 2 of the Particle Model of Matter gives the relation between the force acting between two particles and the potential energy due to this force: the force is equal to the derivative of the potential energy with respect to the particle separation. The force is always in the direction that lowers the potential energy.  Use this relationship to deduce the shape of the graph of $PE$ as a function of $r$ over all possible separations.
	\label{2.4.2-2a}
	
	\item Redo Part~\eqref{2.4.2-2a} for two negative charges (Hint: what will be different and what do you know about the forces between the two negative charges?)
	
	\item Redo Part~\eqref{2.4.2-2a} for two unlike charges (Hint: what will be different and what do you know about the forces between these two charges?)
\end{enumerate}

\end{FNTenv}

\FNT~\ref{fnt2.4.2-2}  (\unit[\textless10]{min})  
\begin{itemize}
\item Why do we define the $PE$ between atoms or molecules to be zero when $r$ is very large (i.e., when the atoms are far apart instead of when they are close together)? 
\item Sketch three separate graphs of $PE$ vs.\ separation distance on the board: one for a pair of positively charged particles, one for a pair of negatively charged particles, and one for two particles with opposite charges. Make sure everyone in your \SG{} is ready to explain why the graphs are drawn as they are.
\end{itemize}


\WCD
