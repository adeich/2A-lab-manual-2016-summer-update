\label{fnt1.2.1-5}

\todo[inline]{1.2.1-5) print FNT doesn't match Canvas FNT - change instructor notes to cover canvas FNT}

%\subsubsection*{Print Version}

%Use your insulated cup (and some larger microwaveable container, if available) and your thermometer to determine the effective ``cooking power'' of a microwave oven for different amounts of water. Try to use 4 or 5 different amounts of water spread over as wide a range of volumes as possible. [from a few tens of cubic centimeters (cc) to 5000 or more]. Use what you know about the thermal properties of water to design an experiment to test how much energy is transferred to the water in a given amount of time. From this measurement you can determine the actual power, in watts, that the microwave delivers to the water. This will typically not be the same for different amounts of water, so you should make measurements using different amounts. How does the maximum ``cooking power'' you measure compare to the electrical power the microwave uses (printed on the back of your microwave in watts)? Based on your data, what amounts of water corresponded to the most efficient use of electrical power?

%A microwave oven works by converting electrical energy to microwaves. Some of the energy in the microwaves is absorbed by the food/liquid placed in the oven. In terms of energy systems this means that the energy in the microwave system decreases and the thermal energy in the liquid/food increases. However, not all of the energy in the microwaves gets absorbed by the food/liquid. You are going to determine how much of the energy used by the microwave actually goes into heating your food.


%\subsubsection*{Canvas Version}

This is an actual experiment that you should perform at home:\\

\noindent Put approximately 1 cup of cold water in a microwave safe dish (Tupperware, drinking glass, etc.), measure the temperature (you can use a thermometer you would use to measure your body temperature), then heat it for a set period of time (somewhere between 30 seconds and a minute will work well). When you take the water out, measure the temperature again.

\begin{enumerate}[(a)]
	\item Draw an \EnergyDiagram{} for the water in your cup, and determine the change in thermal energy of the water. Convert this to Watts (a unit of power).
	
	\item Compare the number of Watts you determined in part a) to the power rating listed on the microwave information plate (usually located on the back of the oven, but you can always look up the model online). Which is greater? What does this mean?
	
	\item Where might the energy go if it is not heating up the food?
	
	\item Does the percentage of energy going to the food depend on the amount of water? You might choose a small amount, say half a cup, and a large amount, say two cups to investigate this question.
\end{enumerate}