%\section{Manipulating the Interval}

%Changing this to just be the ice-tea activity
\section{Defining Appropriate Intervals for Analysis}
\label{act1.2.3}

\begin{overview}

\textbf{Overview:} Remember that we \hyperref[act1.2.2D]{talked about} how asking different questions makes us choose different intervals of interest for our investigations using the \EnergyInteractionModel? Here, we'll explore this issue a bit further.

\end{overview}

% We already did this activity and FNT in D/L 4. Commenting this out here.
%\subsection{}
%
%\begin{fnt}
%	\label{FNT1.2.1-4}

\ref{FNT1.2.1-2} and \ref{FNT1.2.1-3} emphasized the significance of the beginning and end of the interval -- we've called them ``initial'' and ``final'' states so far. This \FNT{} illustrates
\begin{enumerate}[(i)]

	\item that the question you're asking determines the interval being analyzed with the \EnergyInteractionModel{}; and

	\item that sometimes it's necessary to adjust the size of the interval in order to be able to use the model.

\end{enumerate}

\noindent The physical process we want to analyze with the \ThreePhaseModel{} and the \EnergyInteractionModel{} is the following:\\

\noindent Imagine you are using your hot pot to gradually heat a \unit[500]{g} block of ice that has just been removed from a freezer with an internal temperature of \unit[-25]{\textdegree C}. The hot pot is fairly well insulated, so it is reasonable to assume that all of the heat put into the pot from the electrical heater located in the bottom of the pot goes into the \ce{H2O}. We can also assume that the heat capacity of the pot is much smaller than the heat capacity of \unit[500]{g} of \ce{H2O} (whether solid or liquid), so we can ignore the thermal energy system of the pot itself.

One question we could ask is, ``What is the final state of the water (phase and temperature) after the addition of \unit[252]{kJ} of heat?'' 

\begin{enumerate}[(a)]

	\item Explain why -- based {\em only} on the given information and without further analysis or calculation -- it is \textbf{\em not possible} to construct a {\em single} \EnergyDiagram{} that could be used to determine the final state of the water.
	
		[Hint: try to construct such a diagram. Another way to think about this is whether you can define the final state so that it depends on only one variable without making unjustified assumptions?]

	\item In cases like this, you must first use the model to analyze a {\em shorter interval}. Can you construct an \EnergyDiagram{} that could be used to determine how much energy is needed to increase the temperature of the ice to \unit[0]{\textdegree C}?
	
		Now explain in a few sentences how to proceed using the \EnergyInteractionModel{} to find the final state of the \ce{H2O}.

\end{enumerate}
%\end{fnt}

%\note{Timing: \unit[\about10]{min}}{
	
%}

%\begin{enumerate}
%	\item Why is it {\em not} possible to apply the \textbf{\EnergyInteractionModel{}} in a straightforward way to the interval of interest? (The interval that ends when all of the \unit[252]{kJ} has been added to the water.) In other words, why can't the overall interval be diagrammed using {\em only} the given information prior to doing any further analysis?   Draw a \TempGraph{} to help you make sense of this and to use to explain it. In particular, be sure to make explicit which part of the graph in the diagram refers to which ``energy bubble'' on the \EnergyDiagram{}. \textbf{[To save time, it is ok for explanations on the board to be more fragmentary and shorthand than explanations on exams, because anything that is unclear can be easily clarified in the Whole Class Discussion.]}
	
%	\item Explain how to use the \EnergyInteractionModel{} to analyze {\em shorter intervals} in order to determine the final state (phase and temperature) of the \ce{H2O}. Use a \TempGraph{} in your analysis and to use in your explanation.
%\end{enumerate}

%\WCD 

%\subsection{Two objects coming to thermal equilibrium}
\todo[inline]{Notes on Ice-Tea activity

	Needs to be restructured as chart addition doesn't fit well. Perhaps do original ice tea activity and THEN do chart if time?}

\noindent\textbf{Consider the following phenomenon:} A \textbf{big} chunk of ice (water) at an initial temperature of  \unit[-65]{\textdegree C} is placed inside a well-insulated container with some tea at an initial temperature of \unit[20]{\textdegree C}, and the two are allowed to come to thermal equilibrium. (The tea can be treated as water with respect to thermal properties.)

\begin{enumerate}
	\item There are several possible final states of this process. We will be using and reusing the same \ThreePhaseModel{} to determine possible final states. Draw two \TempGraphs{}, one for tea and one for \ce{H2O}.
	\begin{enumerate}
		\item Why can't either substance reach thermal equilibrium in the gaseous phase nor the mixed (liquid/gas) phase nor the gaseous phase?
		\item Using the \hyperref[act123-grid]{grid below}, your instructor will assign your group one of the potentially possible final states. Trace your \TempGraph{} to show how your assigned ending phase is or is not possible.
		\label{act123-gridprob}
	\end{enumerate}

\newcommand{\gridsize}{3cm}
\begin{table}[h]
	\caption{Grid for \ref{act123-gridprob}}
	\label{act123-grid}
	\begin{tabular}{cr>{\centering}b{\gridsize}>{\centering}b{\gridsize}b{\gridsize}<{\centering}}
							&						& \multicolumn{3}{c}{\textbf{Tea}} \\
							&						& Solid 						& Mixed Phase\newline\scriptsize(Solid/Liquid) & Liquid \\\cline{3-5}
							&						& \multicolumn{1}{|l|}{i.} 			& \multicolumn{1}{|l|}{ii.}	& \multicolumn{1}{|l|}{iii.} \\
							& 						& \multicolumn{1}{|c|}{} 			& \multicolumn{1}{|c|}{}	& \multicolumn{1}{|c|}{} \\
							& Solid					& \multicolumn{1}{|c|}{} 			& \multicolumn{1}{|c|}{}	& \multicolumn{1}{|c|}{} \\
							& 						& \multicolumn{1}{|c|}{} 			& \multicolumn{1}{|c|}{}	& \multicolumn{1}{|c|}{} \\
							& 						& \multicolumn{1}{|c|}{} 			& \multicolumn{1}{|c|}{} 	& \multicolumn{1}{|c|}{} \\\cline{3-5}
							&  						& \multicolumn{1}{|l|}{iv.} 			& \multicolumn{1}{|l|}{v.}	& \multicolumn{1}{|l|}{vi.} \\
\multirow{3}{*}{\rotatebox[origin=l]{90}{\textbf{Ice}}}	& 	 		& \multicolumn{1}{|c|}{}			& \multicolumn{1}{|c|}{} 	& \multicolumn{1}{|c|}{} \\
							& Mixed Phase				& \multicolumn{1}{|c|}{} 			& \multicolumn{1}{|c|}{}	& \multicolumn{1}{|c|}{} \\
							& \scriptsize(Solid/Liquid)		& \multicolumn{1}{|c|}{} 			& \multicolumn{1}{|c|}{}	& \multicolumn{1}{|c|}{} \\
							&  						& \multicolumn{1}{|c|}{} 			& \multicolumn{1}{|c|}{}	& \multicolumn{1}{|c|}{} \\\cline{3-5}
							&  						& \multicolumn{1}{|l|}{vii.} 			& \multicolumn{1}{|l|}{viii.}	& \multicolumn{1}{|l|}{ix.} \\
							& 	 					& \multicolumn{1}{|c|}{} 			& \multicolumn{1}{|c|}{}	& \multicolumn{1}{|c|}{} \\
							& Liquid					& \multicolumn{1}{|c|}{} 			& \multicolumn{1}{|c|}{}	& \multicolumn{1}{|c|}{} \\
							&  						& \multicolumn{1}{|c|}{} 			& \multicolumn{1}{|c|}{}	& \multicolumn{1}{|c|}{} \\
							&  						& \multicolumn{1}{|c|}{} 			& \multicolumn{1}{|c|}{}	& \multicolumn{1}{|c|}{} \\\cline{3-5}
	\end{tabular} 
\end{table}


\WCD

	\item Now that we have a class consensus on which states are possible, your instructor will assign you one of the possible final states:
	\begin{enumerate}
		\item For your possible final state, draw an \EnergyDiagram{} (with energy conservation equation, as always).
		\item Would your final state be possible if there were A LOT of ice?  What about with A LOT of tea?

\WCD

	\end{enumerate}

	\bitem{Thinking about how to proceed:  How do we determine the final state?}
	
	Put a \TempGraph{} on the board for each of the two systems: The water that starts out as ice at \unit[-65]{\textdegree C} and the tea (liquid water) that starts out at \unit[20]{\textdegree C}. Put a big dot on each graph at the initial state.
	
	\begin{enumerate}
		\item Have two separate group members place a finger on the two starting dots. Then a third group member should start explaining how energy is transferred from one physical system to the other, specifically naming the energy systems that are losing energy and that are gaining energy. Move the fingers along each graph in response to the explanation. Make sure every group member agrees with the explanation and the motion along the graphs. STOP when one substance gets to a phase change temperature.
		
		\item We need to use quantitative skills to figure out \textbf{which substance gets to its phase change temperature first}. With your group members, think about ways that you might be able to determine this. Hint: Are there two values you can compare? Write in words what you need to do to figure this out.
		\label{act123-compare-energies}
		
		\item After you have explained in words how you will figure out which substance gets to its phase change first, draw an \EnergyDiagram for each calculation you want to perform, and then perform the calculation.

\WCD 

		\item Is it possible to determine the final state of the substances yet?  Is it possible to eliminate any of the possible final states? Which one(s)?
		
		\item We now need to repeat \hyperref[act123-compare-energies]{Part~\ref*{act123-compare-energies}} to figure out which happens first: the tea decreases to \unit[0]{\textdegree C},  or the ice goes through an entire phase change. Draw {\em NEW initial points} and determine which changes in energy you need to compare. Then draw the \EnergyDiagrams, and do the calculations.
		
		\item Are the two physical systems in thermal equilibrium at the end of this step?  How do you know this?
		
		\item Can you now determine the final state of the ice and tea system? Draw an appropriate \EnergyDiagram on the board for this. If you finish early, you may solve for the final temperature or state.

\WCD

	\end{enumerate}
\end{enumerate}