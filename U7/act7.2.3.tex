\section{Check for Understanding: Momentum}
\label{act7.2.3}

\begin{overview}

	\textbf{Overview:} In this section, we'll pull everything we've discussed about momentum together to apply our understanding to a few new situations.
	
\end{overview}


\subsection{Collisions in One and Two Dimensions}

\begin{fnt}
	\label{fnt7.1.1-7}

Two asteroids collide head-on and stick together. Before the collision, asteroid A (mass 1,\unit[000]{kg}) moved at \unitfrac[100]{m}{s}, and asteroid B (mass 2,\unit[000]{kg}) moved at \unitfrac[80]{m}{s} in the opposite direction. Use momentum conservation (make a complete \pchart{}) to find the velocity of the asteroids after the collision.
\end{fnt}

\begin{fnt}
	\label{fnt7.2.1-7}

Two asteroids identical to those in \ref{fnt7.1.1-7} collide at right angles and stick together. ``Collide at right angles'' means that their initial velocities were perpendicular to each other. You can assume that Asteroid A initially moved to the right and Asteroid B initially moved up.

Use the \pModel{} (make a complete \pchart{}) to find the velocity (magnitude \emph{and} direction, expressed as the angle with the initial velocity vector of Asteroid A) of the asteroids after the collision.  
\end{fnt}

\begin{fnt}
	\label{fnt7.2.1-8}

Determine the decrease in total kinetic energy $\Delta KE_\text{total}$ of the two asteroids in \ref{fnt7.1.1-7} and \ref{fnt7.2.1-7} when they collide.  If the average specific heat of the material composing the asteroids is assumed to be that of ice (\unitfrac[2.05]{kJ}{kg$\cdot$\textdegree{}C}), by how much does the temperature of the asteroids rise as a result of the collision in each case?
\end{fnt}

\begin{enumerate}
	\item In your group, compare your individually created \pcharts{} for \ref{fnt7.1.1-7} and \ref{fnt7.2.1-7}. Note that you do not need to complete the entire chart to find the final velocity. That is, you don't need to find the change in momentum of each asteroid separately. However, finding the individual changes is a good opportunity to hone your skills with momentum conservation.
	
	\textbf{Note:} You may find it useful to choose an $x$-$y$ coordinate system and fill in the chart for \ref{fnt7.2.1-7} with $x$- and $y$-components.
	
	\item On the board, put up complete \pcharts{} (with equations worked through to numerical values) for the 1-D linear collision and the 2-D collision in these two asteroid collision FNTs.
	
	\item Compare your answers to \ref{fnt7.2.1-8}, come to a consensus, and put your response on the board.
	
	\item \textbf{Extension:} Let's consider the asteroid collision as a ``Many Body Problem:'' Three asteroids collide and stick together. Create a possible \pchart{}.
\end{enumerate}

\WCD

\subsection{Horizontally Thrown and Dropped Balls}

\begin{fnt}
	\label{fnt7.2.1-9}

Throw an object (like a ball) horizontally and observe its motion. You are now going to analyze this motion using conservation of momentum. We assume there is no air friction on the ball. Consider the motion of the ball just after it has left your hand moving in a horizontal direction.

\begin{enumerate}[(a)]
	\item Draw a force diagram for the ball. What direction is the net force?
	
	\item Complete the first two rows in the \pchart{} below. The $\vec{p}_i$ of each successive step will be the $\vec{p}_f$ from the previous step. What is the direction of $\Delta \vec{p}~$?
	\vspace{-8pt}
	\begin{center}
		\begin{tikzpicture}[thin,scale=0.9, every node/.style={transform shape},background rectangle/.style={fill=white}, show background rectangle]
			% draw table
			\draw (0,0) -- (0,3);
			\draw[very thick] (2,0) -- (2,3);
			\draw (4,0) -- (4,3);
			\draw (6,0) -- (6,3);
			\draw (8,0) -- (8,3);
			\draw[dashed] (0,0) -- (8,0);
			\draw[very thick] (0,1) -- (8,1);
			\draw[very thick] (0,2) -- (8,2);
			\draw (0,3) -- (8,3);
			
			% label table
			\node[text width=2cm, align=center] at (1,2.5)
				{Open $\vec{p}$ system};
			\node[text width=2cm, align=center] at (3,2.5)
				{$\vec{p}_i$};
			\node[text width=2cm, align=center] at (5,2.5)
				{$\Delta\vec{p}$};
			\node[text width=2cm, align=center] at (7,2.5)
				{$\vec{p}_f$};
			\node[text width=2cm, align=center] at (1,1.5)
				{Ball $\Delta t_1$};
			\node[text width=2cm, align=center] at (1,0.5)
				{Ball $\Delta t_2$};
				
			% arrows
			\draw[-Stealth] (2.5,1.5) -- (3.5,1.5);
			\draw[-Stealth] (6.5,1.75) -- (7.5,1.25);
			\draw[-Stealth] (2.5,0.75) -- (3.5,0.25);
		\end{tikzpicture}
	\end{center}
	\vspace{-12pt}
	\item Add three more rows to the \pchart{} (just show the three vectors $\vec{p}_i$, $\Delta \vec{p}$, and $\vec{p}_f$). Assume the same time interval for each change such that $\Delta \vec{p}$ will be \nicefrac{1}{5} the length of the initial momentum, $\vec{p}_i$.
	
	\item Why does $\Delta \vec{p}$ stay constant for each step?
	
	\item After you have carefully constructed a series of final momenta, use them and what you know about the relationship of the direction of momentum to the path of an object to construct a sketch of the path of the ball.
\end{enumerate}
\end{fnt}

\begin{fnt}
	\label{fnt7.2.1-10}

Repeat what you did in \ref{fnt7.2.1-9}, but this time do two separate \pcharts{}, one for the horizontal and one for the vertical components of the motion. Describe in words how the motion changes in the two directions. Compare your two sketches from part~(e) in \ref{fnt7.2.1-9} and \ref{fnt7.2.1-10}. Are the paths you constructed the same?
\end{fnt}

\begin{fnt}
	\label{fnt7.2.1-11}

\begin{enumerate}[(a)]
	\item Wrap up what you have done in \ref{fnt7.2.1-9} and \ref{fnt7.2.1-10} by explaining, in as few words as possible, why the ball moves along the path you sketched.
	\item For any object to be moving in a curved path what is necessary about the relationships between $\vec{p}_i$, $\Delta \vec{p}$, $\vec{p}_f$, and $\sum F$?
	\item If $\Delta \vec{p}$ were always perpendicular to $\vec{p}_i$ what type of path would this result in?
\end{enumerate}
\end{fnt}

\begin{enumerate}
	\item In your group, compare your individually created \pcharts{} for \ref{fnt7.2.1-9} and \ref{fnt7.2.1-10} and negotiate a consensus for each one.
	
	\item Put your consensus charts on the board. It is simplest to put one header row, with labels $\vec{p}_i$, $\Delta \vec{p}$, and $\vec{p}_f$, and then start a new row in your chart for each time step.
	
	\item Discuss and be prepared to share with the class your response to the prompt, ``Describe in words how the motion changes in the two directions.'' Compare your diagrams from Part~(e) between the two FNTs. Are the paths you constructed the same? Come to a consensus in your small group on your response to Question~(a) in \ref{fnt7.2.1-11} and be prepared to share it with the whole class.
	
	\item Come to a consensus on Parts~(b) and (c)  in  \ref{fnt7.2.1-11} and be prepared to share them with the whole class.
\end{enumerate}

\WCD