\label{fnt1.1.3-2}

When using the \EnergyInteractionModel{} to make sense of the behavior of a heat pack (or really, any thermal cycle), it helps to divide the overall process (cycle) into multiple sub-processes.  Use the following sub-processes:

\begin{center}
\begin{tabular}{llll}
	\hline\hline
	&	Initial conditions	&	Action	&	Final Conditions\\
	\hline
	(a)	&	liquid at \unit[100]{\textdegree C}	&	taken out of boiler	&	liquid at \unit[23]{\textdegree C}\\
	(b)	&	liquid at \unit[23]{\textdegree C}	&	triggered	&	solid/liquid at \unit[54]{\textdegree C}\\&& (and insulated)	& shortly after triggering\\&&& (in mixed state)\\
	(c)	&	solid/liquid at \unit[54]{\textdegree C}	&	sitting on table	&	 solid at \unit[23]{\textdegree C}\\
	(d)	&	solid at \unit[23]{\textdegree C}	&	placed in boiler	&	solid/liquid at \unit[54]{\textdegree C}\\
	(e)	&	solid/liquid at \unit[54]{\textdegree C}	&	left in boiler	&	liquid at \unit[100]{\textdegree C}\\
	\hline\hline
\end{tabular}
\end{center}

\noindent Make four \EnergyDiagrams{}, one for each sub-process (a), (c), (d), and (e), but NOT (b). Also, make a \TempGraph{} for each process. Make sure you can describe each process {\em in your own words} using both of the representations. 