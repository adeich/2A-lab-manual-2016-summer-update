\label{fnt7.3.2-1}

A physical therapy patient contracts her biceps muscle, and exerts a horizontal force of 180 N on the spring shown in the figure at right.  Assume the forearm rotates at the elbow.
a)	Draw an extended force diagram of the forearm (everything from the elbow to the tip of the fingers) showing the pivot point and all of the forces acting on the forearm where they actually are applied.   You won't know the magnitudes (or even the directions) of these forces until you finish the whole problem but just draw reasonable vectors for now.   
b)	Now draw a force diagram with the forearm as a dot.
c)	Use the fact that the forearm is perfectly stationary at all times (it is ``static'') to find the horizontal force that her biceps exerts on her forearm AND the force exerted by the bone of the upper arm (shown below the biceps) on the forearm (weight = 30N).   Note that muscles can only PULL in the direction along the muscle, but that ligaments (like the ones in the elbow that connect the upper arm with the forearm) can exert forces in any direction.
d)	Explain in complete English sentences (i.e., not just equations) why the bone of the upper arm exerts such a large force on the forearm.
e)	What is the advantage of having the point where the biceps is attached to the forearm so close to the elbow?  What is the disadvantage of having it attached so close?