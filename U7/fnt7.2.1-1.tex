\label{fnt7.2.1-1}

\begin{wrapfigure}{R}{0.25\textwidth}
	\vspace{-20pt}
  	\centering
	\begin{center}
	\begin{tikzpicture}[thick,scale=0.9, every node/.style={transform shape},background rectangle/.style={fill=white}, show background rectangle]
		% draw xy-axes
		\draw[-{Stealth[scale=1.2]}, line width=0.5pt] (0,-0.5) -- (0,5) node[left=6pt,align=center] {$y$};
		\draw[-{Stealth[scale=1.2]}, line width=0.5pt] (-0.5,0) -- (3,0) node[below=6pt,align=center] {$x$};
		
		% draw F1
		\draw[-{Stealth[scale=1.2]}, line width=1.5pt] (0,0) -- (1.15,0.67) node[above=3pt,align=center] {$\vec{F}_1$};
		% Label F1 = 5N
		\node[label={[rotate=30]above:\unit[5]{N}}] at (0.577,0.333) {};
		% draw F1 arc
		\draw[line width=0.5] (0.6,0) arc (0:30:0.6);
		% label F1 arc 30deg
		\node[label={above:30\textdegree}] at (1.25,-0.25) {};
		
		% draw F2
		\draw[-{Stealth[scale=1.2]}, line width=1.5pt] (0,0) -- (0,4) node[right=6pt,align=center] {$\vec{F}_2$};
		% label F2
		\node[label={[rotate=90]left:\unit[15]{N}}] at (-0.125,2.5) {};
		
		% draw and label v_i
		\draw[->] (1.5,3) -- (2.5,3);
		\node[label={center:$\vec{v}_i=\unitfrac[3]{m}{s}$}] at (2,2.5) {};
	\end{tikzpicture}
	\vspace{20pt}
	\end{center}
\end{wrapfigure}

\noindent\textbf{Note:} This is an extension of \ref{fnt6.1.2-6}. Two force vectors ($\vec{F}_1$ and $\vec{F}_2$, as shown to the right) act on a \unit[2]{kg} object that has an initial velocity $\vec{v}_i$ of \unitfrac[3]{m}{s} in the $+x$-direction.

\begin{enumerate}[(a)]\setcounter{enumi}{2}
	\item Use the $x$- and $y$-components of the force you found in (a) to determine the $x$- and $y$-components of impulse that would act on the \unit[2]{kg} object if the forces were applied for a time interval of \unit[0.50]{s}. Always include units with your answers. Start with the relationship of impulse and force. Find each component of impulse separately.
	
	\item Find separately, for each component, the change in velocity of the \unit[2]{kg} object, due to the impulse from (c).
	
	\item Find the magnitude of the velocity of the object after the impulse has acted and the direction the velocity vector makes with the positive $x$-axis.
\end{enumerate}