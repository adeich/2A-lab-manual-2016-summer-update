\section[Energy in Chemical Reactions]{Applying the \EnergyInteractionModel{} to Chemical Reactions}
\label{act1.1.7}

\begin{overview}

\noindent
{\bfseries Overview:} Now that we've used the \EnergyInteractionModel{} to understand a few physical phenomena, we'll apply it to understand a class of phenomena that you might already be familiar with from chemistry: Chemical reactions.

\end{overview}

\note{Purpose}{
	Students think about what bond energy means in the context of chemical reactions, instead of phase changes.
}

\begin{fnt}
	\label{fnt1.1.4-1}

Every general model is applicable to some range of physical phenomena. The \EnergyInteractionModel{} can be used to analyze $v_i$rtually {\em any} process, but up to this point we have been using it to make sense of the phenomena of {\em temperature changes} and {\em phase changes}. The former are modeled as changes in thermal energy systems, with the indicator being temperature, and the latter are modeled as changes in bond energy systems, with the indicator being the mass of substance that is changing phase. But {\em bond energy} changes in chemical reactions just as it does in phase changes, so we use exactly the same approach when modeling chemical reactions.

Consider the following chemical reaction, the hydration of calcium sulfate (Plaster of Paris). Recall that when you mix water with the white power, the paste not only gets hard, but it also gets hot:
\begin{center}
\ce{(Ca2SO4)2 * \ce{H2O} + 3\ce{H2O} -> 2Ca2SO4 * 2\ce{H2O} + Heat_{to~env.}}
\end{center}
Has the bond energy of this \textbf{\em total} system increased, decreased, or stayed the same during this process (mixing water with the powder)?  Use the \EnergyInteractionModel{} to explain. The simplest way to answer this question is to model all of these different chemicals as being a {\em single} physical system with {\em one} thermal-energy system and {\em one} bond-energy system. (Note that the question asked about the {\em total bond energy}, not bond energies of the separate molecular species. If the question had asked about bond energy changes of particular molecular species, we would have to include separate bond energies in our model.)



\end{fnt}

\noindent The chemical reaction that occurs when you mix together powdered Plaster of Paris and water resulting in a paste that can be used for molds and casts as it solidifies is given by:			
\begin{center}
	\ce{(Ca2SO4)2 * \ce{\ce{\ce{H2O}}} + 3\ce{\ce{\ce{H2O}}} -> 2 (Ca2SO4 * 2\ce{\ce{\ce{H2O}}}) + $Q$_{to~env.}}
\end{center}
Represent this process in an \EnergyDiagram{} by modeling {\em all} of the reactants and products {\em together} as having a {\em single} bond energy system and a single {\em thermal} energy system. \textbf{Assume the initial and final temperatures are the same} and that heat ($Q$) has been transferred to the environment during the process, which is obvious if you have ever mixed Plaster of Paris. This should make answering the question asked in \hyperref[fnt1.1.4-1]{the FNT} (``Did the bond energy increase, decrease, or stay the same?'') very straightforward.

\note{}{
	Since heat was produced, the total bond energy must have decreased.
}

\WCD 

\begin{fnt}
	\label{fnt1.1.4-2}

\begin{enumerate}[(a)]
	\item Think about the general case of chemical reactions. When a single compound breaks up into separated atoms, what can you say with absolute certainty regarding the \textbf{\em change} in bond energy of that compound? Why? What can you say with absolute certainty regarding the change in bond energy of a compound that is formed from separated constituent atoms? Why?
	
	\item Consider this chemical reaction: the combustion of propane.
	\begin{center}
		\ce{C3 H8 + 5 O2  ->  3 CO2 + 4 \ce{H2O}}
	\end{center}
	We can model this process as if the reactant molecules are broken down into their constituent atoms, which are then re-assembled into the product compounds.

	\begin{enumerate}[i.]
		\item Represent this process with an \EnergyDiagram{} that contains four separate bond energy systems, one for each molecular species. There will be one for the \ce{C3 H8}, one for the \ce{5 O2}, etc.
	
		\item Using your result from part (a), show the direction of each bond energy change with an arrow in the standard way. The indicator for each compound is the number of moles of that compound.
		
		\item Show the initial and final values of each of the indicators in the standard way.
	\end{enumerate}
	
	\item The magnitude of the bond energy changes for all the molecules involved in this process when they are separated into atoms are:

	\begin{center}
	\ce{C3 H8}: \unitfrac[4002]{kJ}{mol}; \;
	\ce{O2}: \unitfrac[495]{kJ}{mol}; \; 
	\ce{CO2}: \unitfrac[1607]{kJ}{mol}; \;
	\ce{H2O}: \unitfrac[925]{kJ}{mol}.
	\end{center}
	
	\begin{enumerate}[i.]
		\item Determine the bond energy changes, the $\Delta E_\text{bond}$, for each of the reactants and products.
		\item Write a conservation of energy equation in the standard way  ($\Delta E_1 + \Delta E_2 + \Delta E_3 + \Delta E_4 = Q$). Write this equation out with appropriate subscripts to clearly identify each term.
		\item Rewrite it with the numerical values being careful to get the algebraic sign correct.
		\item Calculate $Q$, which will be the heat released from the combustion of one mole of propane.
		\item Is the algebraic sign of $Q$ that you calculate consistent with our sign convention for $Q$?
	\end{enumerate}
\end{enumerate}
\end{fnt}

\noindent
We'll now use the \EnergyInteractionModel{} to get a close approximation of the heat of combustion of a mole of propane in a chemical reaction. This is an example of how our general approach to energy conservation using the \textbf{\EnergyInteractionModel{}} encompasses topics that traditionally seem to be something entirely different; e.g., the heat of combustion of various chemical substances.

\begin{enumerate}
	\item Discuss briefly in your group your responses to question (a) regarding what you can say with certainty about changes in bond energy. Come to a consensus and \textbf{put your response on the board}.
	
	\note{}{
		When a compound breaks up into constituent atoms, its bond energy increases. 
	
		When a compound is formed, its bond energy decreases.
	}
	
	\item Share your responses to part (b) in your group and put an \EnergyDiagram{} up on the board as directed in part (b) of \hyperref[fnt1.1.4-1]{the FNT}. Make sure everyone in your group understands why the values of the indicators are the way they are. Make sure the difference in the values of the indicators between reactants and products makes sense to everyone in your group and why the changes are in the direction shown on your diagram.
	
	\note{}{
		There should be four bond energy systems:
			\begin{itemize}
				\item \ce{C3H8} bond energy increases as bonds break.
				\item \ce{O2} bond energy increases as bonds break.
				\item \ce{CO2} bond energy decreases as bonds form. 
				\item \ce{\ce{\ce{\ce{H2O}}}} bond energy decreases as bonds form.
			\end{itemize}
	}
	
	\item Write your equation expressing energy conservation with the appropriate numerical values to show how to find $Q$.
	
	\note{}{
		\begin{align*}
		\Delta E_{\text{bond} (\ce{C3H8})} + \Delta E_{\text{bond} (\ce{O2})} + \Delta E_{\text{bond} (\ce{CO2})} +\Delta E_{\text{bond} (\ce{\ce{\ce{\ce{H2O}}}})} \&= Q\\
		\unitfrac[4002]{kJ}{mol} + 5(\unitfrac[495]{kJ}{mol}) - 3(\unitfrac[1607]{kJ}{mol}) - 4(\unitfrac[92]{kJ}{mol}) \&= -\unit[2044]{kJ}
		\end{align*}
	}
\end{enumerate}

\note{Note for Instructor}{
	In their introductory chemistry course students will have been introduced to various ``Heats'' expressed as $\Delta H$'s:  Heat of formation, heat of fusion, heat of vaporization, etc. In this FNT, values of the various $\Delta E_\text{bond}$'s were calculated by looking up in a chemistry text the heats of formation of the different molecules. The algebraic signs can be confusing, but thinking about it in terms of how a particular $\Delta E$ changes, you canr $KE$ep it straight. The other confusing thing about using heats of formation, is that the ``zero'' is typically taken to be the state of the element at STP. So, for example, the heat of formation of \ce{O2} is typically listed as zero, but the atomic \ce{O} does have a listed value. Also note that all values of heats of formation, other heats, and heat capacities are listed in chemistry books or tables typically shown as $\Delta H$'s. This is because enthalpy, $H$, is the appropriate energy to use when processes are carried out at constant pressure. This all becomes clear at the end of the quarter when we do some simple thermodynamics.
	\todo[inline]{The last sentence of this paragraph is false and may require updating.}
}

\WCD  
\note{}{
	Aside from summing up and clarifying, a point that is good for you to emphasize is that this is what they were doing in their chemistry class with heats of formation. This is all the same science!  Whether we call it chemistry or physics, it is the same fundamental thing!  Many students really think that physics energy is somehow different from chemistry energy.
}