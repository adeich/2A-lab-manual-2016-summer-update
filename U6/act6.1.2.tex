\section{Working with Position and Displacement Vectors}
\label{act6.1.2}

\begin{overview}

\textbf{Overview:} Now that we know the basic properties of vectors, let's talk about two similar, yet very different kinds of vectors: \textbf{\emph{position}} and \textbf{\emph{displacement vectors}}.

\end{overview}

\noindent\textbf{Phenomenon:} You are in a strange city that has streets that are laid out in a perfectly square grid. Your job will be to move around to different locations as instructed and record your progress.\\

\noindent\textbf{Make sure everyone in your group fully understands the ideas behind each question or part in these activities before going on to the next part.}

\begin{enumerate}
	\item Make a drawing on the board showing the streets in the central city (this should be a square grid with at least 10 streets going in each direction). Decide as a group how you are going to label the streets and use your labeling system. Make sure your diagram is large enough for everyone in the room to clearly see it.

	\item Near one of the edges of the diagram, label a street corner ``o'' for origin of your coordinate system. Your origin should be different from the origins in other groups' drawings. Sketch a coordinate system so that the $y$-axis points North and the $x$-axis points East.
	
	You start walking somewhere in the city, so choose an intersection near the lower middle part of your diagram, and call that location your ``initial'' position. Draw a position vector on your diagram that shows this initial position. Label this vector ``$\vec{R}_i$.''
	
	\begin{enumerate}
		\item Write $\vec{R}_i$, in terms of its $x$- and $y$-components, as $\left(R_{i,x}, R_{i,y}\right)$. Determine the length of $\vec{R}_i$. What units does it have? How would you describe the direction of the position vector?

\WCD
\vspace{12pt}

		\item Starting from your initial location, imagine walking a distance equal to 4~blocks North and then 1~block West and then 1~block South to a ``final'' location. Show the \emph{position vector}, $\vec{R}_f$, for this new location and write $\vec{R}_f$ in terms of its components as $\left(R_{f,x}, R_{f,y}\right)$.
		
		\item As you might expect, we define the ``change in position'' to be $\Delta \vec{R} = \vec{R}_f - \vec{R}_i$, so that $\vec{R}_i + \Delta \vec{R} = \vec{R}_f$.
		
		Using the tail-to-head method shown on Page~38 of the course notes, redraw the vectors $\vec{R}_i$ and $\vec{R}_f$ off to the side of your map. Then show how you can \textbf{graphically obtain} $\Delta \vec{R}$ from $\vec{R}_i$ and $\vec{R}_f$. Make sure you draw these vectors with the same lengths and directions that they have on your map.
		
		Using the same picture that you used to show $\Delta \vec{R} = \vec{R}_f - \vec{R}_i$, show that $\vec{R}_i + \Delta \vec{R} = \vec{R}_f$ is also true. Now transfer your $\Delta \vec{R}$ over to your street diagram.
		
		\item In physics, we are interested in describing motion. If you could choose only one vector from $\vec{R}_i$, $\vec{R}_f$, and $\Delta \vec{R}$ to describe your motion, which one would it be? Why? What do either of the other two vectors, by themselves, tell you about your motion?
		
		\item How are the $x$-components of $\vec{R}_i$ and $\vec{R}_f$ related to the $x$-component of $\Delta \vec{R}$? How about the $y$-components?

\WCD
\vspace{12pt}

		\item Suppose you walked the 4~blocks North and 1~block West and 1~block South in 60~seconds. Draw the \emph{average velocity} vector for this situation. Write this average velocity vector in component form, $\left(v_x , v_y\right)$. What is the ``length'' of this vector? Include the units. Draw another velocity vector for a situation where you took 180~seconds for this 6~block walk. Which arrow is longer? Why?
		
		\item In which direction(s) would your \emph{instantaneous velocity} be?
		
		\item If you now walked back to your initial point and assuming the total journey took eight hours what is your average velocity for the entire trip?
	\end{enumerate}

\WCD
\vspace{12pt}

	\item Look around the room at the different street diagrams. What is similar in each one? What is different? Discuss in your group how you would summarize the meaning of everything you did in this activity. Be prepared to share with the whole class.
\end{enumerate}
 

