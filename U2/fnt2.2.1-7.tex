\label{fnt2.2.1-7}

A \unit[0.4]{kg} mass is attached to a spring that can compress as well as stretch (spring constant \unitfrac[50]{N}{m}). The mass and spring are resting on a horizontal tabletop. The mass is pulled, stretching the spring \unit[48]{cm}. When it is released, the system begins to oscillate.

\begin{enumerate}[(a)]
	\item\label{fnt2.2.1-7a} Assuming the transfer of energy to thermal energy systems is negligible, construct a complete \EnergyDiagram{} that could be used to predict the speed of the mass as it passes a point that is a distance of \unit[39]{cm} from its equilibrium point on the other side of the equilibrium position (spring is compressed).
	
		Substitute all known values of constants and variables into the algebraic expression of energy conservation, and identify any unknown(s). Do you have enough information to find the speed of the mass?
	
	\item\label{fnt2.2.1-7b} Now assume that the effects of friction are not negligible. When pulled back and released as before, the mass now reaches its furthest distance from equilibrium at \unit[40]{cm} on the compressed side (before bouncing back again). Construct a complete \EnergyDiagram{} that could be used to determine the amount of energy transferred to thermal systems when going from the initial stretched position to where it first momentarily stops.
	
		\textbf{Proceed as in part a:} Substitute all known values and identify any unknown(s). Can you determine the increase in thermal energy?
\end{enumerate}