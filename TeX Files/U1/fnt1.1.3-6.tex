\label{fnt1.1.3-6}

Imagine that you place a piece of copper with an initial temperature of \unit[20]{\textdegree C} in contact with an amount of liquid water with an initial temperature of \unit[100]{\textdegree C}. Assume that the physical system consisting of the copper and the water is thermally isolated from everything else; i.e., water and copper can \emph{only} exchange energy \emph{with each other}.

\begin{enumerate}[(a)]
	\item Using the \ThreePhaseModel{} as applied to each substance and your understanding of what ``coming to thermal equilibrium'' means:
	
	\begin{enumerate}[i.]
		\item Sketch two \TempGraphs{} -- one for each substance -- and indicate the initial state for each one.
		\item Explain in a sentence or two how you can tell if either substance will undergo a phase change during the process of coming to thermal equilibrium. You are expected to use what you already know regarding the thermal properties of copper and water, but you do not need to do any calculations.
	\end{enumerate}
	
	\item Construct a complete \EnergyDiagram{} for the process that ends when the two substances are in thermal equilibrium. Don't forget that a complete diagram always includes an algebraic expression of energy conservation.
	
	\item Now consider a similar process. Use the same initial conditions for the copper, but assume that the \ce{H2O} is initially in the gas phase at \unit[100]{\textdegree C}.
	\begin{enumerate}[i.]
		\item Sketch two \TempGraphs{}, one for each substance, and mark the initial state for each one.
		\item How can you determine if the \ce{H2O} will undergo a temperature change? (What calculations and comparisons -- think energy -- would you need to make?  Don't actually do the calculations; just explain what you would need to do.)
	\end{enumerate}
\end{enumerate}