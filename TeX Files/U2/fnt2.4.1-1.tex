\label{fnt2.4.1-1}
  
2.4.1-1) Consider a ball thrown upward from the surface of the Earth, and the same ball thrown upward from the surface of the moon with the same initial speed. (What does that imply about their maximum PE?) Construct a complete \EnergyDiagram{} for each of these situations. Then carefully plot a graph (on graph paper) of $PE_\text{gravity}$ vs. y (height) from the point of release to the maximum height for each ball. (Put both on the same graph.) Set the origin (where y = 0) at the height the ball is released. Assume that gmoon ≅ (1/6)gearth,  and that the magnitude of the force of the moon on the ball is approximately 1/6 the magnitude of the force of the earth on the ball. 
How do the change in potential energy and the maximum height compare for the two situations? Make sure this is reflected in your graph. 