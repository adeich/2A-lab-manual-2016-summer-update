\section{Graphically Representing Energy Relationships}
\label{act2.3.1}

\begin{overview}

\textbf{Overview:} By now, you are very familiar with an \emph{algebraic} representation of the \textbf{energy conservation principle}. In this activity, we are restating this algebraic representation and introducing a \emph{graphical} representation of this principle as a useful tool for understanding certain physical phenomena, for example a \textbf{falling ball}.

\end{overview}

\subsection{Rethinking and Restating Energy Conservation}

\noindent\textbf{Situation 1:} A dropped ball at any time between when it is dropped from rest to {\em just} before it hits ground.\\

\noindent Our standard expression of energy conservation for this situation is $\Delta PE_\text{gravity} + \Delta KE = 0$. However, there are times when it is more useful to have a different expression of energy conservation (e.g., when graphically representing energy amounts). Your instructor will use the definition of the \emph{difference operator} ``$\Delta$'' to algebraically show how the expression:

\begin{displaymath}
	E_\text{total}\text{(at any time)} = \text{const.} = PE_\text{gravity} + KE
\end{displaymath}

\noindent is equivalent to:

\begin{displaymath}
	\Delta PE_\text{gravity} + \Delta KE = 0.
\end{displaymath}
	
\subsection{Using Energy Conservation to Graphically Represent Energy Amounts for the Falling Ball}
	
In order to plot the different amounts of energy ($PE_\text{gravity}$, $KE$, and $E_\text{total}$) of the dropped ball (Situation~1 above) as a function of height, it is necessary to decide where to set $y = 0$. This means, we need to decide where in space to put the origin of the $y$-coordinate, which measures the height of the ball above the surface of the earth.

For example, you could choose $y = 0$ to be at the floor level, or at \unit[3]{m} below the floor. Since we use $PE_\text{gravity} = mgy$ as our standard expression for gravitational potential energy, the place where $PE_\text{gravity} = 0$ depends on where we put $y = 0$. For any given physical situation you can set $y = 0$ {\em wherever you want}.\\
	
\noindent\textbf{Your instructor will demonstrate} how to create the graphical representation of energy amounts with $y = 0$ set at \unit[3]{m} {\em below} the floor. Then, each small group will construct an energy graph for a dropped ball following the directions given below for where to set the origin of the $y$-coordinate system.\\

\label{act2.3.1-2} \noindent\textbf{In your group:} Assume the ball is being dropped from a height of 2~meters above the floor (\textbf{Note:} This physical situation is exactly the same for each group). Graph the three energy amounts in Situation~1 using the $y = 0$ location listed here:

\begin{center}
\begin{tabular}{clccl}
	\hline\hline
	Group	&	$y = 0$ at:	& &	Group	&	$y = 0$ at:\\
	\hline
	1	&	\unit[2]{m} below the floor	& &	4	&	\unit[2]{m} above the floor\\
	2	&	\unit[1]{m} below the floor	& &	5	&	\unit[3]{m} above the floor\\
	3	&	\unit[1]{m} above the floor	& &	6	&	\unit[4]{m} above the floor\\
	\hline\hline
\end{tabular}
\end{center}

\begin{enumerate}[(a)]
	\item	\label{act2.3.1-2a} First, create a small, simple sketch of the physical scenario, indicating where $y = 0$, the ball's initial position, and the floor. Label each of these places with its respective $y$-value and a physical description.
	\item	\label{act2.3.1-2b} Plot $PE_\text{gravity}$, $E_\text{total}$, and $KE$ of the ball (all on the same graph) as functions of height using coordinate axes as described above. Why does it make sense to graph $PE_\text{gravity}$ {\em first}, $E_\text{total}$ {\em second}, and only then $KE$? Be ready to explain how you determined $E_\text{total}$.\footnote{Remember: It is standard convention to plot the \emph{dependent} variable on the vertical axis and the \emph{independent} variable on the horizontal axis. Consequently, the energy is plotted on the vertical axis.} Be prepared to explain how you constructed your graph.
	\item	\label{act2.3.1-2c} Choose an arbitrary value of $y$ on your graph. For this value, indicate on your graph how $E_\text{total} = PE_\text{gravity} + KE$. Then, choose two different values of $y$ on your graph. For those values, indicate on your graph how $\Delta PE_\text{gravity} + \Delta KE = 0$.
\end{enumerate}

\WCD


\subsection{Using Energy Conservation to Graphically Represent Energy Amounts for a Mass-Spring System}
\label{SpringMassActivity}

\textbf{Situation 2:} A mass hanging on a spring is pulled down and released. In your plot, consider all times between when the mass-spring is released and the first time it is momentarily at rest.\\
	
\noindent Repeat \eqref{act2.3.1-2a}, \eqref{act2.3.1-2b}, and \eqref{act2.3.1-2c} from \hyperref[act2.3.1-2]{Part~\ref*{act2.3.1-2}} using Situation~2. The three energy amounts are now plotted as functions of ``distance from equilibrium.'' In this situation, \textbf{everyone} should have $y = 0$ at the same place: The \emph{equilibrium position} of the mass-spring system.

\WCD