\label{fnt8.2.1-6}

Refer to the graphs on the reverse which were made using a motion detector mounted above a basketball that was dropped from a height of \unit[1.5]{m} above the floor. The position measured and indicated is really the position of the top surface of the ball. The velocity graph was computed by the software as the derivative of the position vs.\ time graph. Complete the following tasks related to this situation.

\begin{enumerate}[(a)]
	\item Determine accurately the acceleration from the velocity graph and plot it on the acceleration axis. Make sure you extend the acceleration curve as far as the other two are extended in time.
	
	\item Describe the motion of the ball at the six indicated times numbered [1] to [6]. That is, describe where it is located and say something about its speed, direction of motion, and acceleration. If you look closely, you should see that at position [3] the ball is still in contact with the floor.
	
	\item Draw \forcediags{} for each of the six marked times. For which times are the \forcediags{} identical?
	
	\item For the times when the \forcediags{} are identical, which aspects of the motion are identical? Which aspects are different? Are your answers to the previous two questions consistent with Newton's 2nd law?
	
	\item Determine the average value of the force exerted by the floor on the ball between the times numbered [2] and [3] two different ways:
	\begin{enumerate}[(i)]
		\item from the impulse imparted to the ball from the floor and the velocity graph; and 
		\item using Newton's 2nd law.
	\end{enumerate}
	Explain two ways (once for each approach) why the force of the floor on the basketball when it bounces is so much greater than the basketball's weight.
\end{enumerate}