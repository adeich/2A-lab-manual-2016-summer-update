\section*{\DLM~\ref{dlm7a} Overview}

\subsection*{Brief Overview}

\subsection*{\CLASP{} Activities}
\hyperref[act2.2.2]{Activity \ref{act2.2.2}}:	Follow-up of \hyperref[fnt2.2.1-1]{\FNTs~\thechapter-\ref{fnt2.2.1-1}} through \hyperref[fnt2.2.1-8]{\FNTs~\thechapter-\ref{fnt2.2.1-8}}	(\about\unit[70]{min})

\textbf{Purpose:}
\begin{itemize}
\item Practice relying on the logic of the model to think your way through to understanding and making sense of very difficult/tricky physical phenomena
\item Practice applying the Energy-Interaction Model quantitatively to a variety of mostly mechanical phenomena.
\end{itemize}
\textbf{Learning Outcomes:}
\begin{itemize}
\item Becoming convinced that you can indeed rely on the logic of the model to think your way through to understanding and making sense of very difficult/tricky physical phenomena
\item Becoming confident that you can rapidly apply the Energy-Interaction Model to a variety of phenomena similar to those in the set of \FNTs~\ref{fnt2.2.1-1}-\ref{fnt2.2.1-8}
\end{itemize}
\textbf{Catch-up}		(\about\unit[70]{min})
\begin{itemize}
\item Finish \hyperref[act2.1.3]{Activity~\ref{act2.1.3}} part C OR \hyperref[act2.1.4]{Activity~\ref{act2.1.4}} part C OR whatever you didn�t get to last lab
\item Review practice Quiz from last week included in this packet with solution
\item Practice problem	
\end{itemize}

\subsubsection*{Reminder}

If there is a single most important task for the \DL{} instructor working in the classroom, it is to get the students up out of their chairs and talking together about whatever activity they are working on. There are prompts written into the activities instructing them to do that (e.g., All members of your group must now go to the board and work on the following together), but most of the time they need additional direction from the \DL{} instructor. You should even discourage them from putting up separate parts simultaneously, instead telling them to all work on each part, discussing and making sure everyone in the group understands the response that is being put on the board.  



\subsubsection*{Caution}
Watch out for the order in today�s activities. \hyperref[act2.2.3]{Activity~\ref{act2.2.3}} is last, after \hyperref[act2.3.1]{Activity~\ref{act2.3.1}}. This is to insure that students finish \hyperref[act2.3.1]{Activity~\ref{act2.3.1}}, which they must do in order to be able to do the \FNTs.


\subsubsection*{Equipment needed in the room:}
\begin{itemize}
\item Plenty of graph paper. 
\item Hanging mass-spring at each table. 
\item Mounted pulleys with string and masses. 
\end{itemize}