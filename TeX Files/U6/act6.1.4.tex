\section[Representing the Circular Motion of a Mass]{Using Vectors to Represent the Motion of a Mass Moving in a Circle}
\label{act6.1.4}

\note{For Activity~\ref{act6.1.4} (\about\unit[60]{min})}{
\subsection*{Learning Goals:}
\begin{itemize}
\item Practice carefully observing motion and representing this motion using vectors.  
\item Practice using perpendicular components to represent a vector.
\item Understand that a vector does not depend on the perpendicular axes chosen to define the components. 
\item Practice finding components using the graphical method. 
\end{itemize}
}

\begin{overview}

\textbf{Overview:} So far, we've only talked about linear motion -- an object moving along a straight path. However, many phenomena have objects moving along curved paths. As an example, we'll take a closer look at an object moving along a circle.

\end{overview}

\noindent\textbf{Phenomenon:} You are going to swing a ball in a circle and represent its velocity in several different ways using vectors.\\

\noindent\textbf{Make sure everyone in your group fully understands the ideas behind each question or part in these activities before going on to the next part.}


\begin{enumerate}
	\item \begin{enumerate} \label{614,1}
		\item Get the ball swinging in a \emph{large} horizontal \emph{circle}, going \emph{clockwise} when viewed from above:
	\begin{center}
		\begin{tikzpicture}[decoration={markings,mark=at position 0.5*\pgfdecoratedpathlength-25pt with {\arrow[thick]{<}},mark=at position 0.5*\pgfdecoratedpathlength+5.5cm+5pt with {\arrow[thick]{<}}}]
			\node[inner sep=0pt] (fingers) at (-.5,3) {\includegraphics[width=1cm]{pinchedFingers.png}};
			\draw[postaction={decorate},dashed] (0,0) ellipse (3cm and .8cm);
			\draw (-.18,2.82) -- (1.95,.65);
			\draw[fill=gray] (2,.6) circle (.1);
		\end{tikzpicture}
	\end{center}	
		Imagine looking down on the ball. Pick a position in space that you will identify as the ``12~o'clock'' position. Draw the circular path of the ball on the board with the 12~o'clock position toward the top of the board.
		
		Choose a coordinate system and sketch it on the board.
		
		\item Draw a position vector identifying the position of the ball when it is in the 4~o'clock position. Show the $x$- and $y$-components of this vector on your diagram. Discuss in your group how to do this. Be prepared to share with the whole class.
		
		\item Draw another position vector when the ball is in the 5~o'clock position. To the side of your diagram, graphically subtract the position vectors \textbf{as accurately as possible} to find the displacement vector; label it $\Delta \vec{r}$.
		
		\textbf{\emph{Describe in a sentence what the delta means here.}}
	\end{enumerate}
	\note{\ref{614,1}}{Make sure they do the subtraction carefully so their displacement vector is in a reasonable direction.}

\vspace{6pt}
\hspace{-\textwidth}\hspace{\linewidth} \textbf{Brief}
\hspace{\textwidth}\hspace{-\linewidth}
\WCD
	
	\item \begin{enumerate}\label{614,2} 
		\item \label{614,2a} Describe in words the \textbf{velocity} of the ball as it moves in its circle. Be prepared to share.
	\note{\ref{614,2a}: Important:}{    Students will naturally not put sufficient effort and energy into a prompt like this.  They really need to be FORCED to do this in detail.  They are not used to looking at a phenomenon and describing it carefully.  }
		
		\item  \label{614,2b}What do you think is an \emph{instantaneous velocity vector}?  How might it be different from a position vector?  Does it always point in the same direction? If you are having trouble answering these questions, apply them to a specific situation (e.g. driving northwest on I-5 at \unit[65]{mph} and then curving toward the north).
	\note{\ref{614,2b}:}{   Many will mistakenly describe and draw position vectors as if they were synonymous with velocity vectors.  This will take work on your part to get them to adequately explain to you how they are different. }
	\end{enumerate}
	
	\item \begin{enumerate}
		\item  \label{614,3a} Imagine that you are sitting on the ball and ``driving'' it in a circle. Which way are you moving when you are at the 4~o'clock position?  Draw a velocity vector (put the tail of the vector at the ball's location) showing the velocity of the ball at the 4~o'clock position. Show the $x$- and $y$-components of this velocity vector on your diagram. Discuss in your group how you do this.
	\note{\ref{614,3a}:}{ Students should realize that velocity is something that describes our lives all the time.  It has definite meaning in the physical world as well as on these pictures we draw.    }		
		\item   \label{614,3b}Looking at the vectors you have drawn, is a \emph{position} vector or a \emph{displacement} vector more closely related to a velocity vector? Make sure this agrees with your knowledge of the definition of velocity.
	\note{\ref{614,3b}:}{ Students have an almost automatic response to draw the velocity vector pointing toward the center of the circle. This is why it is important for them to take \ref{614,2} seriously.     }		
		
		\item  \label{614,3c}Which vectors on your diagram would be different if you changed the origin?
	\note{\ref{614,3c}:}{ Make sure their connection between displacement and velocity goes beyond the memorized fact that $v=\Delta r / \Delta t$  }		
	\end{enumerate}
	
\WCD
	
	\item Draw the velocity vector for the ball at the 6~o'clock position. Off to the side, graphically determine the change in velocity, $\Delta \vec{v}$, between the 4~o'clock position and the 6~o'clock position.\label{614,4}
\note{\ref{614,4}}{Again, neatness is important. It is hard to imagine how hard this is for some students.  Be patient with them and encourage them to make sense of it themselves, but in the end, make sure they all see it.}
	
	\item \label{614,5}\begin{enumerate}
		\item Find the $x$- and $y$-components of the 6~o'clock velocity vector.
		
		\item Find the $x$- and $y$-components of the vector $\Delta \vec{v}$.
		
	\end{enumerate}
	\note{\ref{614,5}:}{This is to be done graphically, basically simply ``dropping'' a perpendicular, etc.}
	
	\item \label{614,6}How can you use the $x$- and $y$-components of the 4~o'clock and the 6~o'clock positions to get the change in velocity, $\Delta \vec{v}$~?  Summarize your method and be prepared to share it with the class.
\note{\ref{614,6}:}{The goal here is to have students see that they can add the x components together to get a total $\Delta x$ component, add the y components together, etc (of course, the delta means the difference, so they should add the $�v_{4}$ to the $v_{6}$, etc.).}

\end{enumerate}

\WCD