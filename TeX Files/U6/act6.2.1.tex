\section[Force Model: Net Force and Force Diagrams]{Force Model: Net Force and Force Diagrams\protect{\footnote{See Pages~41-51 of course notes and Force Model Summary.}}}
\label{act6.2.1}

\begin{overview}

\textbf{Overview:} Now that we can represent linear and curved motion with vectors, let's see how vectors are tremendously useful to describe \emph{forces}, as well.

\end{overview}

\noindent\textbf{Phenomenon:} You are going to pull on a metal ring with three ropes attached. A spring is attached to each rope to determine the tension in the rope.

\begin{itemize}
	\item Before you start, make sure each scale is set to zero by twisting the knob.
	\item Work together as a group. No, really: This takes multiple hands!
	\item Attach the ropes with the scales turned so you can read force values in Newtons.
\end{itemize}

\noindent\textbf{Note:} To get the angles to be 130\textdegree{}, trace the ring on a sheet of paper and draw a vertical line pointing down from it. Use a protractor to mark the 130\textdegree{} angles from the vertical line. Line up the ropes along the lines you drew, pull and then note the force scale readings.

\begin{figure}[h!]
	\vspace{-4pt}
	\centering
	\begin{subfigure}[b]{0.45\textwidth}
		\centering
		\includegraphics[width=0.25\textwidth]{act621-scenario1}
		\begin{tikzpicture}[thick,scale=0.65, every node/.style={transform shape},background rectangle/.style={fill=white}, show background rectangle]
			% draw object node
			\draw (0,3.5) node[circle,minimum size=12pt,fill,inner sep=1pt]{} node[right=6pt]{Ring};	
			
			% draw and label F1
			\draw[-{Stealth[scale=1.2]}, line width=1pt] (0,3.5) -- (0,5.25) node[right=3pt,align=center] {$\vec{F}_\text{Scale 1 on Ring}$};
			% draw and label F2
			\draw[-{Stealth[scale=1.2]}, line width=1pt] (0,5.25) -- (0,7) node[right=3pt,align=center] {$\vec{F}_\text{Scale 2 on Ring}$};
			
			% draw and label F3
			\draw[-{Stealth[scale=1.2]}, line width=1pt] (0,3.5) -- (0,0) node[right=3pt,align=center] {$\vec{F}_\text{Scale 3 on Ring}$};
		\end{tikzpicture}
		\caption*{\textbf{Scenario 1:} Person~1 and Person~2 pull directly opposite to Person~3 so that the yellow spring scale for the third person reads \unit[30]{N}. Record the values on all three scales.}
	\end{subfigure}
	\hspace{0.05\textwidth}
	\begin{subfigure}[b]{0.45\textwidth}
		\centering
		\includegraphics[width=0.5\textwidth]{act621-scenario2}
		\caption*{\textbf{Scenario 2:}  Persons~1 and 2 change the direction they are pulling until their ropes make an angle of about 130\textdegree{} with respect to Person~3. Record the readings of all scales.}
	\end{subfigure}
\end{figure}

\begin{enumerate}
	\item Use appropriately scaled force vectors from the sheet of paper you've used as a guide, and graphically add the three vectors while preserving their angles on the board. What do you get? Why is that? Draw and label all vector components on your graph.
	
	\item A properly labeled and scaled force diagram for the ring is shown for Scenario 1 (on the left, above). \textbf{Note:} Two forces acting on the same object and going in the same direction can be drawn as shown here, or both arrows can have their tails attached to the dot.
	
	Draw a properly labeled and scaled force diagram for the ring: model it as an object with the three forces acting on it, as in Scenario 2. Be sure to use the labeling conventions spelled out on Page~43 of the textbook on any force diagram you make.
	
	\item On the board, off to the side of each force diagram, repeat the method of vector addition from (1). What is the net force on the ring in each case? Check: Why does the resultant net force make sense?

\vspace{12pt}
\hspace{-\textwidth}\hspace{\linewidth} \textbf{Brief}
\hspace{\textwidth}\hspace{-\linewidth}
\WCD
	
	\item Use trigonometry (sine and cosine, as appropriate) to find the magnitude of the components ($x$ and $y$) of the two forces, $\vec{F}_\text{Scale 1 on Ring}$ and $\vec{F}_\text{Scale 2 on Ring}$. Put responses to the following on the board:
	\begin{enumerate}
		\item What is the relationship between the components of these two forces and the components of the third force?
		
		\item Develop an explanation for this relationship in your small group and be ready to share.
		
		\item Explain why the scale readings changed when the two people pulling parallel separated to form 130\textdegree{} angles with the third force scale.
	\end{enumerate}
\end{enumerate}

\WCD
\vspace{12pt}

\noindent \textbf{Follow-up Question:} Could two people pull their ropes all the way to 180\textdegree{} apart with the third person still pulling at \unit[30]{N}? Include a force diagram in your response. \textbf{Hint:} Think about the components.
 
\vspace{12pt}
\WCD
