\appendixchapter{Activity Book Emails}

\section*{Miscellaneous}

\todo[inline]{Update all models to remove calculus}



\section*{Unit 2}

%\todo[inline]{FNT 2.2.3-1 (9?) mis marked in ACT 2.3.2}

%\todo[inline]{Hi Peter,
%
%I just checked, and 2.2.3-9 should be 2.2.3-1. So yes, it is a typo. However, since this is basically in regard to atwood machine, and we inserted the extra lab, it may not be necessary to review in class anyway if you've already covered it.
%
%Cassandra}

%\todo[inline]{Hi Cassandra,
%
%Activity 2.3.2 refers at the end to FNT 2.2.3-9 which I could not find in my hard copies or in any of the assignments posted don line.  Is that a typo?  Or am I missing something?
%
%-Peter}

\todo[inline]{Hi Annie,
I like your suggestion better, I think. I was trying to split up 2.3.1 and 2.2.3 because I think they are both very challenging and because the students haven't had a chance to think about them at home, putting them both together would be a particularly taxing DLM. (I realize they were together in the Davis model, but we never completed 2.2.3 in class, they always had to complete it for homework, and I was trying to avoid that.)

That said, I like that your model moves the catch-up to the beginning and splits up the homework more nicely. 

As long as you don't think the issue I've raised is a problem, let's go with what you suggested. As lead TA I give you the power to make the choice. :)

Let's make a note of this for next semester's Activity Book. :)

Cassandra}


\todo[inline]{Here is what you proposed with the FNTs added so we can see them:

DLM 7a:
Act 2.2.2
Act 2.3.1
(if time catch up with what you skipped in previous lab)
Regular FNTs assigned EXCEPT writing up 2.2.3 neatly.
FNTs: 2.3.1-1; 2.3.1-2; 2.2.1-7; 2.2.1-8; 2.3.1-3; 


DLM 7b:
Catch up with anything you are behind on THEN
Act 2.2.3
If you have any extra time, you can solve previous quiz, or review tricky concepts.
FNTs are to write up the solution to Act 2.2.3
FNTs: 2.2.3-1


I propose the following:
DLM7A
Catch-up 
ACT 2.2.2
FNTs: 2.2.1-7; 2.2.1-8; (according to current schedule we go over these in DLM8)

DLM7B
Act 2.3.1
Act 2.2.3
FNTs: 2.3.1-1; 2.3.1-2; 2.3.1-3 (according to current schedule we go over these in DLM8)

We still have the minor issue of not going over HW from 7A until DLM8 as we did in your proposal. I think this solution is better b/c I'm worried that ACT 2.2.2 will take a long time (the Instructor notes don't have an estimate for this one) and we will end up rushing through ACT 2.2.3 (which is a difficult activity). In my proposed schedule, we do Act 2.2.2 (long) and Catch-up (shorter), and then we split the time for both Act 2.3.1 and 2.2.3 which gives them 70min for each activity. Also, the assigned HW is a little more balanced.

What do you think?

Annie}


\todo[inline]{Hi Annie,
For DLM 7, I am going to suggest that we cut 2.2.3 (atwood machine) and make a DLM 7b.

Here is what I propose:

DLM 7a:
Act 2.2.2
Act 2.3.1
(if time catch up with what you skipped in previous lab)
Regular FNTs assigned EXCEPT writing up 2.2.3 neatly.


DLM 7b:
Catch up with anything you are behind on THEN
Act 2.2.3
If you have any extra time, you can solve previous quiz, or review tricky concepts.
FNTs are to write up the solution to Act 2.2.3


Then DLM 8 picks up with FNTs from DLM 7a.
(we will likely also cut 2.4.2 from that set to keep the slower pace.. and also I don't think that activity is relevant.)


What do you think?

Cassandra}


\section*{Unit 6}

\todo[inline]{Act 6.1.1 there are 1's instead of arrows

Unit 6 says ``Whole class discussion''}

\todo[inline]{Change assigned FNTS in DLM 9 to be 6.1.2-1 through 6.1.2-3

Change 6.1.3 in DLM 10 to be 6.1.2-1 through  6.1.2-3
Move 6.1.4 to DLM 10 as activity 2

Cut 7.1.1 from DLM 10a

FNTs from 10a will be FNTs 6.1.2-4 through 6.1.2-7

Create DLM 10b $\rightarrow$ 7.1.1, review FNTs 6.1.4-7}

\section*{Unit 7}

\todo[inline]{get rid of note after but on arm problem 7.3.2-1}

\todo[inline]{FNT 7.1.1-1 ``The three parts of this problem'' when there are actually 5 parts}

\todo[inline]{Hey CP,

Here is what we discussed:
We broke up ACT 7.2.3 so it no longer exists

DLM12A
Second part of ACT 7.1.2
ACT 7.1.3B
FNTs: 7.2.1-1; 7.1.1-4;7.1.1-5; 7.1.1-6; 7.1.1-7

DLM12B
ACT 7.1.3A \& go over FNT 7.1.1-7
ACT 7.2.1
Egg Throw
FNTs: 7.2.1-2; 7.2.1-3; 7.2.1-4; 7.2.1-5; 7.2.1-6; 7.2.1-7; 7.2.1-8; 

The rest is tentative:

DLM 12C
ACT 7.2.2 \& go over 7.2.1-7; 7.2.1-8; 
ACT 7.3.1
FNTs: 7.3.1-1; 7.3.1-2; 7.3.1-3; 7.3.1-4; 7.3.1-5; 7.2.1-9; 7.2.1-10; 7.2.1-11

DLM 13 
ACT 7.3.2
ACT 7.3.3

I've also attached the pic.

Best,
Annie}

\todo[inline]{DLM13B:
Act 7.3.2 (Going over FNTs 7.3.1-1;7.3.1-2;7.3.1-3;7.3.1-4;)
Finish Act 7.3.1
Act 7.3.3 (beam Act - skip bond stuff)
FNTs due DLM14A (7.3.2-1(arm problem);7.3.4-1 (if complete beam Activity))

We need more angular momentum problems where students use the chart.}


\todo[inline]{vectors in picture on FNT 7.2.1-3}

\todo[inline]{Hi all,
If you were in our meeting yesterday, you'll remember that we decided that the students needed to be able to take time to make sense of the angular momentum ideas by drawing on their linear momentum ideas before doing activity 7.3.1.

Therefore, we created a mini activity to do right before 7.3.1. It is attached. Please let me know if you have any questions about what to do in discussion lab. 

Just to sum up, you will be doing:

Activity 7.2.3B (covers FNTs 7.2.1-9 through 7.2.1-11)
Pre7.3.1 activity (attached.... but nothing to hand out to students)
first half of 7.3.1.}

\todo[inline]{Hey CP,

Here's what we decided on yesterday:

DLM12C:
ACT 7.2.2 
      Goes over FNTs: 7.2.1-2;7.2.1-3;7.2.1-4;7.2.1-5;7.2.1-6;
First half of ACT 7.2.3 
       Goes over FNTs: 7.2.1-7;7.2.1-8;
Egg Throw Discussion
FNTs due DLM13A: 7.2.1-9;7.2.1-10;7.2.1-11;

DLM13A
Second half ACT 7.2.3
         Goes over FNTs: 7.2.1-9;7.2.1-10;7.2.1-11;
ACT 7.3.1
FNTs due DLM13B: 7.3.1-1;7.3.1-2;7.3.1-3;7.3.1-4;7.3.1-5;

Best,
Annie}

\todo[inline]{Hi all,

I'd like to request an entry in the activity book for next semester: I think that calling the perpendicular component of the torque ``tangential'' seems highly confusing to me because it's not necessarily tangential to the surface of the object that the torque is acting on. For example, in FNT 7.3.1-2, we have a circle with pivot-point ``theta.'' While F3 and F5 in this drawing have ``tangential'' components that are both tangential to the ``arc of motion'' (for lack of a better term…) and to the circle that's being rotated, the tangential components of all the other forces are not tangential to the circle.

I think if we would call the components ``perpendicular'' and ``parallel'' to the moment arm, we would make things less confusing\ldots

Also, I think we should replace all bolded letters that denote vectors in all materials with ``arrowed" letters. My students still have issues recognizing bold letters as vectors, even though we talked about this numerous times. I think it would be best if we could keep the notation in the materials as close as possible to the notation they can use themselves (they can't bold anything on whiteboards or in homework assignments).

Thanks,
Benedikt}

\section*{Unit 8}


\section*{Unknown}


\todo[inline]{What do you think of these suggestions for first yellow sheet full of specific heats / melting points etc.
	
	Rework so that it is all in degrees C?
	
	Get rid of /mol stuff?
	
	remove conversion from K $\rightarrow$ C and vice versa b/c we will do everything in C?}


\todo[inline]{Instructor notes refers students to pg 8-9 of course notes - may not be the same for us - just insert labeling convention.}


\todo[inline]{expand track and pool ball activity w/ phones}


\todo[inline]{rewrite salt problem.

At home:
Two ice cubes, salt one of them and observe
create energy interaction model to explain process

look at diagrams in class
Repeat experiment and measure temperature.

------------------------------------------------------------------------------

model ice cream making process - 
with ice cream shaking ball}


\todo[inline]{7B get rid of ``ask students why they have a straight line plotted for KE when KE is dependent'' in instructor notes

Get rid of all the ``intuition talk'' in activity book. Benedikt will flag.}


\todo[inline]{Hi Annie,
Here is the egg throw activity with your's and Eric's edits. I hope it's ok that I left the formatting to you.
Cassandra}


\todo[inline]{Get rid of elastic and inelastic language.}




\todo[inline]{inertia sticks activity

problem says hold the sticks at the ends not in the middle.}


\todo[inline]{The solution to the water balloon problem shows the specific heat of water in Joules (which is right you need to covert to joules) but the number is not converted to Joules.}



\todo[inline]{Note: I can't remember how I created the hard-copy version of the Act. book. Did I use the 2A\_ActivitiesX\_X.docx files -- I know for at least the 7B stuff I just printed CP's ``Original2Abook''}


