\section[Momentum and Change in Momentum: 1-D Cases]{Momentum and Change in Momentum:\\One-Dimensional Cases}
\label{act7.1.1}

\todo[inline]{Last semester, we talked about possibly doing "Part C" before "Parts A and B." I rearranged these to do just that. Originally, this first subsection came third.}

\begin{overview}

\textbf{Overview:} We've previously discussed that moving objects have kinetic energy. In this section, we'll see that moving objects have another property, \emph{momentum}. Colloquially, you may be familiar with this property as ``Oomph.''
\end{overview}


\subsection{Getting a Feel for Momentum}
\label{act7.1.1c}

\textbf{Phenomenon:} Collisions of a cart with another object.\\

\noindent In your small group, you are going to observe, talk about, and analyze simple collisions in one dimension using carts that slide almost frictionlessly on a long aluminum track.\\

\noindent\textbf{Please be gentle with the carts and track. Thanks!}\\

\noindent Make sure everyone in your group fully understands the ideas behind each question or part in these activities before going on to the next part.

\begin{enumerate}
	\item Arrange a collision so that a cart sticks to the bumper of the track. Observe the collision several times (observe the motion). Analyze the collision using the \pConsModel{}.
	\begin{enumerate}
		\item Use vectors to represent the various momenta. Draw on the board a \pchart{}. Each column should have appropriately scaled and labeled vectors for either the momentum or the \emph{change} in momentum $\Delta \vec{p}$ during the collision.
		
		\item Write a vector equation for the momentum, off to the side of your chart, with appropriate subscripts on the symbols, to express what you observed during this collision.
		
		\item Describe in words what physically happened (tell a story!) and how conservation of momentum applies in this situation. Draw a force diagram for the cart for the time when its momentum was changing. Put your diagrams, equation, and story on the board, but leave enough space so you can write the responses to (2) for comparison.
	\end{enumerate}
	
	\item Arrange a collision so that a cart bounces off the bumper. Observe the collision several times. Analyze the collision using the \pConsModel{}, and repeat Steps~(a) to (c), as above.
	
	\todo[inline]{We also talked about removing the "elastic/inelastic" language from this and following activities.}
	
%	\item When total kinetic energy is conserved in a collision, we call it an elastic collision. If it is not conserved, we call it an inelastic collision. Classify the two collisions above as elastic or inelastic.
	\item For each of the two collisions above, is the total kinetic energy conserved?
		
%	\item Does an inelastic collision violate conservation of energy?  If not, which other energy systems could the kinetic energy have gone to?
	\item Does a case in which total kinetic energy is not conserved violate the \emph{conservation of energy} law? If not, which other energy systems could the kinetic energy have gone to?
	
%	\item Is momentum conserved in the inelastic case and the elastic case?  Where in your \pchart{} is this shown?
	\item Is momentum conserved in the case where total kinetic energy is conserved and in the case where total kinetic energy is not conserved? Where in your \pchart{} is this shown?
\end{enumerate}

\WCD

\subsection{Change in Momentum and Impulse in One Dimension}
\label{act7.1.1a}

%(See Model Summary \#6)

A change in momentum of an object is caused by a net force acting on the object for a certain amount of time. We define the \emph{Net Impulse} $\Sigma\vec{I}$ as the product of the net force $\Sigma\vec{F}_\text{on object}$ and the amount of time $\Delta t$ during which the net force is acting on the object:
\vspace{-5pt}

\begin{equation*}
	\text{Net Impulse} = \Sigma\vec{I} = \Sigma\vec{F}_\text{on object} \cdot \Delta t
\end{equation*}

\noindent Because a net impulse $\Sigma\vec{I}$ causes a change in momentum $\vec{p}$ of the object, we can also write:
\vspace{-10pt}

\begin{equation*}
	\overset{\mbox{\normalfont\tiny\sffamily cause}}{\text{Net Impulse}} = \Sigma\vec{I} = \Sigma\vec{F}_\text{on object} \cdot \Delta t = \vec{p}_f - \vec{p}_i = \Delta\vec{p} = \overset{\mbox{\normalfont\tiny\sffamily effect}}{\text{Change in Momentum}}
\end{equation*}

\noindent Remember the law of Energy Conservation? We observed that when there is no energy coming into or going out of a given physical system, the total energy of that system is conserved, which means the total change of energy $\sum\Delta E$ is equal to zero. There is a similar phenomenon with momentum, the \emph{Conservation of Momentum}. In this case, the total momentum of a system of objects remains unchanged:
\vspace{-5pt}

\begin{equation*}
	\Delta \vec{p}_\text{system} = \vec{p}_{f,\text{sys}} - \vec{p}_{i,\text{sys}} = 0 \text{  when  } \Sigma\vec{I} = 0 \text{  or  } \Sigma\vec{F} = 0	
\end{equation*}

\begin{enumerate}
	\item In your small group, describe an example of an impulse. Identify two ways you can change this impulse. For instance, how could you make the impulse greater?   Describe both 
	\begin{enumerate}
		\item a system that involves one object and 
		\item a system that involves two interacting objects.
	\end{enumerate}
	Explain what $\Delta \vec{p}_\text{system}$ is for each of your physical systems. Put this on the board.
	
	\item In your small group, develop a statement in your own words of what \textbf{\em conservation of momentum} means for your two systems in (1).
\end{enumerate}

\WCD

\newpage

\section[Vectors and Momentum Conservation]{Representing Momentum Conservation with Vectors}
\label{act7.1.1b}
%(See Page~54 of Course Notes)

\begin{overview}
	\textbf{Overview:} Just as \EnergyDiagrams{} are useful in helping us work through conservation of energy questions/problems, \pcharts{} are useful for questions/problems involving conservation of momentum. The \pchart{}, like an \EnergyDiagram{}, helps us keep track of what we know about the interaction, and it also helps us see what we do not know.
\end{overview}

\noindent\textbf{All \pcharts{} are to be filled in with \emph{scaled} arrows representing momentum vectors.}\footnote{The algebraic expression for a momentum vector is  $\vec{p} = m\vec{v}$, where $\vec{v}$ is the velocity vector. This means that velocity and momentum point in the same direction at any instance in time!}
\vspace{-6pt}

\begin{figure}[h!]
	\centering
	\begin{subfigure}[b]{0.45\textwidth}
		\centering
		\caption*{\textbf{Closed System}\\Typically used for collisions/interactions involving two or more objects:}
\vspace{-6pt}
		\begin{tikzpicture}[thin,scale=0.85, every node/.style={transform shape},background rectangle/.style={fill=white}, show background rectangle]

			    % draw timeline   
			    \draw[->] (1,4.75) -- (7,4.75);

			    % draw vertical lines
			    \foreach \x in {1.5,6.5}
			      \draw (\x cm,4.75cm+2pt) -- (\x cm,4.75cm-2pt);

			    % draw nodes
			    \draw (1.5,4.75) node[below=1pt] {\scriptsize initial conditions} node[above=1pt] {\scriptsize\emph{beginning}};
			    \draw (6.5,4.75) node[below=1pt] {\scriptsize final conditions} node[above=1pt] {\scriptsize\emph{end}};

			% draw table
			\draw (0,0) -- (0,4);
			\draw[very thick] (2,0) -- (2,4);
			\draw (4,0) -- (4,4);
			\draw (6,0) -- (6,4);
			\draw (8,0) -- (8,4);
			\draw (0,0) -- (8,0);
			\draw[very thick] (0,1) -- (8,1);
			\draw (0,2) -- (8,2);
			\draw[very thick] (0,3) -- (8,3);
			\draw (0,4) -- (8,4);
			
			% label table
			\node[text width=2cm, align=center] at (1,3.5)
				{Closed $\vec{p}$ system};
			\node[text width=2cm, align=center] at (3,3.5)
				{$\vec{p}_i$};
			\node[text width=2cm, align=center] at (5,3.5)
				{$\Delta\vec{p}$};
			\node[text width=2cm, align=center] at (7,3.5)
				{$\vec{p}_f$};
			\node[text width=2cm, align=center] at (1,2.5)
				{Object 1};
			\node[text width=2cm, align=center] at (1,1.5)
				{Object 2};
			\node[text width=2cm, align=center] at (1,0.5)
				{Total System};
			\node[text width=2cm, align=center] at (5,0.5)
				{0};
		\end{tikzpicture}
		\caption*{For total system: $\Delta \vec{p} = 0$\\
	For each object: $\vec{p}_i + \Delta \vec{p} = \vec{p}_f$}
	\end{subfigure}
	\hspace{0.05\textwidth}
\vspace{-6pt}
	\begin{subfigure}[b]{0.45\textwidth}
		\centering
		\caption*{\textbf{Open System}\\Typically used when the phenomenon involves a \textbf{net impulse} acting on the system:}
		\vspace{-6pt}
		\begin{tikzpicture}[thin,scale=0.85, every node/.style={transform shape},background rectangle/.style={fill=white}, show background rectangle]
			\draw[white] (0,0) -- (8,0);

			    % draw timeline   
			    \draw[->] (1,4.75) -- (7,4.75);

			    % draw vertical lines
			    \foreach \x in {1.5,6.5}
			      \draw (\x cm,4.75cm+2pt) -- (\x cm,4.75cm-2pt);

			    % draw nodes
			    \draw (1.5,4.75) node[below=1pt] {\scriptsize initial conditions} node[above=1pt] {\scriptsize\emph{beginning}};
			    \draw (6.5,4.75) node[below=1pt] {\scriptsize final conditions} node[above=1pt] {\scriptsize\emph{end}};

			% draw table
			\draw (0,2) -- (0,4);
			\draw[very thick] (2,2) -- (2,4);
			\draw (4,2) -- (4,4);
			\draw (6,2) -- (6,4);
			\draw (8,2) -- (8,4);
			\draw (0,2) -- (8,2);
			\draw[very thick] (0,3) -- (8,3);
			\draw (0,4) -- (8,4);
			
			% label table
			\node[text width=2cm, align=center] at (1,3.5)
				{Open $\vec{p}$ system};
			\node[text width=2cm, align=center] at (3,3.5)
				{$\vec{p}_i$};
			\node[text width=2cm, align=center] at (5,3.5)
				{$\Delta\vec{p}$};
			\node[text width=2cm, align=center] at (7,3.5)
				{$\vec{p}_f$};
			\node[text width=2cm, align=center] at (1,2.5)
				{Total System};
		\end{tikzpicture}
		\caption*{For total system: $\Delta \vec{p} = \Sigma\vec{I}$\\
	\phantom{For total system: } $\vec{p}_i + \Delta\vec{p} = \vec{p}_f$}
	\end{subfigure}
	\vspace{-4pt}
	\caption*{To identify any forces that cause the object's change in momentum (change in motion), it helps to draw a force diagram for each object in the \pchart{}.}
\end{figure}
\vspace{-18pt}

\begin{enumerate}
	\item On your boards, complete the appropriate \pchart{} for each of your examples from Activity~\ref{act7.1.1a}.
	
	\item Each row and each column represent a separate vector equation. Check that every row equation and every column equation are added correctly and are consistent.
	
	\item Which column is significant for showing whether momentum is conserved?
	
	\item Determine what parts of the \pcharts{} are analogous to \EnergyDiagrams{}. List the analogous parts on the board. In what fundamental way do these diagrams differ?
\end{enumerate}
\vspace{-10pt}

\WCD 

