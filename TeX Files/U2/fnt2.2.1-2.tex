\label{fnt2.2.1-2}

\noindent With the same initial conditions as in \ref{fnt2.2.1-1}, use the \EnergyInteractionModel{} \textbf{\em in two different ways} to determine the speed of the ball when it is 4 meters above the floor, {\em headed down}:

\begin{enumerate}[(a)]
	\item Construct a particular model of {\em the entire physical process}, with the initial time when the ball leaves Christine's hand, and the final time when the ball is 4~meters above the floor, {\em headed down}.
	\item Divide the overall process into {\em two physical processes} by constructing two \EnergyDiagrams{} and applying energy conservation for each: one diagram for the interval corresponding to the ball traveling from Christine's hand to the maximum height; and one diagram corresponding to the interval for the ball traveling from the maximum height to 4 meters above the floor, {\em headed down}.
	\item Did you get different answers in parts (a) and (b) for the speed of the ball when it is 4 meters above the floor, {\em headed down}?
\end{enumerate}