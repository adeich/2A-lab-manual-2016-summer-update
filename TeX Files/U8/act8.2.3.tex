\section[\motiongraphs{} and Newton's 2nd Law]{Making Sense of Motion through Graphs And Newton's 2nd Law}
\label{act8.2.3}
\note{For Activity~\ref{act8.2.3} (\about\unit[50]{min})
}{
There are additional questions for the students to work on for all these \FNTs.  In some cases, STUDENTS DO NOT PUT UP ANSWERS, but FOCUS ON THE REASONS FOR THE ANSWERS!  
\\[0.25in]
This activity is highly unpopular with some students�but keep emphasizing that they need to know �how to approach and analyze� these questions, not just �get the answer to a multiple choice question.�
\\[0.25in]
Time management:  depending on time constraints or how your \DLM{} is going it might be best to assign different \FNTs{} to the tables.  For example, assign the follow-ups to \FNT~\thechapter-\ref{fnt8.2.1-4} to three tables while the remaining two tables put up \FNT\FNT~\thechapter-\ref{fnt8.2.1-5} follow-up.  Then have a \WC{} discussion.  Then ALL tables put up the \FNT~\thechapter-\ref{fnt8.2.1-6} follow-up.  And after the \WC{} discussion, all groups put up \FNT~\thechapter-\ref{fnt8.2.1-7}.  This way all \FNTs{} are covered; be sure to let the students do the explaining during the \WC{} presentations.
}
\begin{overview}

\textbf{Overview:} We continue our exploration of \motiongraphs{}, especially in the context of Newton's 2nd Law.

\end{overview}

\begin{FNTenv}
	\label{fnt8.2.1-4}

Interpret the graph to the right as representing the velocity of an object. Rank the points in order of increasing acceleration (from most negative to most positive). Practice walking this plotted motion.
\end{FNTenv}

\noindent Discuss your group's response to the following prompts and then put your responses up on the board.
\begin{enumerate}
	\item Describe in words the motion of the object whose velocity is shown in the graph. Practice walking this graph for the whole class presentation.
	\item Explain exactly how you can determine the acceleration from a graph of velocity vs.\ time.
	\item Construct an acceleration graph from the velocity graph shown in this \FNT{}.
	\item Describe in words how the net force on this object changes from one lettered point to the next.
\end{enumerate}

\WCD

\begin{FNTenv}
	\label{fnt8.2.1-5}

The velocity graph of a \unit[1]{g} coffee filter released from rest is shown below. Note the break in the time axis. Four distinct intervals are shown on the graph.
\begin{enumerate}[(a)]
	\item Speed downward increasing from rest, $\unit[0]{s} \leq t \leq \unit[0.02]{s}$.
	\item Speed downward increasing, $\unit[0.02]{s} \leq t \leq \unit[0.09]{s}$.
	\item Speed downward is constant, $\unit[0.09]{s} \leq t \leq \unit[4.99]{s}$.
	\item Landing on the floor, $\unit[4.99]{s} \leq t \leq \unit[5.00]{s}$.
\end{enumerate}
Draw an acceleration graph for the same time intervals above. You may use the tangent lines drawn on the velocity graph to calculate the average slopes of the velocity curve during the first two intervals.
\end{FNTenv}

\noindent Discuss how you can be sure that the values of the acceleration you determined from the velocity graph in regions (a) and (c) are the correct values according to Newton's 2nd law. Show this ``check'' on the board. Note: you do not have to put the graph on the board.\\

\WCD

\newpage

\note{\FNT~\thechapter-\ref{fnt8.2.1-6} part c:}{The big idea here is that the acceleration is related to the change in velocity by definition and to the net force through Newton�s 2nd law.  These have to agree. Often, as in this case, one knows for certain the net force in certain situations, and this can be used to determine the acceleration.
\\[0.25in]
At the points [2] and [3] on the graph the assumption is that these points are during the time the floor is exerting a force on the ball and that the velocity is smoothly varying between these points for simplification.
}

\begin{FNTenv}
	\label{fnt8.2.1-6}

Refer to the graphs on the reverse which were made using a motion detector mounted above a basketball that was dropped from a height of \unit[1.5]{m} above the floor. The position measured and indicated is really the position of the top surface of the ball. The velocity graph was computed by the software as the derivative of the position vs.\ time graph. Complete the following tasks related to this situation.

\begin{enumerate}[(a)]
	\item Determine accurately the acceleration from the velocity graph and plot it on the acceleration axis. Make sure you extend the acceleration curve as far as the other two are extended in time.
	
	\item Describe the motion of the ball at the six indicated times numbered [1] to [6]. That is, describe where it is located and say something about its speed, direction of motion, and acceleration. If you look closely, you should see that at position [3] the ball is still in contact with the floor.
	
	\item Draw \forcediags{} for each of the six marked times. For which times are the \forcediags{} identical?
	
	\item For the times when the \forcediags{} are identical, which aspects of the motion are identical? Which aspects are different? Are your answers to the previous two questions consistent with Newton's 2nd law?
	
	\item Determine the average value of the force exerted by the floor on the ball between the times numbered [2] and [3] two different ways:
	\begin{enumerate}[(i)]
		\item from the impulse imparted to the ball from the floor and the velocity graph; and 
		\item using Newton's 2nd law.
	\end{enumerate}
	Explain two ways (once for each approach) why the force of the floor on the basketball when it bounces is so much greater than the basketball's weight.
\end{enumerate}
\end{FNTenv}
\note{\FNT~\thechapter-\ref{fnt8.2.1-6} part d:}{
At points [1], [2] and [5] where the velocity is zero, there is a net force acting (unbalanced forces).  Therefore, there must be a non-zero acceleration which is a change in velocity at that instant, regardless of the value of the velocity.  \textbf{Emphasize that unbalanced force diagrams describe a change in motion. } and \text{remind them that a net force is not a real force, but it is the sum of real forces.}
}
\noindent Discuss your group's responses to all parts of this \FNT{} and respond to the following prompts.
\begin{enumerate}
	\item Put your responses to part (a) on the board.
	\item Put only the \textbf{different} \forcediags{} on the board for part (c) and note above each diagram which points -- 1 through 6 -- it applies to. Make sure everyone in your group understands part (d) and is prepared to share with the class.
	\item Make sure everyone in your group understands part (e) and is prepared to share with the class.
\end{enumerate}

\WCD
\begin{FNTenv}
	\label{fnt8.2.1-7}

Consider the following ``problem'' that many beginning physics students struggle with: ``How is it that at a certain instant in time, an object can have zero velocity, but at that same instant, have a non-zero acceleration?''  Figure out how to explain this using the graphs of the motion of the dropped and bouncing basketball, the basic definitions of velocity and acceleration, and Newton's 2nd law.
\end{FNTenv}

\noindent Discuss this \FNT{} in your group and put a \textbf{concise} explanation on the board.\\

\WCD
\begin{FNTenv}
	\label{fnt8.1.1-1}

\textbf{Cart and Horse Paradox:} If a horse pulls on a cart, and the cart pulls back on the horse with an equal magnitude force, how can either possibly begin to move?

Use what you have learned about force to \textbf{give a complete explanation of this paradox.} For a complete explanation, make and refer to complete \forcediags{} for each of the following.
\begin{enumerate}[(a)]
	\item the horse, 
	\item the cart, and 
	\item the whole system (horse \& cart).
\end{enumerate}
\end{FNTenv}

\noindent Discuss this \FNT{} in your group, put the three \forcediags{} on the board, and make sure everyone in your group understands these and is prepared to share with the class.\\

\WCD