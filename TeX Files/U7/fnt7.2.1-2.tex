\label{fnt7.2.1-2}

Turn to the \pModel{} summary page and do the following:

\begin{enumerate}
	\item Use the model relationships found there to analyze each physical situation, and
	\item Make logical arguments that would convince another physics student of your response to the following prompts.
\end{enumerate}

\noindent Remember that a momentum conservation law requires you to compare a quantity at two times so you must always consider an initial time and a final time.\\

\noindent\parbox[t]{\textwidth}{%
      \vspace{-3mm}
      \begin{wrapfigure}[10]{r}{3cm}
        \centering
        \vspace{-\baselineskip}%\vspace{-10pt}
			\begin{tikzpicture}[decoration={markings,mark=at position 0.5*\pgfdecoratedpathlength with {\arrow[thick]{>}},mark=at position 1*\pgfdecoratedpathlength with {\arrow[thick]{>}}},thick,scale=0.8, every node/.style={transform shape},background rectangle/.style={fill=white}, show background rectangle]{r}{2cm}
  	\centering
		\node[inner sep=0pt] (fingers) at (-.32,.16) {\includegraphics[width=1cm]{pinchedFingers.png}};
		\draw[postaction={decorate},dashed] (0,0) circle [radius=1.5cm];
		\draw (0,0) -- (1.06,-1.06);
		\draw[fill=gray] (1.06,-1.06) circle (.1) node[right=2pt] {$P$};
	\end{tikzpicture}
\end{wrapfigure}
\noindent A heavy ball is attached to a string and swung in a circular path counter-clockwise in a horizontal plane as illustrated in the diagram to the right.  At point $P$ indicated in the diagram, the string suddenly breaks and the ball is released.  If these events were observed from directly above, draw the path the ball takes immediately after the string breaks.
}