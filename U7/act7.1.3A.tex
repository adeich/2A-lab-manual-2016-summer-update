%\section{The Importance of $\Delta t$ in Collisions}
\subsection{Automobile crash}
\label{act7.1.3A}

%Use the following phenomenon for \ref{fnt7.1.1-4}, \ref{fnt7.1.1-5}, and \ref{fnt7.1.1-6}.\\

\noindent\textbf{Phenomenon:} You are riding in a car that crashes into a solid wall.  The car comes to a complete stop without bouncing back.  The car has a mass of \unit[1500]{kg} and has a speed of \unitfrac[30]{m}{s} before the crash (this is about \unitfrac[65]{mi}{hr}). 

\begin{fnt}
	\label{fnt7.1.1-4}

%You are riding in a car that crashes into a solid wall.  The car comes to a complete stop without bouncing back.  The car has a mass of \unit[1500]{kg} and has a speed of \unitfrac[30]{m}{s} before the crash (this is about \unitfrac[65]{mi}{hr}).  

\begin{enumerate}[(a)]
	\item What is the car's initial momentum?
	\item What is your initial momentum? Recall that the weight of one kilogram is \unit[2.2]{lbs}.
	\item Draw separate \pcharts{} for the car and the person. Treat both as open systems with a net impulse.
	\item What is the change in the momentum of the car?
	\item What is the change in your momentum?
\end{enumerate}
\end{fnt}

\begin{fnt}
	\label{fnt7.1.1-5}

%You are riding in a car that crashes into a solid wall.  The car comes to a complete stop without bouncing back.  The car has a mass of \unit[1500]{kg} and has a speed of \unitfrac[30]{m}{s} before the crash (this is about \unitfrac[65]{mi}{hr}).  

\begin{enumerate}[(a)]
	\item What is the net impulse that acts on the car to bring it to a stop?
	\item What is the net impulse that acts on you to bring you to a stop?
\end{enumerate}
\end{fnt}

\begin{fnt}
	\label{fnt7.1.1-6}

%You are riding in a car that crashes into a solid wall.  The car comes to a complete stop without bouncing back.  The car has a mass of \unit[1500]{kg} and has a speed of \unitfrac[30]{m}{s} before the crash (this is about \unitfrac[65]{mi}{hr}).  

Refer to your \pcharts{} from \ref{fnt7.1.1-4}. Consider the following two situations.

\begin{enumerate}[I.]
	\item You remain buckled into the seat and the seat remains attached to the center of the car.
	\item You are not buckled into your seat and you fly through the windshield and hit the wall.
\end{enumerate}

\begin{enumerate}[(a)]
	\item Is the magnitude of the impulse the same in both cases?
	\item Are the magnitudes of the forces acting on you the same?
	\item Using the words impulse, force, time, and momentum explain why one scenario is safer for you.  Hint: Think about how the shape of the car changes when it hits the wall.
\end{enumerate}
\end{fnt}

\begin{enumerate}
	\item In your group, decide on two specific scenarios (\ref{fnt7.1.1-6}) and use these scenarios in Parts~2, 3, and 4 below. For these two scenarios, think about how much time passes between the time the force is first applied by the object and the time when you have zero momentum. In which scenario will this time difference $\Delta t$ be larger and why?
	
	\item  On the board, put up complete \pcharts{} (with force diagrams and equations worked through to numerical values) for the car and for the person in each of the two scenarios you have chosen. Describe the scenario above each of the \pcharts{} for the person.
	
	\item Use your \pcharts{} to explain why the net force acting on the person will not be the same in both scenarios.
	
	\item For each of the scenarios that you described in \ref{fnt7.1.1-6}, estimate the magnitude of the average force that would have been acting on you to bring you to a stop.
	\begin{enumerate}
		\item You will have to determine the time during which the impulse acts for the different scenarios. To do this you must first decide on the initial and final momentum for the cases you are describing and over \emph{what distance} the impulse acts.
		
		\item Now you need to \emph{calculate} the time duration of the impulse using your knowledge of how distance and time are related. If you assume that you slow down at a constant rate, then your average speed during this time is one-half your initial speed.
		
		\item Make an estimation of the average force for the two situations. If your answers differ for the two situations, explain what factor is causing this difference.
	\end{enumerate}
	
	\item If you know the initial and the final momentum, you know the impulse. In each of your scenarios did you have the same or a different impulse? What then is the effect of changing $\Delta t$ (with the given constraints in the case of this automobile crash)?
	
	How does this compare to your response to Question~5 for the tablecloth trick? Which parameters are variable and which are constrained in the scenarios for each phenomenon?
\end{enumerate}

Be prepared to explain what $\Delta t$ is and why it is important in a momentum problem!

\vspace{8pt}
\WCD