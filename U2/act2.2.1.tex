\section{Quantitative Modeling of Mechanical Energies}
\label{act2.2.1}

\begin{overview}

	\textbf{Overview:} There are interesting features of some of the Mechanical Energy systems that only become apparent in the algebraic expression for the change in energy. These features can help us tremendously in making sense of physical phenomena. To remind you, the energy systems we are currently working with are $KE_\text{translational}$, $KE_\text{rotational}$, $PE_\text{gravity}$, $PE_\text{spring-mass}$, and $PE_\text{spring}$. In this activity, we will explore some quantitative features of these energy systems.
	
\end{overview}


\subsection{Relationships between Energy Systems and Indicators}

 In your small group, take a few minutes to think about and discuss {\em each} of the questions below. Each question will be followed immediately by a short \WCD

\begin{enumerate}
	\item What are the indicators of change in each of the Mechanical Energy systems?
	\item \begin{enumerate}
		\item Which of these energies depend on the square of the indicator and which depend on the first power of the indicator?
		\item Which of these energies depend on which direction something moves?
		\item What is the connection between the two previous questions?
	\end{enumerate}
	\item Which of these energy systems depend on the mass of the object and which depend on the weight? Why does this matter?
\end{enumerate}

\noindent
The fact that some indicators are linear and some are quadratic has additional consequences as the next two questions illustrate:

\begin{enumerate}\setcounter{enumi}{3}	
	\item Consider a heavy ball. Would it take the same amount of work to vertically lift it from \unit[25]{cm} to \unit[30]{cm} as it does to lift it from \unit[30]{cm} to \unit[35]{cm}? If not, which situation requires more work? Why?
	
	\item Does it take the same amount of work to speed your car up from \unitfrac[25]{m}{s} to \unitfrac[30]{m}{s} as it does to speed it up from \unitfrac[30]{m}{s} to \unitfrac[35]{m}{s}? If not, situation requires more work? Why?
\end{enumerate}

\subsection{Keeping Track of Directionality}
\label{act2.2.1b}

We are about to begin using the \EnergyInteractionModel{} to predict quantitative properties of various phenomena. A potential pitfall of quantitative modeling is that we sometimes tend to lose track of why we are doing something once we start calculating numbers. It is especially easy to lose track of \emph{directionality}, which is mathematically expressed in \emph{algebraic signs}. However, our \EnergyDiagrams{} tell us the signs of the various terms in the conservation of energy equation most of the time. Therefore, don't forget to use the \EnergyDiagrams{} to check the signs as you begin to model scenarios quantitatively!

\noindent\textbf{The Phenomenon:} Christine throws a ball straight up, letting go of the ball at a height of $y_i$ above the ground. When she lets go, the ball has an initial speed $v_i$. The ball travels straight up to its maximum height ($y_{max}$) and falls back down. Assume the frictional effects from air resistance are insignificant.

\subsubsection*{Constructing a particular model to find the maximum height:}

\begin{enumerate}
	\item Construct a particular model of this phenomenon that can be used to determine the maximum height of the ball. ($y_f = y_{max}$) Put a sketch of the path of the ball and a complete \EnergyDiagram{} (with accompanying energy conservation equation) on the board [leave half the board free for the following questions]. What do you know about the ball's speed at its maximum height when it's thrown straight up?
	
	\item Indicate the initial and final positions of the ball on your sketch that correspond to the initial and final conditions in your \EnergyDiagram{}.
	
	\item On the board, rewrite the energy conservation equation by replacing the two terms with algebraic expressions for the changes in energy of the energy systems. Does the resulting algebraic sign of each term in your energy conservation equation agree with the increases and decreases in energy systems in your diagrams?
	
	\item Solve the equation for ($y_f - y_i$). Does the sign of ($y_f - y_i$) make sense for this particular physical situation?
	\label{act2.2.1b4}
	
	\item If you knew the numerical value of $v_i$, could you calculate the maximum height?
\end{enumerate}

\WCD