\section{Momentum and Change in Momentum in One Dimension, Part II}

\begin{FNTenv}
	\label{fnt7.1.1-1}

You're playing with two of the carts (each with mass $m$) that you used in \hyperref[act7.1.1c]{Activity~\ref*{act7.1.1c}}. Initially, these two carts are moving toward each other with the same initial speed $v_i$ along the track. The carts collide and the result is one of these final states:
\begin{enumerate}[(a)]
	\item Assume that the carts hit each other and stop so that the final state of the system has both carts just sitting still (not moving). Draw a \pchart{} for this situation. Make a separate row for each cart.% Refer to \pConsModel{} 
	\item Assume that the carts bounce off each other so that the final state of the system has each cart moving opposite to its initial motion but with the same speed. Draw a \pchart{} for this situation.
	\label{fnt7.1.1-1b}
	\item As in \eqref{fnt7.1.1-1b}, assume that the carts bounce off each other but now assume that the final speeds are smaller than the initial speeds, equal and in opposite directions. Draw a \pchart{}.
	\item For each case above, does the total momentum of the system that contains the two carts change? How do the \pcharts{} help you answer this question?
	\item Is the total kinetic energy constant for all three cases? How do you know?
\end{enumerate}

\end{FNTenv}

\begin{FNTenv}
	\label{fnt7.1.1-2}

A rocket expels gas at a high speed out of its back for a short period of time. We are going to treat the rocket as being far away from any gravitational objects.

\begin{enumerate}[(a)]
	\item Draw a \pchart{} for the rocket expelling gas in space. Take the initial time before expelling gas and the final time after the rocket has finished expelling gas. The rocket has an initial constant speed in the horizontal direction. Put the rocket and the expelled gas on separate rows.
	\item Use your chart to explain why the rocket speed increases. 
	\item Does the rocket have to keep expelling gas to stay at a constant speed? Explain.
\end{enumerate}
\end{FNTenv}

\begin{FNTenv}
	\label{fnt7.1.1-3}

Victoria is standing on a boat, during a perfectly calm day. Initially, both Victoria and the boat are not moving. Then Victoria walks from one end of the boat to the other. Take the initial time to be before she walks and the final time at some point while she is still walking.

\begin{enumerate}[(a)]
	\item Draw a \pchart{} for this situation. Does the boat move, and if so, in which direction?
	\item Compare the speed of the boat with Victoria's speed. Are they the same or different? Why?
\end{enumerate}
\end{FNTenv}

FNTs  7.1.1-1,  7.1.1-2,  and  7.1.1-3
 In Your Small Group
Compare your responses to \FNTs{} 7.1.1-1, 7.1.1-2, and 7.1.1-3 with other members of your small group; come to a consensus on the appropriate \pcharts{} and answers, and put these on the board.

\textit{Added problem:} Draw an appropriately scaled force diagram (for each object) that shows all the forces acting during the interaction (when the impulse occurs); do this for each \FNT.
\WCD

Phenomenon:  Collisions of 2 carts in one dimension, both elastic and inelastic collisions.

Your instructor will assign each group a problem from 1)-3) below and a problem from 4)-5). Use the \pConsModel{} to analyze each of the collisions between two carts in the situations 1- 3 below. We treat the system made up of the two carts as a closed physical system because there is no net external impulse transferred to the system during the collision.

For each collision:
a)	Draw and fill in a \pchart{} to help you describe momentum conservation in this closed physical system.
b)	For each line of the \pchart{}, write an algebraic vector equation, with appropriate subscripts on the symbols, to express what that line tells you about the collision.
c)	Draw a force diagram for each cart that shows the forces during the collision. 
d)   Describe in words what physically happened and then how conservation of momentum applies to each cart and to the system as a whole. Put your,equation and word statement on the board.
e)   Classify each case as elastic or inelastic.

1)	Use two carts of equal mass, physically arranged so the carts will bounce off one another. Start with one cart stationary.
2)	Use two carts of equal mass, physically arranged so the collision ends with the carts locked (stuck together). 
3)	Use two carts of equal mass initially moving toward each other with equal speed and ending with the carts locked.

4)	Place two carts of unequal mass, on the track turned so the collision ends with the carts locked. Start with one cart stationary, and have the other move to collide with it. Make the stationary cart have more mass (repeat, switching carts). 
5)	Place two carts of unequal mass, on the track turned so the collision ends with the carts locked. Start with both carts moving toward each other with the same speed.

Observe what is similar and what is different in your \pcharts{}. What patterns can you observe? Discuss in your group any rules you come up with for the patterns.	
\WCD







 
