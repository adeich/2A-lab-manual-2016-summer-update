\todo{This badly needs formatting! Skipping for now because this is already printed and it does not seem like we have these FNTs in this version of the manual. Need to look at this before next semester!}

Follow-up of Module 2.4 FNTs

➢	Groups 1 \& 2 discuss and respond to FNT 2.4.1-1 as directed below. 
➢	Groups 3 \& 4 discuss and respond to FNT 2.4.2-1 as directed below. 
➢	Groups 5 \& 6 discuss and respond to FNT 2.4.2-2 as directed below. 

FNT 2.4.1-1 (\unit[\textless10]{min})  
α)	Come to a consensus in your SG about how the changes in $PE_\text{grav}$ compare for the two situations. 
β)	Come to a consensus in your SG about how the maximum heights compare. 

All members of your group must now go to the board and work on this together 

χ)	After you reach consensus (and check with your instructor) about (α) and (β), sketch the two plots of $PE_\text{gravitational}$ on the same graph of Energy vs.\ Height.  
δ)	You know that the force of gravity is constant near the surface of the earth and near the surface of the moon, and that it is approximately six times stronger for the earth than for the moon.  Make sure everyone in your SG is ready to explain how your graph conveys this information.  


All members of your group must now go to the board and work on this together 

FNT 2.4.2-1 (\unit[\textless10]{min})  
α)	For two masses connected by a spring the potential energy is given by: 
$PE_\text{spring} = \frac{1}{2} k(r-r_0)^2$
How does doubling the force constant affect $PE_\text{spring}$ at any particular value of $r$? 
β)	On one graph, sketch two plots of $PE$ for the two different springs. Pick some particular value of $r$, and show explicitly on your graph how doubling the spring constant affects the potential energy of the system. How do these two systems differ physically?
χ)	On a second graph, sketch two plots of $PE$ for the two different values of $r_0$. How do these two systems differ physically?  

All members of your group must now go to the board and work on this together 

FNT 2.4.2-2 (\unit[\textless10]{min})  
α)	Why do we define the $PE$ between atoms or molecules to be zero when $r$ is very large (i.e., when the atoms are far apart instead of when they are close together)? 
β)	Sketch three separate graphs of $PE$ vs.\ separation distance on the board: one for a pair of positively charged particles, one for a pair of negatively charged particles, and one for two particles with opposite charges. Make sure everyone in your SG is ready to explain why the graphs are drawn as they are.


\WCD
