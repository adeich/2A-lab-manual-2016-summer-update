\label{FNT1.1.4-1}

Every general model is applicable to some range of physical phenomena. The \EnergyInteractionModel{} can be used to analyze $v_i$rtually {\em any} process, but up to this point we have been using it to make sense of the phenomena of {\em temperature changes} and {\em phase changes}. The former are modeled as changes in thermal energy systems, with the indicator being temperature, and the latter are modeled as changes in bond energy systems, with the indicator being the mass of substance that is changing phase. But {\em bond energy} changes in chemical reactions just as it does in phase changes, so we use exactly the same approach when modeling chemical reactions.

Consider the following chemical reaction, the hydration of calcium sulfate (Plaster of Paris). Recall that when you mix water with the white power, the paste not only gets hard, but it also gets hot:
\begin{center}
\ce{(Ca2SO4)2 * \ce{H2O} + 3\ce{H2O} -> 2Ca2SO4 * 2\ce{H2O} + Heat_{to~env.}}
\end{center}
Has the bond energy of this \textbf{\em total} system increased, decreased, or stayed the same during this process (mixing water with the powder)?  Use the \EnergyInteractionModel{} to explain. The simplest way to answer this question is to model all of these different chemicals as being a {\em single} physical system with {\em one} thermal-energy system and {\em one} bond-energy system. (Note that the question asked about the {\em total bond energy}, not bond energies of the separate molecular species. If the question had asked about bond energy changes of particular molecular species, we would have to include separate bond energies in our model.)


