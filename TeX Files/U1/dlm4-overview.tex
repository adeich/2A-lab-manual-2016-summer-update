\section*{Brief Overview}
The first \CLASP{} Activity, \ref{act1.1.6}, based on \ref{FNT1.1.3-4}, \ref{FNT1.1.3-5}, \ref{FNT1.1.3-6} provides a good review of the qualitative application of both the \ThreePhaseModel{} and the \EnergyInteractionModel{}. You should now be feeling fairly confident with this material. The second Activity, \ref{act1.1.7} based on \ref{FNT1.1.4-1} and \ref{FNT1.1.4-2} apply the \EnergyInteractionModel{} to chemical reactions. The last \CLASP{} Activity continues Module 1.2: Getting Quantitative with Models is a follow-up of \ref{FNT1.2.1-1}, \ref{FNT1.2.1-2}, \ref{FNT1.2.1-3}, \ref{FNT1.2.1-4}.
	
\section*{\CLASP{} Activities}
	
\subsection*{\ref{act1.1.6} Making Sure Things are Making Sense	(\about\unit[35]{min})}
	
\subsubsection*{Purpose}

\begin{itemize}
	\item To provide more additional practice applying both the \ThreePhaseModel{} and the \EnergyInteractionModel{} to fairly simple thermal phenomena.
	\item To provide an opportunity for students to make sure they have gotten the basics of applying these two models.
\end{itemize}

\subsubsection*{Learning Outcomes}
\begin{itemize}
	\item Confidence in being able to apply the two models to simple thermal phenomena.
\end{itemize}

%We're taking the following activity out for now and maybe insert it later if time permits.
%
%\subsection*{\ref{act1.1.7}	Applying the \EnergyInteractionModel{} to chemical reactions	(\about\unit[35]{min})}
%	
%\subsubsection*{Purpose}
%
%\begin{itemize}
%	\item To provide practice applying the \EnergyInteractionModel{} to a different kind of phenomena: chemical reactions.
%\end{itemize}
%
%\subsubsection*{Learning Outcomes}
%
%\begin{itemize}
%	\item Understand how and be able to apply the \EnergyInteractionModel{} to chemical reactions.
%\end{itemize}

\subsection*{\ref{act1.2.2}	Mostly quantitative application of the \EnergyInteractionModel{} to thermal and bond systems		(\about50 mins)}
\subsubsection*{Purpose}

\begin{itemize}
	\item To provide practice in reasoning with the \EnergyInteractionModel{} and explicitly using the constructs of heat capacity and heats of vaporization and melting and the distinction between power and energy.
	\item Practice going through the modeling process without taking shortcuts.
	\item Practice getting quantitative answers using the logic of the model instead of simply ``grabbing an equation and plugging in numbers'' without understanding what is going on.
\end{itemize}

\subsubsection*{Learning Outcomes}

\begin{itemize}
	\item Confidence in the ability to apply the \EnergyInteractionModel{} to answer questions and make predictions for more complicated thermal phenomena.
	\item Understanding that consistently going carefully through the steps of the \EnergyInteractionModel{}, it is possible to avoid the all too common mistakes that occur when you simply grab an equation to plug and chug.
\end{itemize}

\section*{Handouts}

In this DL, each student should receive Handout 1.2.2:  (for use with Act 1.2.2)

\section*{In general, when you run out of time to complete DL Activities}
	
	For any unfinished DL activities from the previous DL (which should have been assigned as an \FNT), give a brief summary (\textless 5 mins.) at the beginning of the current DL and sum up the main points for them. Don't spend time in DL as you normally would having the students discuss and explain.
	
\section*{Suggested times}
	
The suggested times {\em do not} include the \WC{} discussion.

\vspace*{1cm}
Make sure students follow instructions! Prompts and questions are carefully structured to encourage them to process the material in particular ways.

\vspace*{1cm}
\begin{mdframed}
The emphasis in much of this DL is understanding and using the logic of the model, {\em not} getting a numerical answer!
\end{mdframed}