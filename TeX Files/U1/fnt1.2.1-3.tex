\label{FNT1.2.1-3}

Suppose you used a hot pot to convert a \unit[150]{g} piece of ice that was initially at \unit[-15]{\textdegree C} into liquid water at \unit[50]{\textdegree C}.

\begin{enumerate}[(a)]
	\item Represent this process in two complete \EnergyDiagrams{}, one covering the interval from \unit[-15]{\textdegree C} to \unit[0]{\textdegree C} liquid, and the second covering the interval from \unit[0]{\textdegree C} liquid to \unit[50]{\textdegree C}.
	\item Now represent the entire process in one complete \EnergyDiagram{} covering the entire interval.
	\item If your hot pot has a power rating of \unit[600]{watts}, show how to find how long it will take to complete the process. (Make sure you are comfortable using standard units of energy, and that you understand the relation of power to energy.)
\end{enumerate}