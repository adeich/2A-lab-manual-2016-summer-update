\label{fnt8.1.1-3}

Use Newton's laws to analyze the three car train shown in the picture. Car~A is the engine and pulls Cars~B and C. Car~A has a mass of \unit[10000]{kg}, Car~B is \unit[8000]{kg}, and Car~C is \unit[5000]{kg}. The train is initially at rest but then starts to move with an acceleration of $\unitfrac[3]{m}{s^2}$ to the left.

Calculate the force of Car~B on Car~A. Answer the following questions to help you do this.

\begin{enumerate}[(a)]
	\item Draw a \forcediag{} for EACH car with the train at rest.

	\item Car~A powers up its engine and each car starts to accelerate. What provides the force for Car~C to start accelerating?

	\item Draw a \forcediag{} for each car now that each car is accelerating (You may need to come back and update them after you determine the numerical values for each force).

	\item Calculate the force of Car~B on Car~C. 
		
		\textbf{Hint:} the sum of the forces on EACH car is equal to the mass of that car times the car's acceleration. $\sum \vec{F}_A = m_A \vec{a}_A$.
	
	\item What is the net force on Car~B?
	
	\item Calculate the force of Car~A on Car~B.
	
	\item What is the force of Car~B on Car~A?
	
	\item What is the magnitude and direction of the force of the rails on Car~A?
\end{enumerate}