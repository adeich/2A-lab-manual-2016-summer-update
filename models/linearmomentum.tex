\appendixchapter{\pModel{}}

\subsection*{Graphical Representation: The \pchart{}}
%\vspace{-1cm}
\begin{figure}[h!]
	\centering
	\begin{subfigure}[b]{0.45\textwidth}
		\centering
		\caption*{\textbf{Closed System}\\Typically used for collisions/interactions involving two or more objects.}
		\begin{tikzpicture}[thin,scale=0.9, every node/.style={transform shape},background rectangle/.style={fill=white}, show background rectangle]
			% draw table
			\draw (0,0) -- (0,4);
			\draw[very thick] (2,0) -- (2,4);
			\draw (4,0) -- (4,4);
			\draw (6,0) -- (6,4);
			\draw (8,0) -- (8,4);
			\draw (0,0) -- (8,0);
			\draw[very thick] (0,1) -- (8,1);
			\draw (0,2) -- (8,2);
			\draw[very thick] (0,3) -- (8,3);
			\draw (0,4) -- (8,4);
			
			% label table
			\node[text width=2cm, align=center] at (1,3.5)
				{Closed $\vec{p}$ system};
			\node[text width=2cm, align=center] at (3,3.5)
				{$\vec{p}_i$};
			\node[text width=2cm, align=center] at (5,3.5)
				{$\Delta\vec{p}$};
			\node[text width=2cm, align=center] at (7,3.5)
				{$\vec{p}_f$};
			\node[text width=2cm, align=center] at (1,2.5)
				{Object 1};
			\node[text width=2cm, align=center] at (1,1.5)
				{Object 2};
			\node[text width=2cm, align=center] at (1,0.5)
				{Total System};
			\node[text width=2cm, align=center] at (5,0.5)
				{0};
		\end{tikzpicture}
		\caption*{For total system: $\Delta \vec{p} = 0$\\
	For each object: $\vec{p}_i + \Delta \vec{p} = \vec{p}_f$}
	\end{subfigure}
	\hspace{0.05\textwidth}
	\begin{subfigure}[b]{0.45\textwidth}
		\centering
		\caption*{\textbf{Open System}\\Typically used when the phenomenon involves a net impulse acting on the system.}
		\begin{tikzpicture}[thin,scale=0.9, every node/.style={transform shape},background rectangle/.style={fill=white}, show background rectangle]
			\draw[white] (0,0) -- (8,0);
			% draw table
			\draw (0,2) -- (0,4);
			\draw[very thick] (2,2) -- (2,4);
			\draw (4,2) -- (4,4);
			\draw (6,2) -- (6,4);
			\draw (8,2) -- (8,4);
			\draw (0,2) -- (8,2);
			\draw[very thick] (0,3) -- (8,3);
			\draw (0,4) -- (8,4);
			
			% label table
			\node[text width=2cm, align=center] at (1,3.5)
				{Open $\vec{p}$ system};
			\node[text width=2cm, align=center] at (3,3.5)
				{$\vec{p}_i$};
			\node[text width=2cm, align=center] at (5,3.5)
				{$\Delta\vec{p}$};
			\node[text width=2cm, align=center] at (7,3.5)
				{$\vec{p}_f$};
			\node[text width=2cm, align=center] at (1,2.5)
				{Total System};
		\end{tikzpicture}
		\caption*{For total system: $\Delta \vec{p} = \Sigma\vec{I}$\\
	\phantom{For total system: } $\vec{p}_i + \Delta\vec{p} = \vec{p}_f$}
	\end{subfigure}
	\caption*{To identify any forces that cause the object's change in momentum (change in motion), it helps to draw a force diagram for each object in the \pchart{}.}
\end{figure}
%\vspace{-2cm}

\subsection*{Algebraic Representations}

\begin{align*}
	\text{Definition of Momentum }\vec{p}\text{:} && \vec{p} &= m \cdot \vec{v}\\[5mm]
	\text{Net Impulse:} && \sum\vec{I} &= \sum\vec{F}_\text{avg. ext.} \cdot \Delta t\\[5mm]
	\text{Conservation of Momentum:} && \sum\vec{I} &= \sum\vec{F}_\text{avg. ext.} \cdot \Delta t = \vec{p}_f - \vec{p}_i = \Delta \vec{p}_\text{system} = 0\\[5mm]
	\text{Momentum of a Particle System:} && \vec{p}_\text{system} &= \sum_j\vec{p}_j
\end{align*}

\pagebreak

\renewcommand{\leftcolumn}{0.375\linewidth}
\renewcommand{\rightcolumn}{0.625\linewidth}

\noindent
\parbox[b]{\leftcolumn}{
	\textbf{Constructs}}
\parbox[b]{\rightcolumn}{
	\textbf{Relationships}}

\vspace*{-\parskip}
\noindent
\hrulefill

\vspace*{-\parskip}
\noindent
\parbox[c]{\leftcolumn}{
	\noindent
	Momentum ($\vec{p}$)\\

	\noindent
	Position ($\vec{r}$), Displacement ($\Delta\vec{r}$), and Velocity ($\vec{v}$) Vectors\\

	\noindent
	Net Impulse ($\sum\vec{I}$)
		\\\hspace*{1em}$\bullet$ by the net force
		\\\hspace*{1em}$\bullet$ by a single force\\
	
	\vspace{1.5cm}
	\noindent
	Conservation of Momentum\\
	
	\vspace{2cm}		
	\noindent
	Momentum of a System of\\ Particles\\
	
	\noindent
	Collisions
		\\\hspace*{1em}$\bullet$ Energy Conservation
		\\\hspace*{1em}$\bullet$ Momentum Conservation\\
	}
\parbox[c]{\rightcolumn}{
	\begin{enumerate}
		\item \emph{Momentum} is a property of a moving object. Its quantity is defined as the product of the object's mass and velocity
		
		\item The \emph{total} or \emph{net impulse} acting on an object is defined as the net force on the object times the amount of time during which the net force is acting on the object.
		
		\item The \emph{change in momentum} is equal to the net impulse and is independent of the coordinate system used to express $\vec{F}$, $\sum\vec{I}$, and $\vec{p}$.
		
		\item \emph{Conservation of Momentum:} If the net external impulse acting on a physical system is zero, then there is no change in the total linear momentum of that system; otherwise, the change in momentum is equal to the net external impulse.
		
		\item The momentum of a system of particles is the vector sum of the individual momenta.
		
		\item In a collision, the momentum of the system of objects (particles) remains constant if the external impulses are negligible. This is true whether energy is conserved during the collision or not.
	\end{enumerate}
}

\null